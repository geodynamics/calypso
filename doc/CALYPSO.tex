% Template article for preprint document class `elsart'
% SP 2001/01/05

\documentclass[12pt]{article}
\usepackage{graphicx,color,amssymb}
\usepackage{times,textpos}
\usepackage{geometry}
\usepackage{epstopdf,hyperref,float}
%\usepackage{epstopdf,hyperref,float,slashbox}
\usepackage[titletoc,toc,title]{appendix}

\hypersetup{
    colorlinks=true,
    citecolor=blue,
    linkcolor=blue,
    urlcolor=blue,
}
% Use the option doublespacing or reviewcopy to obtain double line spacing
% \documentclass[doublespacing]{elsart}

% The amssymb package provides various useful mathematical symbols
%\usepackage{amssymb}
\begin{document}
%%%%%%%%%%%%%%%%%%%%%%%%%%%%%%
%%% START OF CIG MANUAL COVER TEMPLATE %%%
%%%%%%%%%%%%%%%%%%%%%%%%%%%%%%
% This should be pasted at the start of manuals and appropriate strings entered at locations indicated with FILL.
% Be sure the TeX file includes the following packages.
% \usepackage{graphicx}
% \usepackage{times}
% \usepackage{textpos}

\definecolor{dark_grey}{gray}{0.3}
\definecolor{purple}{RGB}{163,0,107}

\newgeometry{vmargin={20mm,20mm},hmargin={20mm,10mm}}

%LINE 1%
{
\renewcommand{\familydefault}{\sfdefault}

\pagenumbering{gobble}
\begin{center}
\resizebox{\textwidth}{!}{\textcolor{dark_grey}{\fontfamily{\sfdefault}\selectfont
COMPUTATIONAL INFRASTRUCTURE FOR GEODYNAMICS (CIG)
}}

\hrule

%LINE 2%
\color{dark_grey}
\rule{\textwidth}{2pt}

%LINE 3%
\color{dark_grey}
% FILL: additional organizations
% e.g.: {\Large Organization 1\\Organization 2}
{\Large }
\end{center}

%COLOR AND CODENAME BLOCK%
\begin{center}
\resizebox{\textwidth}{!}{\colorbox
% FILL: color of code name text box
% e.g. blue
{purple}{\fontfamily{\rmdefault}\selectfont \textcolor{white} {
% FILL: name of the code
% You may want to add \hspace to both sides of the codename to better center it, such as:
% \newcommand{\codename}{\hspace{0.1in}CodeName\hspace{0.1in}}
\hspace{0.1in}Calypso\hspace{0.05in}
}}}
\end{center}

%MAIN PICTURE%
\begin{textblock*}{0in}(1.0in,0.55in)
% FILL: image height
% e.g. height=6.5in
\begin{center}
\vspace{.1in}
\includegraphics[height=4.2in]
% FILL: image file name
% e.g. cover_image.png
{Images/Interior2.pdf}
\end{center}
\end{textblock*}

%USER MANUAL%
\color{dark_grey}
\hfill{\Huge \fontfamily{\sfdefault}\selectfont User Manual \\
% FILL: manual version
% e.g. 1.0
\raggedleft \huge \fontfamily{\sfdefault}\selectfont Version {2.0}\\}

%AUTHOR(S) & WEBSITE%
\null
\vfill
\color{dark_grey}
\Large \hfill {\raggedleft \fontfamily{\sfdefault}\selectfont
% FILL: author list
% e.g. Author One\\Author Two\\Author Three\\
% be sure to have a newline (\\) after the final author
Hiroaki Matsui \\
}
{\fontfamily{\sfdefault}\selectfont www.geodynamics.org}

%\hrule

%LINE%
\color{dark_grey}
\rule{\textwidth}{2pt}

}

\pagebreak
\restoregeometry
\pagenumbering{arabic}

%%%%%%%%%%%%%%%%%%%%%%%%%%%%%%
%%%   END OF CIG MANUAL COVER TEMPLATE    %%%
%%%%%%%%%%%%%%%%%%%%%%%%%%%%%%

\input{tex_src/mathdef.tex}
%\maketitle
%\input{captions.txt}
%
%

\section*{Preface}
Calypso is a program package of magnetohydrodynamics (MHD) simulations in a rotating spherical shell for geodynamo problems. This package consists of the simulation program, preprocessing program, post processing program to generate field data for visualization programs, and several small utilities. The simulation program runs on parallel computing systems using MPI and OpenMP parallelization.

\newpage

\tableofcontents

%
\newpage
\section{Introduction}
\label{section:introduction}
Calypso is a program package for magnetohydrodynamics (MHD) simulations in a rotating spherical shell for geodynamo problems. This package consists of the simulation program, preprocessing program, post processing program to generate field data for visualization programs, and several small utilities. The simulation program runs on parallel computing systems using MPI and OpenMP parallelization.

Calypso solves the equations that govern convection and magnetic-field generation in a rotating spherical shell. Flow is driven by thermal or compositional buoyancy in a Boussinesq fluid. Calypso also support various boundary conditions (e.g. fixed temperature, heat flux, composition, and compositional flux), and permits a conductive and rotatable inner core. Results are written as spherical harmonics coefficients, Gauss coefficients for the region outside of the fluid shell, and field data in Cartesian coordinate for easily visualization with a number of visualization programs.

This user guide describes the essentials of the magnetohydrodynamics theory and equations behind Calypso, and provides instructions for the configuration and execution of Calypso.

\section{History}
\label{sec:history}
Calypso has its origins in two earlier projects. One is a dynamo simulation code written by Hiroaki Matsui in 1990's using a spectral method. This code solves for the poloidal and toroidal spectral coefficients, like Calypso, but it calculates the nonlinear terms in the spectral domain using a parallelization for SMP architectures. The other project is the thermal convection version of GeoFEM, which is Finite Element Method (FEM) platform for massively parallel computational environment, originally written by Hiroshi Okuda in 2000. Under GeoFEM Project, Lee Chen developed cross sectioning, iso-surfacing, and volume rendering modules for data visualization for parallel computations.. 

Hiroaki Matsui was responsible for adding routines to GeoFEM to perform magnetohydrodynamics simulation in a rotating frame. In 2002 this code successfully performed dynamo simulations in a rotating spherical shell using insulating magnetic boundary conditions.  The following year Matsui implemented a subgrid scale (SGS) model in the FEM dynamo model in collaboration with Bruce Buffett. A module to solve for double diffusive convection was added to the FEM dynamo model by Hiroaki Matsui in 2009.

Progress in understanding the role of subgrid scale models in magnetohydrodynamic simulations relies on quantitative estimates for the transfer of energy between spatial scales. This information is most easily obtained from a spherical harmonic expansion of the simulation results, even when the simulation is performed by FEM. Hiroaki Matsui implemented the spherical harmonic transform in 2007 using a combination of MPI and OpenMP, and later included the spherical harmonic transform routines into his old dynamo code to create Calypso. Additional software in the program package for visualization is based on data formats from the FEM model. In addition, the control parameter file format is adapted from the input formats used in GeoFEM.

Calypso Ver. 1.0 supports the following features and capabilities
%
\begin{itemize}
\item Magnetohydrodynamics simulation for a Boussinesq fluid in a rotating spherical shell.
\item Convection driven by thermal and compositional buoyancy.
\item Temperature or heat flux is fixed at boundaries
\item Composition or compositional flux is fixed at boundaries
\item Non-slip or free-slip boundary conditions
\item Outside of the fluid shell is electrically insulated or pseudo vacuum boundary.
\item A conductive inner core with the same conductivity as the surrounding fluid
\item A rotating inner core driven by the magnetic and viscous torques.
\end{itemize}
%
%
\subsection{Updates for Ver 1.1}
In Version 1.1, a number of bug fixes and additional comments for Doxygen are completed. The following large bugs are fixed:
%
\begin{itemize}
\item \verb|configure| command is updated to find appropriate GNU make command. (see Section \ref{sec:requirements})
\item Label for radial grid type in the file \verb|ctl_sph_shell| \verb|raidal_grid_type_ctl| is changed to \verb|radial_grid_type_ctl|. If the old name is used in the control file, program \verb|gen_sph_grid| will crash.
\end{itemize}
%

And, the following features are implemented
\begin{itemize}
\item New ordering is used for spherical harmonics data to reduce communication time. The old version of spectrum indexing data, which is generated by \verb|gen_sph_grids| in Ver. 1.0 is also supported in Ver. 1.1.
\item Evaluation of Coriolis term is updated. Now, Adams-Gaunt integrals are evaluated in the initialization process in the simulation program \verb|sph_mhd|, so the data file for Adams-Gaunt integrals which is made by \verb|gen_sph_grids|  is not required.
\item Add a program \verb|sph_add_initial_field|. to modify existed initial field data. This program is used to modify or add new fields in spectrum data. (See Section \ref{sec:add_initial_field}.)
\item Heat and composition source terms are implemented. These source terms are fixed with time, and defined as spectrum data. The source terms are defined by using initial field generation program \\ \verb|sph_initial_field| or \verb|sph_add_initial_field|. (See section \ref{sec:sph_initial_field} and  \ref{sec:add_initial_field}.)
\item The boundary conditions for temperature and composition can be defined by using spherical harmonics coefficients. (i.e. inhomogeneous boundary conditions can be applied.) These boundary conditions are defined by using single external data file. (See Section \ref{sec:boundary_file})
\end{itemize}

%
\subsection{Updates for Ver 1.2}
In Version 1.2, the following features are implemented:
\begin{itemize}
\item To reduce the number of calculation, Legendre transform is calculated with taking account to the symmetry with respect to the equator. Time for Legendre transform is approximately half of that in Ver 1.1.
\item BLAS library can be used for the Legendre transform optionally.
\item Cross sectioning and isosurfacing module are newly implemented. These modules are re-written by Fortran90 from the parallel sectioning modules in GeoFEM by Lee Chen in C, and some features are added for visualizations of geodynamo simulations. See section \ref{section:PSF} and \ref{section:ISO}.
\item Initial data assemble program \verb|assemble_mhd| is parallelized. This program can perform with any number of MPI processes, but we recommend to run the program with {\bf one} process or the same number of processes as the number of subdomains for the target configuration which is defined by \verb|num_new_domain_ctl|. See section \ref{sec:add_initial_field}.
\item The time and time step information in the restart data can be modifield by  \verb|assemble_mhd|. See section \ref{sec:add_initial_field}
\end{itemize}

\subsection{Updates for Ver 2.0}
In Version 2.0, there are a number of changes as;
\begin{itemize}
\item Performance is improved. Ver. 2.0 is approximately 30\% faster than the Ver. 1.2.
\item Drop the build process using {\bf Cmake}. Please use {\tt Configure} command for the configuration to build.
\item Number of array information is not required to define "array" block in control files.
\item Block layout of the control file is changed for spatial resolution and spherical harmonic data IO. See Section \ref{section:resolution_block}

\item Using MPI-IO for data IO to generate a single date file from MPI processes
\item Include interface to zlib library (\url{https://www.zlib.net}) for data IO with data compression. zlib is pre-installed in MacOS and most of Linux distributions.
\item For easier data handling, Calypso now support various format of data IO (ascii, binary, gzipped ascii, and gzipped binary) through MPI-IO.
\item Include spherical harmonics index generator into simulation program. Consequently, we can start program without preprocessing.
\item Modules to generate longitudinal average data is included into the simulation program.
\item Due to the performance drops on massively parallel environment, spectrum data \\
{\tt [picked\_sph\_prefix].dat} is splitted into the data files for each spherical harmonics mode {\tt [picked\_sph\_prefix]\_l[degree]\_m[order][c/s].dat}. Consequently, the file prefix [picked\_sph\_prefix] is recommends to includer subdirectory and that files are saved undere a subdirectory. (See Section \ref{sec:pickup_spectr_ctl} and tests under \verb|tests/Dynamobench_case2|.)
\item The monitor data files can also be compressed by zlib libraly. These compressed data files can be decompressed by {\tt gunzip} command.
\item In internally, Fortran structures to reduce global instances.
\end{itemize}






\section{Acknowledgements}
\label{section:acknowledgements}
Calypso was primarily developed by Dr. Hiroaki Matsui in collaboration with Prof. Bruce Buffett at the University of California, Berkeley. The following NSF grants supported the development of Calypso, 
%
\begin{itemize}
\item B.A. Buffett, NSF EAR-0509893; Models of sub-grid scale turbulence in the Earth's core and the geodynamo; 2005 - 2007.
\item B.A. Buffett and D. Lathrop,  NSF EAR-0652882; CSEDI Collaborative Research: Integrating numerical and experimental geodynamo models, 2007 - 2009
\item B.A. Buffett, NSF EAR-1045277; Development and application of turbulence models in numerical geodynamo simulations ;  2010 - 2012
\end{itemize}
%

\section{Citation}
\label{section:citation}

Computational Infrastructure for Geodynamics (CIG) and the Calypso developers are making the source code to Calypso available to researchers in the hope that it will aid their research and teaching. A number of individuals have contributed a significant amount of time and energy into the development of Calypso. We request that you cite the appropriate papers and make acknowledgements as necessary. The Calypso development team asks that you cite the following papers:

Matsui, H., E. King, and B.A. Buffett, Multi-scale convection in a geodynamo simulation with uniform heat flux along the outer boundary, {\it Geochemistry, Geophysics, Geosystems}, {\bf 15}, 3212 -- 3225, 2014.


\newpage
\section{Model of Simulation}
\subsection{Governing equations}
%
\begin{figure}[htbp]
\begin{center}
\includegraphics*[width=130mm]{Images/Spherical_shell}
\end{center}
\caption{Rotating spherical shell modeled on the Earth's outer core.}
\label{fig:shell}
\end{figure}
%
This model performs a magnetohydrodynamics (MHD) simulation in a rotating spherical shell modeled on the Earth's outer core (see Figure \ref{fig:shell}). We consider a spherical shell from the inner core boundary (ICB) to the core mantle Boundary (CMB) in a rotating frame which constantly rotates with angular velocity $\bvec{\Omega} = \Omega \hat{z}$. The fluid shell is filled with a conductive fluid with constant diffusivities (kinematic viscosity $\nu$, magnetic diffusivity $\eta$, thermal diffusivity $\kappa_{T}$, and compositional diffusivity $\kappa_{C}$). The inner core ($0 < r < r_{i}$) is solid, and may be considered an electrical insulator  or may have the same conductivity as the outer core. We assume that the region outside of the core is an electrical insulator. The rotating spherical shell is filled with Boussinesq modeled fluid. The governing equations of the MHD dynamo problem are the following,
%
\begin{eqnarray}
\frac{\partial \bvec{u}}{\partial t} + \left(\bvec{\omega} \times \bvec{u}\right)
 & = & - \nabla \left(P+\frac{1}{2}u^{2} \right) -\nu \nabla \times \nabla \times \bvec{u}
\nonumber \\
 & &  - 2 \Omega \left(\hat{z} \times \bvec{u} \right)
     + \left( \frac{\rho}{\rho_{0}} \bvec{g} \right)
     + \frac{1}{\rho_{0}} \left(\bvec{J} \times \bvec{B} \right),
\label{eq:momentum} \\
%
 \frac{\partial \bvec{B}}{\partial t}
 & = & -\eta \nabla \times \nabla \times \bvec{B}
       + \nabla \times \left(\bvec{u} \times \bvec{B} \right),
\label{eq:induction} \\
%
\frac{\partial T}{\partial t} + \left(\bvec{u} \cdot \nabla \right) T
 & = & \kappa_{T} \nabla^{2} T + q_{T},
\label{eq:heat} \\
%
\frac{\partial C}{\partial t} + \left(\bvec{u} \cdot \nabla \right) C
 & = & \kappa_{C} \nabla^{2} C + q_{C},
\label{eq:composition} \\
\nabla \cdot \bvec{u} & = & \nabla \cdot \bvec{B} = 0,
\label{eq:conservation} \\
%
\bvec{\omega} & = & \nabla \times \bvec{u},
\label{eq:def_vorticity}
\end{eqnarray}
%
and
\begin{eqnarray}
\bvec{J} & = & \frac{1}{\mu_{0}} \nabla \times \bvec{B},
\label{eq:def_current}
\end{eqnarray}
%
where, $\bvec{u}$, $\bvec{\omega}$, $P$, \bvec{B}, \bvec{J}, $T$, $C$, $q_{T}$, and $q_{C}$ are the velocity, vorticity, pressure, magnetic field, current density, temperature, compositional variation, heat source, and source of light element, respectively. Coefficients in the governing equations are the kinetic viscosity $\nu$, thermal diffusivity $\kappa_{T}$, compositional diffusivity $\kappa_{C}$, and magnetic diffusivity $\eta$. The density $\rho$ is written as a function of $T$, $C$, average density $\rho_{0}$, thermal expansion $\alpha_{T}$, and density ratio of light element to main composition $\alpha_{C}$,
%
\begin{eqnarray}
\rho & = & \rho_{0} \left[1 - \alpha_{T} \left( T - T_{0} \right)
                               - \alpha_{C} \left( C - C_{0} \right) \right]
\label{eq:def_density}
\end{eqnarray}
%
In Calypso, the vorticity equation and divergence of the momentum equation are used for solving $\bvec{u}$, $\bvec{\omega}$, and $P$ as,
\begin{eqnarray}
\frac{\partial \bvec{\omega}}{\partial t} + \nabla \times \left(\bvec{\omega} \times \bvec{u}\right)
 & = & -\nu \nabla \times \nabla \times \bvec{\omega}
     - 2 \Omega \nabla \times \left(\hat{z} \times \bvec{u} \right)
\nonumber \\
 & & + \nabla \times \left( \frac{\rho}{\rho_{0}} \bvec{g} \right)
     + \frac{1}{\rho_{0}} \nabla \times \left(\bvec{J} \times \bvec{B} \right),
\label{eq:vorticity_eq}
\end{eqnarray}
%
and
\begin{eqnarray}
\nabla \cdot \left(\bvec{\omega} \times \bvec{u}\right)
 & = & -\nabla^{2} \left(P+\frac{1}{2}u^{2} \right) - 2 \Omega \nabla \cdot \left(\hat{z} \times \bvec{u} \right)
\nonumber \\
 & & + \nabla \cdot \left( \frac{\rho}{\rho_{0}} \bvec{g} \right)
     + \frac{1}{\rho_{0}} \nabla \cdot \left(\bvec{J} \times \bvec{B} \right)
\label{eq:div_momentum}
\end{eqnarray}


\subsection{Spherical harmonics expansion}
In Calypso, fields are expanded into spherical harmonics. A scalar field (for example, temperature $T(r, \theta, \phi)$) is expanded as
%
\begin{eqnarray}
T(r, \theta, \phi) &=& \sum_{l=0}^{L} \sum_{m=-l}^{l} T_{l}^{m}(r) Y_{l}^{m}(\theta,\phi),
\label{eq:scalar_expansion}
\end{eqnarray}
where  $Y_{l}^{m}$ are the spherical harmonics. Solenoidal fields (e.g. velocity $\bvec{u}$, vorticity $\bvec{\omega}$, magnetic field $\bvec{B}$, and current density $\bvec{J}$) are decomposed into poloidal and toroidal components. For example, the magnetic field is described as 
\begin{eqnarray}
\bvec{B}(r, \theta, \phi) & = & \sum_{l=1}^{L} \sum_{m=-l}^{l} 
\left( \bvec{B}_{Sl}^{\ m} + \bvec{B}_{Tl}^{\ m} \right),
\label{eq:poloidal_toroidal}
\end{eqnarray}
where
\begin{eqnarray}
\bvec{B}_{Sl}^{\ m}(r, \theta, \phi) & = & \nabla \times \nabla \times \left( B_{Sl}^{\ m}(r) Y_{l}^{m}(\theta,\phi) \hat{r} \right),
\label{eq:def_poloidal} \\
\bvec{B}_{Tl}^{\ m}(r, \theta, \phi) & = & \nabla \times \left( B_{Tl}^{\ m}(r) Y_{l}^{m}(\theta,\phi) \hat{r} \right).
\label{eq:def_toroidal}
\end{eqnarray} 

The spherical harmonics are defined as real functions. $P_{l}^{m} \cos \left( m\phi \right)$ is assigned for positive $m$, $P_{l}^{m} \sin \left( m\phi \right)$ is assigned for negative $m$, where $P_{l}^{m}$ are Legendre polynomials. Because Schmidt quasi normalization is used for the Legendre polynomials $P_{l}^{m}$, the orthogonality relation for the spherical harmonics is 
%
\begin{eqnarray}
\int Y_{l}^{m} Y_{l'}^{m'} \sin \theta d\theta d\phi &=& 4\pi \frac{1}{2l+1} \delta_{ll'}\delta_{mm'},
\label{eq:sph_normalization}
\end{eqnarray}
%
where, $\delta_{ll'}$ is Kronecker delta.

\subsubsection{Forward spherical harmonics transform}
Calypso uses spherical harmonics expansion method (or so-called pseudo-spectrum method). In the present method, linear terms are solved by using the coefficients of the spherical harmonics expansion, and nonlinear terms are calculated in the physical grid space. Consequently, fields are transformed to grids space by the backward spherical transform using equations (\ref{eq:scalar_expansion}) to (\ref{eq:def_toroidal})., and nonlinear terms are transferred into the spherical harmonics coefficients by forward spherical harmonics transform.  

The nonlinear terms (advection, Lorentz force, and induction, and heat flux terms) are evaluated in the physical space $(r, \theta, \phi)$ and are not solenoidal fields (i.e. the divergence of the nonlinear terms are not to be 0). The forward spherical transform for a non-solenoidal vector field $\bvec{A}$ can be expressed by the radial, horizontal divergence, and toroidal components as
%
\begin{eqnarray}
A_{Rl}^{\ m} (r,t)  &=& N_{l}^{-1} \int \frac{l \left( l+1 \right)}{r^{2}} Y_{l}^{m} A_{r}(r, \theta, \phi) d\sigma, 
\label{eq:radial_forward} \\
A_{Hl}^{\ m} (r,t) 
 &=& N_{l}^{-1} \int  \frac{1}{r} \left[ \frac{\partial Y_{l}^{m} }{\partial \theta} A_{\theta}(r, \theta, \phi)
 + \frac{1}{\sin \theta} \frac{\partial Y_{l}^{m} }{\partial \phi} A_{\phi}(r, \theta, \phi) \right] d\sigma,
\label{eq:horizontal_div_forward}
\end{eqnarray}
and, 
\begin{eqnarray}
A_{Tl}^{\ m} (r,t) 
 &=& N_{l}^{-1} \int  \frac{1}{r} \left[ \frac{1}{\sin \theta}  \frac{\partial Y_{l}^{m} }{\partial \phi} A_{\theta}(r, \theta, \phi)
 - \frac{\partial Y_{l}^{m} }{\partial \theta} A_{\phi}(r, \theta, \phi) \right] d\sigma,
\label{eq:toroidal_forward} \\
\end{eqnarray}
%
where, $d \sigma = \sin \theta d \theta d\phi$ is the integration over a sphere. 
The non-solenoidal vector $\bvec{A}$ can also describes the spherical harmonics coefficients for the poloidal $A_{Sl}^{\ m}$, toroidal $A_{Tl}^{\ m}$, and scalar potential $\varphi_{l}^{m}$ as
%
\begin{eqnarray}
\bvec{A} = \sum \left[ -\nabla \varphi_{l}^{m} 
+ \nabla \times \nabla \times \left(A_{Sl}^{\ m} \hat{r} \right) 
+  \nabla \times \left(A_{Tl}^{\ m} \hat{r} \right) \right].
\end{eqnarray}
%
Consequently, the poloidal and toroidal components of the rotation of $\bf{A}$ can be described by
%
\begin{eqnarray}
\left( \nabla \times \bf{A} \right)_{Sl}^{\ m} &=& A_{Tl}^{\ m} \\
\left( \nabla \times \bf{A} \right)_{Tl}^{\ m}
 &=& \frac{\partial^{2} A_{Sl}^{\ m}}{\partial r^{2}} - \frac{l \left( l+1 \right)}{r^{2}} A_{Sl}^{\ m} 
 \nonumber \\
 &=& - A_{Vl}^{\ m} + \frac{\partial A_{Hl}^{\ m}}{\partial r},
\end{eqnarray}
%
and, the divergence of $\bf{A}$ can be obtained by
%
\begin{eqnarray}
\left( \nabla \cdot \bf{A} \right)_{Tl}^{\ m}
 &=& - \frac{\partial^{2} \varphi_{l}^{m}}{\partial r^{2}} 
    - \frac{2}{r} \varphi_{l}^{m} 
    + \frac{l \left( l+1 \right)}{r^{2}} \varphi_{l}^{m} 
 \nonumber \\
 &=& \frac{\partial A_{Vl}^{\ m}}{\partial r} + \frac{2}{r} A_{Vl}^{\ m}
 - \frac{l \left( l+1 \right)}{r^{2}} A_{Hl}^{\ m}.
\end{eqnarray}
%

\subsection{Evaluation of Coriolis term}
The curl of the Coriolis force $-2\Omega \nabla \times \left(\hat{z} \times \bvec{u} \right)$ is evaluated in the spectrum space using the triple products of the spherical harmonics. These 3j-symbols (or Gaunt integral $G_{Lll'}^{Mmm'}$ and Elsasser integral $E_{Lll'}^{Mmm'}$) are written as
%
\begin{eqnarray}
G_{Lll'}^{Mmm'} & = & \int Y_{L}^{M} Y_{l}^{m} Y_{l'}^{m'}
\sin\theta d\theta d\phi,
\label{eq:gaunt} \\
%
E_{Lll'}^{Mmm'} & = & \int Y_{L}^{M} \left (
   \frac{\partial Y_{l}^{m}}{\partial \theta} \frac{\partial Y_{l'}^{m'}}{\partial\phi}
 - \frac{\partial Y_{l}^{m}}{\partial \phi} \frac{\partial Y_{l'}^{m'}}{\partial \theta}
\right) d\theta d\phi.
\label{eq:elsasser}
\end{eqnarray}
%
The Gaunt integral $1/(4\pi) G_{Lll'}^{Mmm'}$ and Elsasser integral $1/(4\pi) E_{Lll'}^{Mmm'}$ for the Coriolis terms are evaluated in the simulation program.

\subsection{Radial discretization}
In Calypso, spherical harmonic coefficients $f_{l}^{m}(r, t)$ are discretized by the second order finite difference method. A non-equidistant grids is widely used in geodynamo simulations to resolve boundary layers. Now, considering the $n$-th grid point at $r = r_{n}$ and neighboring grid points. The Taylor expansion of the spherical harmonic coefficients at $n+1$-th and $n-1$-th points $f_{l}^{m}(r_{n+1}, t)$ and $f_{l}^{m}(r_{n-1}, t)$ can be described by
%
\begin{eqnarray}
\left( \begin{array}{c}
f_{l}^{m}(r_{n}) \\
f_{l}^{m}(r_{n-1}) \\
f_{l}^{m}(r_{n+1})
\end{array} \right)
 & = & 
A_{n}
\left( \begin{array}{c}
f_{l}^{m}(r_{n}) \\
\partial_r  f_{l}^{m}(r_{n}) \\
\partial_rr f_{l}^{m}(r_{n}) \\
\end{array} \right),
\end{eqnarray}
%
where, $A_{n}$ is
%
\begin{eqnarray}
A_{n} = 
\left( \begin{array}{ccc}
1 & 0 & 0  \\
1 & r_{n} - r_{n-1} & \left(r_{n} - r_{n-1} \right)^2 / 2 \\
1 & r_{n+1} - r_{n} & \left(r_{n+1} - r_{n} \right)^2 / 2
\end{array} \right),
\end{eqnarray}
%
The first and second derivatives are expressed by
%
\begin{eqnarray}
\left( \begin{array}{c}
f_{l}^{m}(r_{n}) \\
\partial_r  f_{l}^{m}(r_{n}) \\
\partial_rr f_{l}^{m}(r_{n}) \\
\end{array} \right)
 & = & 
A_{n}^{-1}
\left( \begin{array}{c}
f_{l}^{m}(r_{n-1}) \\
f_{l}^{m}(r_{n}) \\
f_{l}^{m}(r_{n+1}) \\
\end{array} \right).
\end{eqnarray}
%

At the boundaries, the first radial derivative of the coefficients $\partial_{r} f_{l}^{m}(r_{1})$ is used  instead of the grid outside of the boundaries. For example, $A_{n}$ at the inner boundary $n = 1$ can be written by using the first differenciation as 
%
\begin{eqnarray}
\left( \begin{array}{c}
f_{l}^{m}(r_{1}) \\
\partial_{r} f_{l}^{m}(r_{1}) \\
f_{l}^{m}(r_{2}) \\
\end{array} \right)
 & = & 
A_{1}
\left( \begin{array}{c}
f_{l}^{m}(r_{1}) \\
\partial_r  f_{l}^{m}(r_{1}) \\
\partial_{rr} f_{l}^{m}(r_{1}) \\
\end{array} \right).
\end{eqnarray}
%
where, $A_{1}$ is
%
\begin{eqnarray}
A_{1} = 
\left( \begin{array}{ccc}
1 & 0 & 0  \\
0 & 1 & 0  \\
1 & r_{N+1} - r_{N} & \left(r_{N+1} - r_{N} \right)^2 / 2
\end{array} \right),
\end{eqnarray}
%
When $f_{l}^{m}(r_{1})$ is fixed at the boundary, the second derivative $\partial_{rr} f_{l}^{m}(r_{1})$ is not used. When the boundary condition for $\partial_r  f_{l}^{m}(r_{1})$ is given, the second derivative  $\partial_{rr} f_{l}^{m}(r_{1})$ is obtained by the boundary condition and $f_{l}^{m}(r_{1})$.

\subsection{Boundary conditions}
Calypso currently supports the following boundary conditions for velocity $\bvec{u}$, magnetic field $\bvec{B}$, temperature $T$, and composition variation $C$. These boundary conditions are defined in the control file \verb|control_MHD|.

\subsubsection{Non-slip boundary}
The velocity $\bvec{u}$ is set to be 0 at the boundary. For poloidal and toroidal coefficients of velocity, $U_{Sl}^{\ m}(r)$ and $U_{Tl}^{\ m}(r)$, the boundary condition can be described as
%
\begin{eqnarray}
U_{Sl}^{\ m}(r) & = & \frac{\partial U_{Sl}^{\ m}}{\partial r} = 0,
\nonumber
\end{eqnarray}
%
and 
%
\begin{eqnarray}
U_{Tl}^{\ m}(r) & = & 0.
\nonumber
\end{eqnarray}
%
\subsubsection{Free-slip boundary}
For a free slip boundary, shear stress and radial flow vanish at the boundary. The boundary condition for poloidal and toroidal coefficients are described as
%
\begin{eqnarray}
U_{Sl}^{\ m}(r) = \frac{\partial^2}{\partial r^2} \left( \frac{1}{r} U_{Sl}^{\ m}(r) \right) & = & 0,
\nonumber
\end{eqnarray}
%
and 
%
\begin{eqnarray}
\frac{\partial}{\partial r} \left( \frac{1}{r^2} U_{Tl}^{\ m}(r) \right) & = & 0.
\nonumber
\end{eqnarray}
%
\subsubsection{Fixed rotation rate}
If the boundary rotates with a rotation vector $\bvec{\Omega}_{b} = \left(\Omega_{bx}, \Omega_{by}, \Omega_{bz}\right)$, the boundary conditions for poloidal and toroidal coefficients are described as
%
\begin{eqnarray}
U_{Sl}^{\ m}(r) & = & \frac{\partial U_{Sl}^{\ m}}{\partial r} = 0,
\nonumber \\
U_{T1}^{\ 1s}(r) & = & r^{2} \Omega_{by},
\nonumber \\
U_{T1}^{\ 0}(r) & = &  r^{2} \Omega_{bz},
\nonumber \\
U_{T1}^{\ 1c}(r) & = & r^{2} \Omega_{bx},
\nonumber
\end{eqnarray}
%
and 
%
\begin{eqnarray}
U_{Tl}^{\ m}(r) & = & 0 \mbox{ for } l > 2.
\nonumber
\end{eqnarray}
%
\subsubsection{Fixed homogenous temperature}
When a constant temperature $T_{b}$ is is applied, the spherical harmonic coefficients are
%
\begin{eqnarray}
T_{0}^{0}(r) & = &  T_{b},
\nonumber
\end{eqnarray}
%
and 
%
\begin{eqnarray}
T_{l}^{m}(r) & = & 0 \mbox{ for } l > 1.
\nonumber
\end{eqnarray}
%
\subsubsection{Fixed homogenous heat flux}
A constant heat flux is imposed by setting the radial temperature gradient to $F_{Tb}$. The spherical harmonic coefficients are
%
\begin{eqnarray}
\frac{\partial T_{0}^{0}}{\partial r} & = &  F_{Tb},
\nonumber
\end{eqnarray}
%
and 
%
\begin{eqnarray}
\frac{\partial T_{l}^{m}}{\partial r} & = & 0 \mbox{ for } l > 1.
\nonumber
\end{eqnarray}
%
\subsubsection{Fixed composition}
When a constant composition $C_{b}$ is applied, the spherical harmonic coefficients are 
%
\begin{eqnarray}
C_{0}^{0}(r) & = & C_{b},
\nonumber
\end{eqnarray}
%
and 
%
\begin{eqnarray}
C_{l}^{m}(r) & = & 0 \mbox{ for } l > 1.
\nonumber
\end{eqnarray}
%
\subsubsection{Fixed composition flux}
A constant composition flux is imposed by setting the radial composition gradient to $F_{Cb}$. The spherical harmonic coefficients are
%
\begin{eqnarray}
\frac{\partial C_{0}^{0}}{\partial r} & = &  F_{Cb},
\nonumber
\end{eqnarray}
%
and 
%
\begin{eqnarray}
\frac{\partial C_{l}^{m}}{\partial r} & = & 0 \mbox{ for } l > 1.
\nonumber
\end{eqnarray}
%
\subsubsection{Connection to the magnetic potential field}
If the regions outside the fluid shell are assumed to be electrical insulators, current density vanishes in the electric insulator
 %
\begin{eqnarray}
\bvec{J}_{ext} &= & 0,
\nonumber
\end{eqnarray}
%
where the suffix ${}_{ext}$ indicates fields outside of the fluid shell. At the boundaries of the fluid shell, the magnetic field $\bvec{B}_{fluid}$, current density $\bvec{J}_{fluid}$ , and electric field $\bvec{E}_{fluid}$ in the conductive fluid satisfy:
 %
\begin{eqnarray}
\left (\bvec{B}_{fluid} - \bvec{B}_{ext} \right)  = 0,
\nonumber \\
\left (\bvec{J}_{fluid} - \bvec{J}_{ext} \right)  \cdot \hat{r}   = 0,
\nonumber
\end{eqnarray}
and 
\begin{eqnarray}
\left (\bvec{E}_{fluid} - \bvec{E}_{ext} \right) \times \hat{r}  = 0,
\nonumber
\end{eqnarray}
%
where, $\hat{r}$ is the radial unit vector (i.e. normal vector for the spherical shell boundaries). 
Consequently, radial current density $\bvec{J} $ vanishes at the boundary as
  %
\begin{eqnarray}
\bvec{J} \cdot \hat{r}  = 0
 \mbox{ at } r = r_{i}, r_{o}
\nonumber
\end{eqnarray}
%

In an electrical insulator the magnetic field can be described as a potential field
 %
\begin{eqnarray}
\bvec{B}_{ext} = - \nabla W_{ext},
\nonumber
\end{eqnarray}
%
where $W_{ext}$ is the magnetic potential. The boundary conditions can be satisfied by  connecting the magnetic field in the fluid shell at boundaries to the potential fields. The magnetic field is connected to the potential field in an electrical insulator. At CMB ($r = r_{o}$), the boundary condition can be described by the poloidal and toroidal coefficients of the magnetic field as
%
\begin{eqnarray}
\frac{l}{r} B_{Sl}^{\ m}(r) & = & - \frac{\partial B_{Sl}^{\ m}}{\partial r},
\nonumber
\end{eqnarray}
%
and 
%
\begin{eqnarray}
B_{Tl}^{\ m}(r) & = & 0.
\nonumber
\end{eqnarray}
%

If the inner core is also assumed to be an insulator, the magnetic boundary conditions for ICB ($r = r_{i}$) can be described as
%
\begin{eqnarray}
\frac{l+1}{r} B_{Sl}^{\ m}(r) & = & \frac{\partial B_{Sl}^{\ m}}{\partial r},
\nonumber
\end{eqnarray}
%
and 
%
\begin{eqnarray}
B_{Tl}^{\ m}(r) & = & 0.
\nonumber
\end{eqnarray}
%

\subsubsection{Magnetic boundary condition for center}
If the inner core has the same conductivity as the outer core, we solve the induction equation for the inner core as for the outer core with the boundary conditions for the center. The poloidal and toroidal coefficients at center are set to 
%
\begin{eqnarray}
B_{Sl}^{\ m}(0) &=& B_{Tl}^{\ m}(0) = 0.
\nonumber
\end{eqnarray}
%

\subsubsection{Pseudo-vacuum magnetic boundary condition}
Under the pseudo-vacuum boundary condition, the magnetic field has only a radial component at the boundaries. Considering the conservation of the magnetic field, the magnetic boundary condition will be
%
\begin{eqnarray}
\frac{\partial}{\partial r}\left(r^{2} B_{r} \right) =  B_{\theta} = B_{\phi} = 0
 \mbox{ at } r = r_{i}, r_{o}.
\nonumber
\end{eqnarray}
%
The present boundary condition is also described by using the poloidal and toroidal coefficients as
%
\begin{eqnarray}
\frac{\partial B_{Sl}^{\ m}}{\partial r} & = &  B_{Tl}^{\ m}(r) = 0
 \mbox{ at } r = r_{i}, r_{o}.
\nonumber
\end{eqnarray}



\newpage
\section{Installation}


\subsection{Compiler Requirements}
Most of source code of Calypso are written in Fortran2003. Consequently, Fortran compiler with supporting fortran 2003 is required. We can obtain a number of information about Fortran from \url{http://fortranwiki.org/}, and you can also find a table of the supported features of Fortran 2003 standard at \url{http://fortranwiki.org/fortran/show/Fortran+2003+status}. In addition, C compiler is optionally required to us zlib support for compressed data IO. 

\subsection{Library Requirements}
\label{sec:requirements}
Calypso requires the following libraries.
\begin{itemize}
\item GNU make
\item MPI libraries (OpenMPI, MPICH, etc)
\item FFTPACK Ver 5.1D (\url{https://people.sc.fsu.edu/~jburkardt/f_src/fftpack5.1d/fftpack5.1d.html}). The source files for FFTPACK are included in {\tt src/EXTERNAL\_libs} directory.
\end{itemize}
Linux and Max OS X use GNU make as a default 'make' command, but some system (e.g. BSD or SOLARIS) does not use GNU make as default. \verb|configure| command searches and set correct GNU make command. MPI library such as OpenMPI (\url{https://www.open-mpi.org}) or MPICH (\url{https://www.mpich.org}) can be installed by the most of package manager.

In addition, the following environment and libraries can be used (optional).
\begin{itemize}
\item OpenMP 
\item BLAS
\item zlib (https://www.zlib.net)
\item FFTW version 3 (\url{http://www.fftw.org}) including Fortran wrapper
\item PARALLEL HDF5 (\url{https://support.hdfgroup.org/HDF5/PHDF5})  including Fortran wrapper.
\end{itemize}
Note: Calypso does NOT use MPI and OpenMP features in FFTW3. 

In the most of platforms, the Fourier transform by FFTW is faster than that by FFTPACK. 

Zlib is used for compressed data IO. Zlib is installed in most of UNIX platforms.

HDF5 is used for field data output with XDMF format instead of VTK format. The comparison of field data format is described in section ref{sec:VTK}. 

OpenMP is used for the parallelization under the shared memory. Better choice to use both MPI and OpenMP parallelization (so-called Hybrid parallelization) or only using MPI (so-called flat MPI) is depends on the computational platform and compiler. For example, flat MPI has much better performance on Linux cluster with Intel Xeon processors and with Intel fortran compiler, but Hybrid model has better performance on Hitachi SR24000 with Power 8 processors.

\subsubsection{Installation of required softwares for Linux}
GCC, the GNU Compiler Collection (\url{https://gcc.gnu.org}) is already installed in the most of Linux distributions. However, packages for development are need to be installed. For Ubuntu 20, for example, the required compilers and  libraries can be installed by using \verb|apt| command as following::
%
\begin{verbatim}
% sudo apt install pkg-config
% sudo apt install git
% sudo apt install gfortran
% sudo apt install libopenmpi-dev
% sudo apt install zlib1g
% sudo apt install zlib1g-dev
% sudo apt install libblas-dev
% sudo apt install libfftw3-dev
% sudo apt install libhdf5-openmpi-dev
\end{verbatim}

 
\subsubsection{Installation of required softwares for Mac OS}
For MacOS, any fortran compiler needs to be installed because Xcode does not have fortran compiler. The easiest way is installing GCC by using a package manager such as macports (\url{https://www.macports.org}) or homebrew (\url{https://brew.sh/index}). By using the Macports, 

\paragraph{IMacports}
The required compiler and packages can be installed as followings as an example using GCC9. GCC in Macports includes gfortran compiler.
%
\begin{verbatim}
% sudo port install gcc11
% sudo port install openmpi-gcc11
% sudo port install fftw-3 +gcc11 +openmpi
% sudo port install hdf5 +fortran +gcc11 +openmpi +threadsafe
\end{verbatim}

\paragraph{Homebrew}
The required compiler and packages can be installed as followings: GCC in Homebrew includes gfortran compiler.
%
\begin{verbatim}
% brew install gcc
% brew install open-mpi
% brew install fftw
% brew install lhdf5-mpi
\end{verbatim}


\subsection{Known problems}
\subsubsection*{FFTPACK and Intel compiler}
FFTPACK fails to compile with Intel fortran using the {\tt `-warn all'} option. Currently the {\tt `-warn all'} option is excluded by Makefile when FFTPACK is compiled.

\subsubsection*{XL fortran}
In XL fortran, preprocessor options is not specified by \verb|-D...|, but \verb|-Wf, '-D...'|. Please edit preprocessor macro option \verb|F90CPPFLAGS| in \verb|work/Makefile| by an editor.

\subsubsection*{Cross compiler support}
{\tt configure} command in Calypso does not support cross compilation. If you want to compile with a cross compiler, please set the variables in Makefile manually (see section \ref{section:no_configure})

\subsection{Directories}

The top directory of Calypso (ex. \verb|[CALYPSO_HOME]|) contains the following directories.
\begin{verbatim}
% cd [CALYPSO_HOME]
% ls
CMakeLists.txt	Makefile.in	configure.in	examples
INSTALL		bin		doc		src
LICENSE		configure	doxygen		work

\end{verbatim}

\begin{description}
\item{\tt bin:      } directory for executable files
\item{\tt cmake:    } directory for cmake configurations
\item{\tt cmake:    } directory for document generated by doxygen
\item{\tt doc:      } documentations
\item{\tt examples: } examples
\item{\tt src:      } source files
\item{\tt work:     } work directory. Compile is done in this directory.
\end{description}

\subsection{Doxygen}
Doxygen (\url{http://www.doxygen.org}) is an powerful document generation tool from source files. We only save a configuration file in this directory because thousands of html files generated by doxygen. The documents for source codes are generated by the following command:
% 
\begin{verbatim}
% cd [CALYPSO_HOME]/doxygen
% doxygen ./Doxyfile_CALYPSO
\end{verbatim}
%
The html documents can see by opening \verb|[CALYPSO_HOME]/doxygen/html/index.html|.  Automatically generated documentation is also available on the CIG website at \url{http://www.geodynamics.org/cig/software/calypso/}.

\subsection{Install using {\tt configure} command }
\subsubsection{Configuration using {\tt configure} command }
Calypso uses the configure script for configuration to install. The simplest way to install programs is the following process in the top directory of Calypso.
% 
\begin{verbatim}
%pwd
[CALYPSO_HOME]
% ./configure
...
% make
...
% make install
\end{verbatim}
%
After the installation, object modules can be deleted by the following command;
% 
\begin{verbatim}
% make clean
\end{verbatim}
%

{./configure} generates a Makefile in the current directory.  Available options for {\tt configure} can be checked using the {\tt ./configure --help} command. The following options are available in the {\tt configure} command.
%
{\small
\begin{verbatim}
Optional Features:
  --disable-option-checking  ignore unrecognized --enable/--with options
  --disable-FEATURE       do not include FEATURE (same as --enable-FEATURE=no)
  --enable-FEATURE[=ARG]  include FEATURE [ARG=yes]
  --enable-fftw3          Use fftw3 library 
 Optional Packages:
  --with-PACKAGE[=ARG]    use PACKAGE [ARG=yes]
  --without-PACKAGE       do not use PACKAGE (same as --with-PACKAGE=no)
  --with-hdf5=yes/no/PATH full path of h5pcc for parallel HDF5 configuration
  --with-blas=<lib>       use BLAS library <lib>
  --with-zlib=DIR root directory path of zlib installation defaults to
                    /usr/local or /usr if not found in /usr/local
  --without-zlib to disable zlib usage completely

Some influential environment variables:
  CC          C compiler command
  CFLAGS      C compiler flags
  LDFLAGS     linker flags, e.g. -L<lib dir> if you have libraries in a
              nonstandard directory <lib dir>
  LIBS        libraries to pass to the linker, e.g. -l<library>
  CPPFLAGS    (Objective) C/C++ preprocessor flags, e.g. -I<include dir> if
              you have headers in a nonstandard directory <include dir>
  FC          Fortran compiler command
  FCFLAGS     Fortran compiler flags
  MPICC       MPI C compiler command
  MPIFC       MPI Fortran compiler command
  PKG_CONFIG  path to pkg-config utility
  CPP         C preprocessor
  FFTW3_CFLAGS
              C compiler flags for FFTW3, overriding pkg-config
  FFTW3_LIBS  linker flags for FFTW3, overriding pkg-config

\end{verbatim}
}
%
An example of usage of the configure command is the following;
\begin{verbatim}
% ./configure --prefix='/Users/matsui/local' \
? CFLAGS='-O -Wall -g' \
? PKG_CONFIG_PATH='/Users/matsui/local/lib/pkgconfig'
\end{verbatim}

At the end of the configuration, The following message can use to check if libraries can be refered correctly:

{\small
\begin{verbatim}
-----   Configuration summary   -------------------------------

host:        "x86_64-apple-darwin16.7.0"
host_alias:  ""
XL_FORTRAN:    ""
Use OpenMP ...        yes

Use BLAS ...          yes

Use FFTW3 ...         yes
Use parallel HDF5 ... yes

Use zlib at ...       yes

---------------------------------------------------------------
\end{verbatim}
}


\subsubsection{Compile}
Compile is performed using the {\tt make} command. The Makefile in the top directory is used to generate another Makefile in the {\tt work} directory, which is automatically used to complete the compilation. The object file and libraries are compiled in the {\tt work} directory. Finally, the executive files are assembled in {\tt bin} directory. You should find the following programs in the {\tt bin} directory.
%
\begin{description}
\item{\tt gen\_sph\_grids:    }\\
 Preprocessing program for data transfer for spherical harmonics transform
\item{\tt check\_sph\_grids:    }\\
 Check program for data communication for spherical harmonics transform
\item{\tt sph\_mhd:          }\\
 Simulation program
\item{\tt sph\_initial\_field: }\\
 Example program to generate initial field
\item{\tt sph\_add\_initial\_field: }\\
 Example program to add initial field in existing spectum data
\item{\tt sph\_snapshot:     }\\
 Data transfer from spectrum data to field data
\item{\tt sph\_dynamobench:  }\\
 Data processing for dynamo benchmark test by Christensen {\it et. al.} (2002)
\item{\tt assemble\_sph:     }\\
 Data transfer program to change number of subdomains.
\item{\tt sectioning:     }\\
 Generate cross section and isosurface from field data and FEM mesh data.
\item{\tt field\_to\_VTK:   }\\
 Data transfer program from field and FEM mesh data to VTK format.
\item{\tt psf\_to\_vtk:     }\\
 Data transfer program from section and isosurface data to VTK format.
\item{\tt t\_ave\_sph\_mean\_square:     }\\
 Time averaging program for the mean square data.
\item{\tt t\_ave\_picked\_sph\_coefs:     }\\
 Time averaging program for the picked spectrum data.
\item{\tt t\_ave\_nusselt:     }\\
Time averaging program for the Nusselt number data.
\item{\tt check\_sph\_grids:   }\\
                   Check program for tests.
\item{\tt make\_f90depends:  }\\
 Program to generate dependency of the source code ({\tt make} command uses to generate {\tt work/Makefile})
\end{description}
%
The following library files are also made in {\tt work} directory.
%
\begin{description}
\item{\tt libcalypso.a:    } Calypso library
\item{\tt libcalypso\_c.a:    } Calypso library from C sources
\item{\tt libfftpack.5d.a: } FFTPACK 5.1 library
\end{description}
%

\subsubsection{Clean}
The object and fortran module files in {\tt work} directory is deleted by typing
\begin{verbatim}
% make clean
\end{verbatim}
This command deletes files with the extension {\tt .o}, {\tt .mod}, {\tt .par}, {\tt .diag}, and {\tt ~}.

\subsubsection{Distclean}
To revert the files and directory to the original package, use make distclean as
\begin{verbatim}
% make distclean
\end{verbatim}

\subsubsection{Install}
 The executive files are copied to the install directory \verb|$(INSTDIR)/bin|. The install directory \verb|$(INSTDIR)| is defined in Makefile, and can also set by  \verb|${--prefix}| option for \verb|configure| command. Alternatively, you can use the programs in \verb|${SRCDIR}/bin| directory without running \verb|make install|. If directory \verb|${PREFIX}| does not exist, \verb|make install | creates  \verb|${PREFIX}|,  \verb|${PREFIX}/lib|,  \verb|${PREFIX}/bin|, and  \verb|${PREFIX}/include| directories. No files are installed in \verb|${PREFIX}/lib| and \verb|${PREFIX}/include|.

\subsubsection{Construct dependecies (only for developper)}
Fortran90 routines need to be build after modules which are used in the routines. C source files also need dependency among include files. Consequently, list of dependency of source files are saved in the file \verb|Makefile.depends| in each directory. When you modify the source files with changing the module usage,  \verb|Makefile.depends| files need to be updated. To update the  \verb|Makefile.depends|files, use the  \verb|make| command at the \verb|[CALYPSO_HOME]| directory as \\
%
\begin{verbatim}
% make depends
\end{verbatim}

This process generate dependencies of the Fortran modules by program \verb|make_f90depends|. For C source files, the dependency is generated by the gcc with \verb|-MM -w -DDEPENDENCY_CHECK| option. Consequently, the dependencies need to be generated by the environment with gcc or compatible compiler. After generating the dependency, you can transfer the modified package and build without using gcc.

\subsection{Install without using configure}
\label{section:no_configure}
It is possible to compile Calypso without using the \verb|configure| command. To do this, you need to edit the \verb|Makefile|. First, copy \verb|Makefile| from template \verb|Makefile.in| as
%
\begin{verbatim}
% cp Makefile.in Makefile
\end{verbatim}
In Makefile, the following variables should be defined.
%
\begin{description}
\item{\verb|SHELL|}    Name of shell command.
\item{\verb|SRCDIR|}   Directory of this Makefile. 
\item{\verb|INSTDIR|}  Install directory.
\item{\verb|MPICHDIR|} Directory names for MPI implementation. If you set fortran90 compiler name for MPI programs in \verb|MPIF90|, you do not need to define this valuable.
\item{\verb|MPICHINCDIR|} Directory names for include files for MPI implementation. If you set fortran90 compiler name for MPI programs in \verb|MPIF90|, you do not need to define this valuable.
\item{\verb|MPILIBS|}   Library names for MPI implementation. If you set fortran90 compiler name for MPI programs in \verb|MPIF90|, you do not need to define this valuable.
\item{\verb|F90_LOCAL|} Command name of local Fortran 90 compiler to compile module dependency listing program.
\item{\verb|MPIF90|} Command name of Fortran90 compiler and linker for MPI programs. If command does not have MPI implementation, you need to define the definition of MPI libraries \verb|MPICHDIR|, \verb|MPICHINCDIR|, and \verb|MPILIBS|.
\item{\verb|AR|}     Command name for archive program (ex. \verb|ar|) to generate libraries. If you need some options for archive command, options are also included in this valuable.
\item{\verb|RANLIB|} Command name for \verb|ranlib| to generate index to the contents of an archive. If system does not have \verb|ranlib|, set \verb|true| in this valuable. \verb|true| command does not do anything for libraries.
\item{}
\item{\verb|F90OPTFLAGS|}  Optimization flags for Fortran90 compiler (including OpenMP flags)
\item{\verb|BLAS_LIBS|} Library lists for BLAS  (ex.  \verb|-lblas|)
\item{\verb|ZLIB_CFLAGS|} Option flags for zlib  (ex.  \verb|-I/usr/include|)
\item{\verb|ZLIB_LIB|}   Library lists for zlib (ex. \verb|-L/usr/lib -lz|)
\item{\verb|FFTW3_CFLAGS|} Option flags for FFTW3  (ex.  \verb|-I/usr/local/include|)
\item{\verb|FFTW3_LIBS|}   Library lists for FFTW3 (ex. \verb|-L/usr/local/lib -lfftw3 -lm|)
\item{\verb|HDF5_FFLAGS|}  Option flags to compile with HDF5. This setting can be found by using hfd5 command \verb|h5pfc -show|.

\item{\verb|HDF5_LDFLAGS|}    Option flags  to link with  HDF5. This setting can be found by using hfd5 command \verb|h5pfc -show|.

\item{\verb|HDF5_FLIBS|}   Library lists for HDF5. This setting can be found by using hfd5 command \verb|h5pfc -show|.

\end{description}
%

\subsection{Install using cmake}
CMake is a cross-platform, open-source build system. CMake can be downloaded from \url{http://www.cmake.org}. The following procedure is required to install.
%
\begin{enumerate}
\item Create working directory (you can also use \verb|[CALYPSO_HOME]/work|).
\item Generate Makefile and working directories by {\tt cmake} command.
\item Compile programs by {\tt make} command.
\end{enumerate}
%
In this section, \verb|[CALYPSO\_HOME]/work| is used as the working directory.
Options for CMake can be checked by \verb|cmake -i [CALYPSO_HOME]| command at \verb|[CALYPSO_HOME]| \\
\verb|/work|. There are a number of options can be found, but the following valuables are important settings for installation:
%
\begin{itemize}
\item Install directory
\begin{description}
\item{\verb|CMAKE_INSTALL_PREFIX|}  \\
Install directory
\end{description}

\item Compiler settings
\begin{description}
\item{\verb|CMAKE_Fortran_COMPILER|} \\
Fortran90 compiler.
\item{\verb|CMAKE_c_COMPILER|} C compiler.

\item{\verb|CMAKE_Fortran_FLAGS|} \\
Optimization flags for Fortran90 compiler.
\item{\verb|CMAKE_c_FLAGS|} \\
Optimization flags for C compiler.
\end{description}

\item Option settings
\begin{description}
\item{\verb|CMAKE_DISABLE_FIND_PACKAGE_OpenMP_Fortran|} \\
OpenMP is not used if 'yes' is set in this valuable.
\item{\verb|CMAKE_DISABLE_FIND_PACKAGE_BLAS|}  \\
BLAS library is not linked if 'yes' is set in this valuable.
\item{\verb|CMAKE_DISABLE_FIND_PACKAGE_FFTW|}  \\
FFTW3 library is not linked if 'yes' is set in this valuable.
\item{\verb|CMAKE_DISABLE_FIND_PACKAGE_ZLIB|}  \\
Zlib library is not linked if 'yes' is set in this valuable.
\item{\verb|CMAKE_DISABLE_FIND_PACKAGE_HDF5|}  \\
HDF5 library is not linked if 'yes' is set in this valuable.
\end{description}

\item Manual settings for optional features
\begin{description}
\item{\verb|CMAKE_LIBRARY_PATH|}   \\
CMake library  search paths. This directory is used to search FFTW3 library.
\item{\verb|CMAKE_INCLUDE_PATH|}   \\
CMake include search paths. This directory is used to search include file for FFTW3.
\item{\verb|HDF5_INCLUDE_DIRS|}  \\
Include file directories to compile with HDF5. This setting can be found by using hfd5 command \verb|h5pfc -show|.
\item{\verb|HDF5_LIBRARY_DIRS|}    \\
Location of HDF5 library. This setting can be found by using hfd5 command \verb|h5pfc -show|.
\item{\verb|HDF5_LIBRARIES|}   \\
Library lists for HDF5. This setting can be found by using hfd5 command \verb|h5pfc -show|.
\end{description}
%
\end{itemize}
%
The easiest example of using CMake on Mac OS X with gcc9 is the following:
 \begin{verbatim}
% cd build
% cmake ~/CALYPSO/ -DCMAKE_Fortran_COMPILER=/opt/local/bin/gfortran-mp-9 \
? -DCMAKE_c_COMPILER=/opt/local/bin/gcc-mp-9 \
? -DCMAKE_Fortran_FLAGS="-O3 -g" -DCMAKE_c_FLAGS="-O3"
\end{verbatim}
%
After configuration, compile and install are started by
 \begin{verbatim}
% make
...
% make install
\end{verbatim}
%

After running \verb|make| command, execute files are built in \verb|[CALYPSO_HOME]/work/bin| directory.

%
\newpage
\newpage
%

\section{Simulation program ({\tt sph\_mhd})}
\label{section:sph_mhd}
%

The name of the simulation program is {\tt sph\_mhd}. This program requires {\tt control\_MHD} as a Control file. This program performs with the paramer file {\tt control\_MHD}, boundary condition data file \verb|[boundary_data_name]| (optional), and  indexing file for spherical harmonics generated by the preprocessing program {\tt gen\_sph\_grid} (optional).
%
\begin{figure}[htbp]
\begin{center}
\includegraphics*[width=130mm]{Images/flow_2}
\end{center}
\caption{Data flow for the simulation program.}
\label{fig:flow_2}
\end{figure}
%
Data files for this program are listed in Table \ref{table:sph_mhd}. Indexing data for spherical harmonics which starting with \verb|[sph_prefix]| are obtained by the preprocessing program \verb|gen_sph_grid|. If these indexing data files do not exist, the spherical harmonics indexing data files are also generated by using information in \verb|spherical_shell_ctl| block. The boundary condition data file \verb|[boundary_data_name]| is optionally required if boundary conditions for temperature and composition are not homogenous.
%
\begin{table}[htp]
\caption{List of Inoput/Output files of {\tt sph\_mhd} }
\begin{center} 
\begin{tabular}{|c|c|}
\hline
 name  & I/O \\ \hline \hline
\verb|control_MHD| & Input \\ \hline
\verb|[boundary_data_name]| & Input (opthional) \\ \hline
\verb|[rst_prefix].[step #].[rst_extension]| & Input/Output  \\ \hline
\end{tabular}
\end{center}
\label{table:sph_mhd}
\end{table}
%
\begin{table}[htp]
\caption{List of output files for visualization and data analysis of {\tt sph\_mhd} }
\begin{center} 
\begin{tabular}{|c|}
\hline
 name  \\ \hline \hline
\verb|[vol_pwr_prefix]_s.[dat/dat.gz]|   \\ \hline
\verb|[vol_pwr_prefix]_l.[dat/dat.gz]|   \\
\verb|[vol_pwr_prefix]_m.[dat/dat.gz]|   \\
\verb|[vol_pwr_prefix]_m0.[dat/dat.gz]|   \\
\verb|[vol_pwr_prefix]_lm.[dat/dat.gz]|   \\
\verb|[vol_ave_prefix].[dat/dat.gz]|   \\ \hline
\verb|[layer_pwr_prefix]_s.[dat/dat.gz]|   \\
\verb|[layer_pwr_prefix]_l.[dat/dat.gz]|   \\
\verb|[layer_pwr_prefix]_m.[dat/dat.gz]|   \\
\verb|[layer_pwr_prefix]_m0.[dat/dat.gz]|   \\
\verb|[layer_pwr_prefix]_lm.[dat/dat.gz]|   \\ \hline
\verb|[gauss_coef_prefix].[dat/dat.gz]|     \\
\verb|[picked_sph_prefix].[dat/dat.gz]|     \\ \hline
\verb|[nusselt_number_prefix].[dat/dat.gz]|     \\ \hline
\verb|[dipolarity_file_prefix].[dat/dat.gz]|     \\ \hline
\verb|[dynamo_benchmark_data_ctl].[dat/dat.gz]|     \\ \hline
\verb|[field_on_circle_prefix].[dat/dat.gz]|     \\ \hline
\verb|[spectr_on_circle_prefix].[dat/dat.gz]|     \\ \hline
\verb|[fld_prefix].[step#].[domain#].[extension]|   \\ \hline
\verb|[section_prefix].[step#].[extension]|   \\
\verb|[isosurface_prefix].[step#].[extension]|  \\ \hline
\end{tabular}
\end{center}
\label{table:sph_mhd_out}
\end{table}
%
%
\begin{table}[htp]
\caption{List of optional files to use data by Ver.1.x}
\begin{center} 
\begin{tabular}{|c|c|}
\hline
 name & I/O \\ \hline \hline
\verb|[sph_prefix].[rj_extension]|  & Input / (Output) \\
\verb|[sph_prefix].[rlm_extension]| & Input / (Output) \\
\verb|[sph_prefix].[rtm_extension]| & Input / (Output) \\
\verb|[sph_prefix].[rtp_extension]| & Input / (Output) \\ \hline
\verb|[sph_prefix].[fem_extension]| & (Input / Output) \\ \hline
\end{tabular}
\end{center}
\label{table:sph_mhd_optional}
\end{table}
%

{\color{red} {\bf Caution:} Calypso can save data files into subdirectories where is defined in control files. However, These directories have to prepare before the simulations because Fortran does not have a feature to make a new directory.}

%
\newpage
\subsection{Control file}
Control files for Calypso consists of blocks starting and ending with \verb|begin| and \verb|end|, respectively. Entities with more than one components are defined between \verb|begin array| and \verb|end array| flags. The number of components of an array must be defined at \verb|begin array| line. If blocks to be defined in an external file, the external file name is defined by \verb|file| flag. 

The format of the control file \verb|control_MHD| is described below. The detail of each block is described in section \ref{section:def_control}. You can jump to detailed description by clicking each item. \\
\\
%
Block \verb|MHD_control|  (Top block of the control file)
\label{href_i:MHD_control}
%
\begin{itemize}
\item Block \hyperref[href_t:data_files_def]{\tt data\_files\_def}
	\label{href_i:data_files_def}
%
	\begin{itemize}
	\item \hyperref[href_t:num_subdomain_ctl]
			{\tt num\_subdomain\_ctl    [Num\_PE]}
	\item \hyperref[href_t:num_smp_ctl]
			{\tt num\_smp\_ctl    [Num\_Threads]}
	\item \hyperref[href_t:sph_file_prefix]
			{\tt sph\_file\_prefix    [sph\_prefix]}
	\item \hyperref[href_t:boundary_data_file_name]
		{\tt boundary\_data\_file\_name    [File\_Name]}
%	\item \hyperref[href_t:radial_field_file_name]
%		{\tt radial\_field\_file\_name    [File_Name]}
%
	\item \hyperref[href_t:restart_file_prefix]
		{\tt restart\_file\_prefix    [rst\_prefix]}
	\item \hyperref[href_t:field_file_prefix]
			{\tt field\_file\_prefix    [fld\_prefix]}
%
	\item \hyperref[href_t:sph_file_fmt_ctl]
			{\tt sph\_file\_fmt\_ctl    [sph\_format]}
	\item \hyperref[href_t:restart_file_fmt_ctl]
			{\tt restart\_file\_fmt\_ctl    [rst\_format]}
	\item \hyperref[href_t:field_file_fmt_ctl]
			{\tt field\_file\_fmt\_ctl    [fld\_format]}
	\end{itemize}
%
\item File or Block \hyperref[href_i:spherical_shell_ctl]
			{\tt spherical\_shell\_ctl        [resolution\_control]}  \\
			 (See Section \ref{section:resolution_block})
	\begin{itemize}
	\item Block \hyperref[href_i:FEM_mesh_ctl]
        {\tt FEM\_mesh\_ctl} (See Section \ref{section:resolution_block})
	\item Block \hyperref[href_i:num_domain_ctl]
		{\tt num\_domain\_ctl} (See Section \ref{section:resolution_block})
	\item Block \hyperref[href_i:num_grid_sph]
		{\tt num\_grid\_sph} (See Section \ref{section:resolution_block})
	\end{itemize}
%
\item Block \verb|model|
	\begin{itemize}
	\item Block \hyperref[href_t:phys_values_ctl]{\tt phys\_values\_ctl}
		\begin{itemize} \label{href_i:phys_values_ctl}
		\item Array \hyperref[href_t:nod_value_ctl]
			{\tt nod\_value\_ctl    [Field]  [Viz\_flag]  [Monitor\_flag]}
		\end{itemize}
%
	\item Block \hyperref[href_t:time_evolution_ctl]{\tt time\_evolution\_ctl}
		\begin{itemize} \label{href_i:time_evolution_ctl}
		\item Array \hyperref[href_t:time_evo_ctl]
			{\tt time\_evo\_ctl    [Field]}
		\end{itemize}
%
	\item Block \hyperref[href_t:boundary_condition]{\tt boundary\_condition}
		\begin{itemize} \label{href_i:boundary_condition}
		\item Array \hyperref[href_t:bc_temperature]
            {\tt bc\_temperature       [Group]  [Type]  [Value]}
		\item Array \hyperref[href_t:bc_velocity]
			{\tt bc\_velocity          [Group]  [Type]  [Value]}
		\item Array \hyperref[href_t:bc_composition]
			{\tt bc\_composition       [Group]  [Type]  [Value]}
		\item Array \hyperref[href_t:bc_magnetic_field]
			{\tt bc\_magnetic\_field    [Group]  [Type]  [Value]}
		\end{itemize}
%
	\item Block \hyperref[href_t:forces_define]{\tt forces\_define}
		\begin{itemize} \label{href_i:forces_define}
		\item Array \hyperref[href_t:force_ctl]{\tt force\_ctl    [Force]}
		\end{itemize}
%
	\item Block \hyperref[href_t:dimensionless_ctl]{\tt dimensionless\_ctl}
		\begin{itemize} \label{href_i:dimensionless_ctl}
		\item Array \hyperref[href_t:dimless_ctl]{\tt dimless\_ctl    [Name]  [Value]}
		\end{itemize}
%
	\item Block \hyperref[href_t:coefficients_ctl]{\tt coefficients\_ctl}
		\begin{itemize} \label{href_i:coefficients_ctl}
		\item Block \hyperref[href_t:thermal]{\tt thermal}
			\begin{itemize} \label{href_i:thermal}
			\item Array \hyperref[href_t:coef_4_termal_ctl]
					{\tt coef\_4\_termal\_ctl          [Name] [Power]}
			\item Array \hyperref[href_t:coef_4_t_diffuse_ctl]
					{\tt coef\_4\_t\_diffuse\_ctl      [Name] [Power]}
			\item Array \hyperref[href_t:coef_4_heat_source_ctl]
					{\tt coef\_4\_heat\_source\_ctl    [Name] [Power]}
			\end{itemize}
%
		\item Block \hyperref[href_t:momentum]{\tt momentum}
			\begin{itemize} \label{href_i:momentum}
			\item Array \hyperref[href_t:coef_4_velocity_ctl]
				{\tt coef\_4\_velocity\_ctl          [Name] [Power]}
			\item Array \hyperref[href_t:coef_4_press_ctl]
				{\tt coef\_4\_press\_ctl             [Name] [Power]}
			\item Array \hyperref[href_t:coef_4_v_diffuse_ctl]
                {\tt coef\_4\_v\_diffuse\_ctl         [Name] [Power]}
			\item Array \hyperref[href_t:coef_4_buoyancy_ctl]
				{\tt coef\_4\_buoyancy\_ctl          [Name] [Power]}
			\item Array \hyperref[href_t:coef_4_Coriolis_ctl]
				{\tt coef\_4\_Coriolis\_ctl          [Name] [Power]}
			\item Array \hyperref[href_t:coef_4_Lorentz_ctl]
				{\tt coef\_4\_Lorentz\_ctl           [Name] [Power]}
			\item Array \hyperref[href_t:coef_4_composit_buoyancy_ctl]
				{\tt coef\_4\_composit\_buoyancy\_ctl [Name] [Power]}
			\end{itemize}
%
		\item Block \hyperref[href_t:induction]{\tt induction}
			\begin{itemize} \label{href_i:induction}
			\item Array \hyperref[href_t:coef_4_magnetic_ctl]
				{\tt coef\_4\_magnetic\_ctl   [Name] [Power]}
			\item Array \hyperref[href_t:coef_4_m_diffuse_ctl]
				{\tt coef\_4\_m\_diffuse\_ctl [Name] [Power]}
			\item Array \hyperref[href_t:coef_4_induction_ctl]
				{\tt coef\_4\_induction\_ctl  [Name] [Power]}
			\end{itemize}
%
		\item Block \hyperref[href_t:composition]{\tt composition}
			\begin{itemize} \label{href_i:composition}
			\item Array \hyperref[href_t:coef_4_composition_ctl]
				{\tt coef\_4\_composition\_ctl         [Name] [Power]}
			\item Array \hyperref[href_t:coef_4_c_diffuse_ctl]
				{\tt coef\_4\_c\_diffuse\_ctl          [Name] [Power]}
			\item Array \hyperref[href_t:coef_4_composition_source_ctl]
				{\tt coef\_4\_composition\_source\_ctl [Name] [Power]}
			\end{itemize}
		\end{itemize}
%
%	\item Block \hyperref[href_t:gravity_define]{\tt gravity\_define}
%		\begin{itemize} \label{href_i:gravity_define}
%		\item \verb||
%				\hyperref[href_t:gravity_type_ctl]{\tt gravity\_type\_ctl    [Name]}
%		\end{itemize}
%
%	\item Block \hyperref[href_t:Coriolis_define]{\tt Coriolis_define}
%		\begin{itemize} \label{href_i:Coriolis_define}
%		\item Array \hyperref[href_t:rotation_vec]
%				{\tt rotation\_vec    [Direction] [Value]}
%		\end{itemize}
%
	\item Block \hyperref[href_t:temperature_define]{\tt temperature\_define}
		\begin{itemize} \label{href_i:temperature_define}
		\item \verb||
				\hyperref[href_t:ref_temp_ctl]{\tt ref\_temp\_ctl        [REFERENCE\_TYPE]}
		\item \verb||
				\hyperref[href_t:ref_field_file_name]{\tt ref\_field\_file\_name [File\_Name]}
		\item Block \hyperref[href_t:low_temp_ctl]{\tt low\_temp\_ctl}
			\begin{itemize}
			\item \hyperref[href_t:depth]      {\tt depth        [RADIUS]}
			\item \hyperref[href_t:temperature]{\tt temperature  [TEMPERATURE]}
			\end{itemize}
%
		\item Block \hyperref[href_t:high_temp_ctl]{\tt high\_temp\_ctl}
			\begin{itemize}
			\item \hyperref[href_t:depth]      {\tt depth        [RADIUS]}
			\item \hyperref[href_t:temperature]{\tt temperature  [TEMPERATURE]}
			\end{itemize}
		\end{itemize}
%
	\item Block \hyperref[href_t:composition_define]{\tt tcomposition\_define}
		\begin{itemize} \label{href_i:composition_define}
		\item \verb||
				\hyperref[href_t:ref_comp_ctl]{\tt ref\_comp\_ctl        [REFERENCE\_TYPE]}
		\item \verb||
				\hyperref[href_t:ref_field_file_name]{\tt ref\_field\_file\_name [File\_Name]}
		\item Block \hyperref[href_t:low_comp_ctl]{\tt low\_comp\_ctl}
			\begin{itemize}
			\item \hyperref[href_t:depth]      {\tt depth        [RADIUS]}
			\item \hyperref[href_t:composition]{\tt composition  [COMPOSITION]}
			\end{itemize}
%
		\item Block \hyperref[href_t:high_comp_ctl]{\tt high\_comp\_ctl}
			\begin{itemize}
			\item \hyperref[href_t:depth]      {\tt depth        [RADIUS]}
			\item \hyperref[href_t:composition]{\tt composition  [COMPOSITION]}
			\end{itemize}
		\end{itemize}
	\end{itemize}
%
\item Block \verb|control|
	\begin{itemize}
	\item Block \hyperref[href_t:time_step_ctl]{\tt time\_step\_ctl}
		\begin{itemize} \label{href_i:time_step_ctl}
		\item \hyperref[href_t:elapsed_time_ctl]
			{\tt elapsed\_time\_ctl        [ELAPSED\_TIME]}
		\item \hyperref[href_t:i_step_init_ctl]
			{\tt i\_step\_init\_ctl        [ISTEP\_START]}
		\item \hyperref[href_t:i_step_finish_ctl]
			{\tt i\_step\_finish\_ctl      [ISTEP\_FINISH]}
		\item \hyperref[href_t:i_step_check_ctl]
			{\tt i\_step\_check\_ctl       [ISTEP\_MONITOR]}
		\item \hyperref[href_t:i_step_rst_ctl]
			{\tt i\_step\_rst\_ctl         [ISTEP\_RESTART]}
		\item \hyperref[href_t:i_step_field_ctl]
			{\tt i\_step\_field\_ctl       [ISTEP\_FIELD]}
		\item \hyperref[href_t:i_step_sectioning_ctl]
			{\tt i\_step\_sectioning\_ctl  [ISTEP\_SECTION]}
		\item \hyperref[href_t:i_step_isosurface_ctl]
			{\tt i\_step\_isosurface\_ctl  [ISTEP\_ISOSURFACE]}
		\item \hyperref[href_t:dt_ctl]
			{\tt dt\_ctl                   [DELTA\_TIME]}
		\item \hyperref[href_t:time_init_ctl]
			{\tt time\_init\_ctl           [INITIAL\_TIME]}
		\end{itemize}
%
	\item Block \hyperref[href_t:restart_file_ctl]{\tt restart\_file\_ctl}
		\begin{itemize} \label{href_i:restart_file_ctl}
		\item \hyperref[href_t:rst_ctl]{\tt rst\_ctl      [INITIAL\_TYPE]}
		\end{itemize}
%
	\item Block \verb||
    		\hyperref[href_t:time_loop_ctl]{\tt time\_loop\_ctl}
		\begin{itemize} \label{href_i:time_loop_ctl}
		\item \hyperref[href_t:scheme_ctl]{\tt scheme\_ctl              [EVOLUTION\_SCHEME]}
		\item \hyperref[href_t:coef_imp_v_ctl]{\tt coef\_imp\_v\_ctl    [COEF\_INP\_U]}
		\item \hyperref[href_t:coef_imp_t_ctl]{\tt coef\_imp\_t\_ctl    [COEF\_INP\_T]}
		\item \hyperref[href_t:coef_imp_b_ctl]{\tt coef\_imp\_b\_ctl    [COEF\_INP\_B]}
		\item \hyperref[href_t:coef_imp_c_ctl]{\tt coef\_imp\_c\_ctl    [COEF\_INP\_C]}
		\item \hyperref[href_t:FFT_library_ctl]{\tt FFT\_library\_ctl   [FFT\_Name]}
		\item \hyperref[href_t:Legendre_trans_loop_ctl]
			{\tt Legendre\_trans\_loop\_ctl [Leg\_Loop]}
		\end{itemize}
%
	\end{itemize}
%
\item Block \hyperref[href_t:sph_monitor_ctl]{\tt sph\_monitor\_ctl}
	\begin{itemize} \label{href_i:sph_monitor_ctl}
	\item \hyperref[href_t:volume_average_prefix]
			{\tt volume\_average\_prefix        [vol\_ave\_prefix]}
	\item \hyperref[href_t:volume_pwr_spectr_prefix]
			{\tt volume\_pwr\_spectr\_prefix    [vol\_pwr\_prefix]}
	\item \hyperref[href_t:volume_pwr_spectr_format]
		{\tt volume\_pwr\_spectr\_format    [file\_format]}
	\item \hyperref[href_t:degree_spectra_switch]
		{\tt degree\_spectra\_switch           [ON/OFF]}
	\item \hyperref[href_t:order_spectra_switch]
		{\tt order\_spectra\_switch           [ON/OFF]}
	\item \hyperref[href_t:diff_lm_spectra_switch]
		{\tt diff\_lm\_spectra\_switch           [ON/OFF]}
	\item \hyperref[href_t:axisymmetric_power_switch]
		{\tt axisymmetric\_power\_switch           [ON/OFF]}
%
	\item \hyperref[href_t:nusselt_number_prefix]
			{\tt nusselt\_number\_prefix        [nusselt\_number\_prefix]}
	\item \hyperref[href_t:nusselt_number_format]
			{\tt nusselt\_number\_format    [file\_format]}
%
	\item Array \hyperref[href_t:volume_spectrum_ctl]{\tt volume\_spectrum\_ctl}
		\begin{itemize}
		\item Block \verb|volume_spectrum_ctl|
			\begin{itemize}
			\item \hyperref[href_t:volume_average_prefix]
				{\tt volume\_average\_prefix      [vol\_ave\_prefix]}
			\item \hyperref[href_t:volume_pwr_spectr_prefix]
				{\tt volume\_pwr\_spectr\_prefix  [vol\_pwr\_prefix]}
        	\item \hyperref[href_t:volume_pwr_spectr_format]
		    	{\tt volume\_pwr\_spectr\_format    [file\_format]}
%
			\item \hyperref[href_t:degree_spectra_switch]
				{\tt degree\_spectra\_switch           [ON/OFF]}
			\item \hyperref[href_t:order_spectra_switch]
				{\tt order\_spectra\_switch           [ON/OFF]}
			\item \hyperref[href_t:diff_lm_spectra_switch]
				{\tt diff\_lm\_spectra\_switch           [ON/OFF]}
			\item \hyperref[href_t:axisymmetric_power_switch]
				{\tt axisymmetric\_power\_switch           [ON/OFF]}
%
			\item \hyperref[href_t:inner_radius_ctl]
				{\tt inner\_radius\_ctl           [radius]}
			\item \hyperref[href_t:outer_radius_ctl]
				{\tt outer\_radius\_ctl           [radius]}
			\end{itemize}
		\end{itemize}
%
	\item Block \hyperref[href_t:layered_spectrum_ctl]{\tt layered\_spectrum\_ctl}
		\begin{itemize}
		\item \hyperref[href_t:layered_pwr_spectr_prefix]
				{\tt layered\_pwr\_spectr\_prefix         [layer\_pwr\_prefix]}
	       	\item \hyperref[href_t:layered_pwr_spectr_format]
		    	{\tt volume\_pwr\_spectr\_format    [file\_format]}
%
		\item \hyperref[href_t:degree_spectra_switch]
			{\tt degree\_spectra\_switch           [ON/OFF]}
		\item \hyperref[href_t:order_spectra_switch]
			{\tt order\_spectra\_switch           [ON/OFF]}
		\item \hyperref[href_t:diff_lm_spectra_switch]
			{\tt diff\_lm\_spectra\_switch           [ON/OFF]}
		\item \hyperref[href_t:axisymmetric_power_switch]
			{\tt axisymmetric\_power\_switch           [ON/OFF]}
%
		\item Array \hyperref[href_t:spectr_radius_ctl]
				{\tt spectr\_radius\_ctl [Radius] }
		\item Array \hyperref[href_t:spectr_layer_ctl]
				{\tt spectr\_layer\_ctl [Layer \#] }
		\end{itemize}
%
	\item Block \hyperref[href_t:gauss_coefficient_ctl]{\tt gauss\_coefficient\_ctl}
		\begin{itemize}
		\item \hyperref[href_t:gauss_coefs_prefix]
			{\tt gauss\_coefs\_prefix                [gauss\_coef\_prefix]}
       	\item \hyperref[href_t:gauss_coefs_format]
		    	{\tt gauss\_coefs\_format    [file\_format]}
		\item \hyperref[href_t:gauss_coefs_radius_ctl]
            {\tt gauss\_coefs\_radius\_ctl           [gauss\_coef\_radius]}
			\item Array \hyperref[href_t:pick_gauss_coefs_ctl]
                		{\tt pick\_gauss\_coefs\_ctl  [Degree]   [Order]}
		\item Array \hyperref[href_t:pick_gauss_coef_degree_ctl]
                    {\tt pick\_gauss\_coef\_degree\_ctl  [Degree]}
		\item Array \hyperref[href_t:pick_gauss_coef_order_ctl]
					{\tt pick\_gauss\_coef\_order\_ctl   [Order]}
		\end{itemize}
%
	\item Block \hyperref[href_t:pickup_spectr_ctl]{\tt pickup\_spectr\_ctl}
		\begin{itemize}
		\item \hyperref[href_t:picked_sph_prefix]
			{\tt picked\_sph\_prefix                    [picked\_sph\_prefix]|}
       	\item \hyperref[href_t:picked_sph_format]
		    	{\tt picked\_sph\_format     [file\_format]}
		\item Array \hyperref[href_t:pick_radius_ctl]
					{\tt pick\_radius\_ctl             [Radius]}
		\item Array \hyperref[href_t:pick_layer_ctl]
					{\tt pick\_layer\_ctl               [Layer \#]}
		\item Array \hyperref[href_t:pick_sph_spectr_ctl]
					{\tt pick\_sph\_spectr\_ctl         [Degree]  [Order]}
		\item Array \hyperref[href_t:pick_sph_degree_ctl]
					{\tt pick\_sph\_degree\_ctl         [Degree]}
		\item Array \hyperref[href_t:pick_sph_order_ctl]
					{\tt pick\_sph\_order\_ctl          [Order]}
		\end{itemize}
%
	\item Block \hyperref[href_t:sph_dipolarity_ctl]{\tt sph\_dipolarity\_ctl}
		\begin{itemize}
		\item \hyperref[href_t:dipolarity_file_prefix]
			{\tt dipolarity\_file\_prefix                    [dipolarity\_file\_prefix]|}
       	\item \hyperref[href_t:dipolarity_file_format]
		    	{\tt dipolarity\_file\_format     [file\_format]}
		\item Array \hyperref[href_t:dipolarity_truncation_ctl]
					{\tt dipolarity\_truncation\_ctl               [Degree]}
		\end{itemize}
%
	\item Block \hyperref[href_t:dynamo_benchmark_data_ctl]
				{\tt dynamo\_benchmark\_data\_ctl}
		\begin{itemize}
			\item \hyperref[href_t:dynamo_benchmark_file_prefix]
					{\tt dynamo\_benchmark\_file\_prefix   [File\_Prefix]}
					\label{href_i:dynamo_benchmark_file_prefix}
			\item \hyperref[href_t:dynamo_benchmark_file_format]
					{\tt dynamo\_benchmark\_file\_format   [ASCII or gzip]}
					\label{href_i:dynamo_benchmark_file_format}
			\item \hyperref[href_t:nphi_mid_eq_ctl]
					{\tt nphi\_mid\_eq\_ctl   [Nphi\_mid\_equator]}
					\label{href_i:nphi_mid_eq_ctl}
		\end{itemize}
%
	\item Array \hyperref[href_t:fields_on_circle_ctl]
				{\tt fields\_on\_circle\_ctl}
		\begin{itemize}
		\item Block {\tt fields\_on\_circle\_ctl}
			\begin{itemize}
				\item \hyperref[href_t:field_on_circle_prefix]
						{\tt field\_on\_circle\_prefix   [File\_Prefix]}
						\label{href_i:field_on_circle_prefix}
				\item \hyperref[href_t:spectr_on_circle_prefix]
						{\tt spectr\_on\_circle\_prefix   [File\_Prefix]}
						\label{href_i:spectr_on_circle_prefix}
				\item \hyperref[href_t:field_on_circle_format]
						{\tt field\_on\_circle\_format   [ASCII or gzip]}
						\label{href_i:field_on_circle_format}
				\item \hyperref[href_t:pick_circle_coord_ctl]
						{\tt pick\_circle\_coord\_ctl   [Coordinate]}
						\label{href_i:pick_circle_coord_ctl}
				\item \hyperref[href_t:nphi_mid_eq_ctl]
						{\tt nphi\_mid\_eq\_ctl   [Nphi\_mid\_equator]}
						\label{href_i:nphi_mid_eq_ctl}
				\item \hyperref[href_t:pick_cylindrical_radius_ctl]
						{\tt pick\_cylindrical\_radius\_ctl   [S]}
						\label{href_i:pick_cylindrical_radius_ctl}
				\item \hyperref[href_t:pick_vertical_position_ctl]
						{\tt pick\_vertical\_position\_ctl    [Z]}
						\label{href_i:pick_vertical_position_ctl}
			\end{itemize}
		\end{itemize}
	\end{itemize}
%
%
\item Block \hyperref[href_t:visual_control]{\tt visual\_control}
    \begin{itemize} \label{href_i:visual_control}
	\item \hyperref[href_t:i_step_sectioning_ctl]
        {\tt i\_step\_sectioning\_ctl  [ISTEP\_SECTION]}
    \item Array \hyperref[href_t:cross_section_ctl]{\tt cross\_section\_ctl}
		\begin{itemize}
        \item File or Block {\tt cross\_section\_ctl} \\
                            {\tt [section\_control\_file]} \\
								(See section \ref{section:section_control})
		\end{itemize}
%
    \item \hyperref[href_t:i_step_isosurface_ctl]
		{\tt i\_step\_isosurface\_ctl  [ISTEP\_ISOSURFACE]}
    \item Array \hyperref[href_t:isosurface_ctl]{\tt isosurface\_ctl}
		\begin{itemize}
		\item File or Block {\tt isosurface\_ctl} \\
                            {\tt [isosurface\_control\_file]} \\
								(See section \ref{section:isosurface_control})
		\end{itemize}
    \end{itemize}
%
\item Block \hyperref[href_t:dynamo_vizs_control]{\tt dynamo\_vizs\_control}
	\begin{itemize} \label{href_i:dynamo_vizs_control}
		\item File or Block \hyperref[href_t:zonal_mean_section_ctl]
							{\tt zonal\_mean\_section\_ctl} \\
							{\tt [zonal\_mean\_section\_control\_file]} \\
								(See section \ref{section:section_control})
		\item File or Block \hyperref[href_t:zonal_RMS_section_ctl]
							{\tt zonal\_RMS\_section\_ctl} \\
                            {\tt [zonal\_RMS\_section\_control\_file]} \\
                                (See section \ref{section:section_control})
%
		\item Block \hyperref[href_t:crustal_filtering_ctl]{\tt crustal\_filtering\_ctl}
			\begin{itemize}
				\item \hyperref[href_t:crustal_filtering_ctl]
						{\tt truncation\_degree\_ctl          [Degree]}
			\end{itemize}
	\end{itemize}
\end{itemize}
%
%
%
\subsubsection{Spatial resolution definition block}
\label{section:resolution_block}

Geometry of spherical shell, spatial resolution, and parallelization is defined in the block named \verb|spherical_shell_ctl|. The \verb|spherical_shell_ctl| block can be included into the file {\tt control\_MHD} or saved into another file.  if the \verb|spherical_shell_ctl| block in the independent file {\tt [resolution\_control]}, the file name is defined by
%
\begin{itemize}
\item {\tt   file    spherical\_shell\_ctl        [resolution\_control]})
\end{itemize}
The \verb|spherical_shell_ctl| block consists of the following parameters
%
\verb|spherical_shell_ctl|
\label{href_i:spherical_shell_ctl}
\\
\\
%
Block \hyperref[href_t:spherical_shell_ctl]{\tt spherical\_shell\_ctl}
%
		\begin{itemize}
		\item (Block \hyperref[href_t:FEM_mesh_ctl]{\tt FEM\_mesh\_ctl})
		\begin{itemize} \label{href_i:FEM_mesh_ctl}
		\item (\hyperref[href_t:FEM_mesh_output_switch]{\tt FEM\_mesh\_output\_switch [ON or OFF]})
		\end{itemize}
%
		\item Block \hyperref[href_t:num_domain_ctl]{\tt num\_domain\_ctl}
			\begin{itemize} \label{href_i:num_domain_ctl}
			\item \hyperref[href_t:ordering_set_ctl]{\tt ordering\_set\_ctl [ORDERING\_SET]}

			\item \hyperref[href_t:num_radial_domain_ctl]{\tt num\_radial\_domain\_ctl [Ndomain]}
			\item \hyperref[href_t:num_horizontal_domain_ctl]{\tt num\_horizontal\_domain\_ctl [Ndomain]}
%
            		\item {\color{magenta} Array \hyperref[href_t:num_domain_sph_grid]
				{\tt num\_domain\_sph\_grid    [Direction]    [Ndomain]} \\
				(Depricated)}
			\item {\color{magenta} Array \hyperref[href_t:num_domain_legendre]
				{\tt num\_domain\_legendre    [Direction]    [Ndomain]} \\
				(Depricated)}
			\item {\color{magenta} Array \hyperref[href_t:num_domain_spectr]
				{\tt num\_domain\_spectr      [Direction]    [Ndomain]} \\
				(Depricated)}
			\end{itemize}
%
		\item Block \hyperref[href_t:num_grid_sph]{\tt num\_grid\_sph}
			\begin{itemize} \label{href_i:num_grid_sph}
	        \item \hyperref[href_t:truncation_level_ctl]{\tt truncation\_level\_ctl	[Lmax]}
			\item \hyperref[href_t:ngrid_meridonal_ctl]{\tt ngrid\_meridonal\_ctl [Ntheta]}
			\item \hyperref[href_t:ngrid_zonal_ctl]{\tt ngrid\_zonal\_ctl [Nphi]}
%
			\item \hyperref[href_t:radial_grid_type_ctl]{\tt radial\_grid\_type\_ctl} \\
				\verb|[explicit, Chebyshev, or equi_distance]| \label{href_i:radial_grid_type_ctl}
%
			\item \hyperref[href_t:num_fluid_grid_ctl]{\tt num\_fluid\_grid\_ctl  [Nr\_shell]}
			\item \hyperref[href_t:fluid_core_size_ctl]{\tt fluid\_core\_size\_ctl  [Length]}
			\item \hyperref[href_t:ICB_to_CMB_ratio_ctl]{\tt ICB\_to\_CMB\_ratio\_ctl  [R\_ratio]}
			\item \hyperref[href_t:Min_radius_ctl]{\tt Min\_radius\_ctl  [Rmin]}    
				\label{href_i:Min_radius_ctl}
			\item \hyperref[href_t:Max_radius_ctl]{\tt Max\_radius\_ctl  [Rmax]}    
				\label{href_i:Max_radius_ctl}
%
\\
			\item Array \hyperref[href_t:r_layer]{\tt r\_layer  [Layer \#]  [Radius]}    
%
			\item Array \hyperref[href_t:boundaries_ctl]{\tt boundaries\_ctl  [Boundary\_name]  [Layer \#]}    
			\end{itemize}
		\end{itemize}

If \verb|num_radial_domain_ctl| and \verb|num_horizontal_domain_ctl| are defined, the following arrays \verb|num_domain_sph_grid|, \verb|num_domain_legendre|, and \verb|num_domain_spectr| are not necessary. \\
(see \hyperref[href_t:gen_w_innercore]{example} \verb|spherical_shell/with_inner_core|)

The external file for resoultion and parallelization information \verb|[resolution_control]| needs the following contorl blocks:
%
	\begin{itemize}
	\item Block \hyperref[href_i:spherical_shell_ctl]{\tt spherical\_shell\_ctl}
		\begin{itemize}
		\item Block \hyperref[href_i:FEM_mesh_ctl]{\tt FEM\_mesh\_ctl}
		\item Block \hyperref[href_i:num_domain_ctl]{\tt num\_domain\_ctl}
		\item Block \hyperref[href_i:num_grid_sph]{\tt num\_grid\_sph}
		\end{itemize}
	\end{itemize}
%
%
Calypso obtains resolution and parallelization information in the following order:
\begin{description}
\item[Step 1:] If spherical harmonics indexing files {\tt  [sph\_prefix]} defined by
\hyperref[href_t:sph_file_prefix]{\tt sph\_file\_prefix    [sph\_prefix]}  are exist, read these files and go to simulation.
\item[Step 2:] If files  {\tt  [sph\_prefix]} does not exist, construct resolution and parallelization information from the parameters in \verb|spherical_shell_ctl| block.
\item[Step 3:] If the parameter \hyperref[href_t:sph_file_prefix]{\tt sph\_file\_prefix    [sph\_prefix]}  is defined, the spherical harmonics indexing files {\tt  [sph\_prefix]} are written (not necessary). 
\end{description}
%
Various data format can be chosen for the spherical harmonic indexing files  {\tt  [sph\_prefix]} (see Section \ref{section:gen_sph_grid}).
%
\subsubsection{Position of radial grid}
The preprocessing program sets the radial grid spacing, either by a list in the control file or by setting an equidistant grid or Chebyshev collocation points.

In equidistance grid, radial grids are defined by
%
\begin{eqnarray}
r(k) & = & r_{i} + \left(r_{o}-r_{i} \right) \frac{k-k_{ICB}}{N},
\nonumber
\end{eqnarray}
%
where, $k_{ICB}$ is the grid points number at ICB. The radial grid set from the closest points of minimum radius defined by \hyperref[href_i:Min_radius_ctl]{\tt [Min\_radius\_ctl]} in control file to the closest points of the maximum radius defined by \hyperref[href_i:Max_radius_ctl]{\tt [Max\_radius\_ctl]} in control file, and radial grid number for the innermost points is set to $k = 1$.

In Chebyshev collocation points, radial grids in the fluid shell are defined by
%
\begin{eqnarray}
r(k) & = & r_{i} + \frac{\left(r_{o}-r_{i} \right)}{2} \left[ \frac{1}{2} - \cos \left(\pi \frac{ k-k_{ICB}}{N} \right) \right],
\nonumber
\end{eqnarray}
%
For the inner core ($r<r_{i}$), grid points is defined by
%
\begin{eqnarray}
r(k) & = & r_{i} - \frac{\left(r_{o}-r_{i} \right)}{2} \left[ \frac{1}{2} - \cos \left(\pi \frac{ k-k_{ICB}}{N} \right) \right],
\nonumber
\end{eqnarray}
%
and, grid points in the external of the shell ($r>r_{o}$) is defined by
%
\begin{eqnarray}
r(k) & = & r_{o} + \frac{\left(r_{o}-r_{i} \right)}{2} \left[ \frac{1}{2} - \cos \left(\pi \frac{ k-k_{CMB}}{N} \right) \right],
\nonumber
\end{eqnarray}
%
where, $k_{CMB}$ is the grid point number at CMB.

\subsubsection{How to define spatial resolution and parallelization?}
  Calypso uses spherical harmonics expansion method and in horizontal discretization and finite difference methods in the radial direction. In the spherical harmonics expansion methods, nonlinear terms are solved in the grid space while time integration and diffusion terms are solved in the spectrum space. We need to set truncation degree $l_{max}$ of the spherical harmonics and number of grids in the three direction $(N_{r}, N_{\theta}, N_{\phi})$ in the preprocessing program. The following condition is required (or recommended) for $l_{max}$ and $(N_{r}, N_{\theta}, N_{\phi})$. $l_{max}$ is defined by \verb|truncation_level_ctl|, and $N_{r}$ for the fluid shell (outer core) is defined by  \verb|num_fluid_grid_ctl|.  $N_{\theta}$ and $N_{\phi}$ is defined by \verb|ngrid_meridonal_ctl| and \verb|ngrid_zonal_ctl|, respectively.
%
\begin{itemize}
\item $N_{\phi} = 2 N_{\theta}$.
\item $N_{\theta}$ must be more than $l_{max}+1$, but
\item To eliminate aliasing in the spherical transform, $N_{\theta} \ge 1.5 \left( l_{max}+1 \right)$ is highly recommended.
\item $N_{\phi}$ should consists of products among power of 2, power of 3, and power of 5.
\end{itemize}
%
%
\begin{figure}[htbp]
\begin{center}
\includegraphics*[width=130mm]{Images/parallelize}
\end{center}
\caption{Parallelization and data communication in Calypso in the case using 9 (3x3) processors. Data are decomposed in radial and meridional direction for nonlinear term evaluations, decomposed in radial and harmonic order for Legendre transform, and decomposed in spherical harmonics for linear calculations.}
\label{fig:parallelization}
\end{figure}
%
Calypso is parallelized 2-dimensionally and direction of the parallelization is changed in the operations in the spherical transform (See Figure \ref{fig:parallelization}). Two dimensional parallelization delivers many parallelize configuration. Here is the approach  how to find the best configuration:
%
\begin{itemize}
\item Maximum parallelization level in horizontal direction is $\left( l_{max} + 1 \right)  /2$,  and $N_{r}+1$ is the maximum level in radial direction.
\item Decompose number of radial points $N_{r}+1$ and truncation degree $\left( l_{max} + 1 \right) / 2 $ into prime numbers.
\item Decide number of MPI processes from the prime numbers.
\item Choose the number of decomposition in the radial and horizontal direction as close as possible.
\end{itemize}
% 
Here is an example for the case with $(N_{r}, l_{max}) = (89, 95)$. The maximum number of parallelization is $90 \times 48  = 4320$ processes.  $N_{r}+1$ and $\left( l_{max} + 1 \right)  /2$ can be decomposed into $90 = 2 \times 3^2 \times 5$ and $48 = 2^4 \times 3 $. Now, if 160 processes run is intended, $160 = 10 \times 16$ is the closest number of decompositions. Comparing with the prime numbers of the spatial resolution, radial and horizontal decomposition will be 10 and 16, respectively.

\subsubsection{Radial grid data}
The program generates radius of each layer in \verb|radial_info.dat| if \verb|radial_grid_type_ctl| is set to \verb|Chebyshev| or \verb|equi_distance|. This file consists of blocks \verb|array r_layer| and \verb|array boundaries_ctl| for control file. This data may be useful if you want to modify radial grid spacing by yourself.

\subsubsection{Thermal and compositional boundary condition data file}\label{sec:boundary_file}
Thermal and compositional heterogeneity at boundaries are defined by a external file named  \verb|[boundary_data_name]|. In this file, temperature, composition, heat flux, or compositional flux at ICB or CMB can be defined by spherical harmonics coefficients. To use boundary conditions in \verb|[boundary_data_name]|, file name is defined by \verb|boundary_data_file_name| column in control file, and boundary condition type \verb|[type]| is set to \verb|fixed_file| or \verb|fixed_flux_file| in \verb|bc_temperature| or \verb|bc_composition| column. By setting \verb|fixed_file| or \verb|fixed_flux_file| in control file, boundary conditions are copied from the file \verb|[boundary_data_name]|.

An example of the boundary condition file is shown in Figure \ref{fig:boundary_file}. As for the control file, a line starting from '\verb|#|' or '\verb|!|' is recognized as a comment line. In \verb|[boundary_data_name]|, boundary condition data is defined as following:
%
\begin{enumerate}
\item  Number of total boundary conditions to be defined in this file.
\item  Field name to define the first boundary condition
\item  Place to define the first boundary condition (\verb|ICB| or \verb|CMB|)
\item  Number of spherical harmonics modes for each boundary condition
\item  Spectrum data for the boundary conditions (degree $l$, order $m$, and harmonics coefficients)
\item  After finishing the list of spectrum data return to Step 2 for the next boundary condition
\end{enumerate}
%
If harmonics coefficients of the boundary conditions are not listed in item 5, 0.0 is automatically applied for the harmonics coefficients of the boundary conditions. So, only non-zero components need to be listed in the boundary condition file. Lines starting from "\#" or "!" are ignored as comment lines.
%
\begin{figure}[htbp]
\begin{center}
{\small
\begin{verbatim}
#
#  number of boundary conditions
      4
#
#   boundary condition data list
#
#    Fixed temperature at ICB
temperature
ICB
    3
  0  0   1.0E+00
  1  1   2.0E-01
  2  2   3.0E-01
#
#    Fixed heat flux at CMB
heat_flux
CMB
   2
  0  0    -0.9E+0
  1  -1    5.0E-1
#
#    Fixed composition flux at ICB
composite_flux
ICB
   2
  0  0    0.0E+00
  2  0   -2.5E-01
#
#    Fixed composition at CMB
composition
CMB
   2
  0   0   1.0E+00
  2  -2   5.0E-01
\end{verbatim}
}
\end{center}
\caption{An example of boundary condition file.}
\label{fig:boundary_file}
\end{figure}
%
%
\subsubsection{Radial stationary field file}\label{sec:radial_field_file}
Horizontally homogeneous heat and light element source/sink can be defined by the text file defined by  \hyperref[href_t:radial_field_file_name]{\tt [radial\_field\_file\_name]}. If the radial field data file is read, the heat and composition source term in the restart data are overwritten. The data in the file consist of the radius and fields data. The read data are interpolated into the each grids points in the program. If there are more than  one field data, the radial grids in the data file has to be shared for the data. An example of the radial data file is shown in Figure \ref{fig:radial_field_file}. Lines starting from "\#" or "!" are ignored as comment lines. Calypso ignore the field in data file if this field is not used. 
%
\begin{figure}[htbp]
\begin{center}
{\small
\begin{verbatim}
#  domain ID
               0
#  time step number
               0
#  time, Delta t
   0.000000000000000E+000   0.000000000000000E+000
#  Number of grids, fields, and component
    7    3
    1    1    1
radius
   0.000000000000000E+000
   5.384615384615384E-001
   5.440461253489739E-001
   5.440461253489742E-001
   1.488945972412747E+000
   1.488945972412750E+000
   1.538461538461538E+000
heat_source
   0.000000000000000E+000
   0.000000000000000E+000
   0.000000000000000E+000
   0.000000000000000E+000
  -1.000000000000000E+001
  -1.000000000000000E+001
composition_source
   1.000000000000000E+000
   1.000000000000000E+000
   1.000000000000000E+000
   0.000000000000000E+001
   0.000000000000000E+001
   0.000000000000000E+000
   0.000000000000000E+000
\end{verbatim}
}
\end{center}
\caption{An example of boundary condition file.}
\label{fig:radial_field_file}
\end{figure}
%
\subsubsection{Reference fields output file}\label{sec:ref_field_output}
Calypso can solve the heat and composition equation using the reference and perturbation of the temperature and composition when the scalar satisfies the conservation.
When reference field for is used, reference temperature, radial temperature gradient, heat source, composition, radial gradient of composition, and composition source is written in the ifle {\tt reference\_fields.dat}. This file has the same format as the input radial stationary field file  \hyperref[href_t:radial_field_file_name]{\tt [radial\_field\_file\_name]}. This file is useful to figure out the diffusive temperature and composition profile when \hyperref[href_t:ref_temp_ctl]{\tt [ref\_temp\_ctl]} or \hyperref[href_t:ref_comp_ctl]{\tt [ref\_comp\_ctl]} is set to {\tt numerical\_solution}. 
%
\begin{eqnarray}
 0 = \nabla^{2} T_{0} + q_{0},
\nonumber
\end{eqnarray}
%
where $q_{0}$ is the heat source/sink as a function of radius. Simple examples to evaluate the reference field is in {\tt examples/get\_reference\_1} and {\tt examples/get\_reference\_2} directory.

({\color{red} Caution}) When fixed boundary condition is not set for the inner nor outer boundary, it is almost impossible to solve the diffusive profile numerically. Consequently, the diffusive profile is evaluated with applying the fixed boundary condition $T = 0$ at the outer boundary. 
%
%
\subsection{Spectrum data for restarting}
Spectrum data is used for restarting data and generating field data by Data transform program \verb|sph_snapshot|, \verb|sph_zm_snapshot|, or \verb|sph_dynamobench|. This file is saved for each subdomain (MPI processes), then \verb|[step #]| and \verb|[domain #]| are added in the file name. The \verb|[step #]| is calculated by \verb|time step| / \verb|[ISTEP_RESTART]|. Data format is defined by \verb|[restart_file_fmt_ctl]| as shown in Table \ref{table:restart_format}.
%
\begin{table}[htp]
\caption{data format flag {[\tt restart\_file\_fmt\_ctl]} and extensions for the restart file. An example of data size and output time is also listed.}
\begin{center} 
\begin{tabular}{|c|c|c|c|c|}
\hline
File format & \verb|[sph_file_fmt_ctl]|  & \verb|extension| & size (MByte) & time (sec)  \\ \hline \hline
              &  \verb|ascii| & \verb|[#].fld|  & 1,286 & 7.10 \\
Parallel  &  \verb|binary| & \verb|[#].flb|  & 410 & 2.36 \\
              &  \verb|gzip| &  \verb|[#].fld.gz| & 418 &  4.65 \\
              &  \verb|bin_gz| &\verb|[#].flb.gz| & 336 &  2.84 \\ \hline
              & \verb|merged|  &  \verb|.fld| & 1,312 & 50.2 \\
Merged  &  \verb|merged_bin| &  \verb|.flb| & 413 &  2.34 \\
              &  \verb|merged_gz| &  \verb|.fld.gz| & 455 & 8.60 \\
              &  \verb|merged_bin_gz| &  \verb|.flb.gz| & 340 & 3.57 \\ \hline
\end{tabular}
\end{center}
\label{table:restart_format}
\end{table}
%

\subsection{Field data for visualization}
\label{sec:VTK}
Field data is used for the visualization processes. Field data are written with XDMF format (\url{http://www.xdmf.org/index.php/Main_Page}), merged VTK, or distributed VTK format (\url{http://www.vtk.org/VTK/img/file-formats.pdf}). The output data format is defined by \verb|fld_format|. Visualization applications which we checked are listed in Table \ref{table:Viz_app}. Because the field data is written by using Cartesian coordinate $(x,y,z)$ system, coordinate conversion is required to plot vector field in spherical coordinate $(r, \theta, \phi)$ or cylindrical coordinate $(s,\phi, z)$. We will introduce a example of visualization process using ParaView in Section \ref{sec:paraview}. Field data also output merged ASCII or binary format including compression using zlib. These original formats have smaller file size than VTK format because of excluding grid information. Program \hyperref[sec:field_to_VTK]{\tt field\_to\_VTK} generates VTK file from FEM mesh data and field data.
%
\begin{table}[htp]
\caption{Checked visualization application}
\label{table:Viz_app}
\begin{center} 
\begin{tabular}{|c|c|c|}
\hline
Control flag & \verb|fld_format| & Application \\ \hline \hline
\verb|VTK| & Distributed VTK & ParaView \\ \hline
\verb|single_VTK| & Merged VTK & ParaView, VisIt, or Mayavi \\ \hline
\verb|VTK_gzip| & Compressed Distributed VTK & ParaView \\
 & & after expanding by {\tt gzip} \\ \hline
\verb|single_VTK_gz| & Compressed Merged VTK & ParaView, VisIt or Mayavi \\
 & &  after expanding by {\tt gzip} \\ \hline
\verb|single_HDF5| & XDMF   & ParaView, VisIt   \\ \hline
\verb|ascii| & Distributed ASCII & - \\
\verb|binary| & Distributed binary & - \\
\verb|gzip| & Distributed compressed ASCII & - \\
\verb|bin_gz| & Distributed compressed binary & - \\
\verb|merged| & Merged ASCII & - \\
\verb|merged_bin| & Merged binary & - \\
\verb|merged_gzip| & Merged compressed ASCII & - \\
\verb|merged_bin_gz| & Merged compressed binary & - \\ \hline
\end{tabular}
\label{table:fld_to_vtk}
\end{center}
More informations about ParaView is in \url{https://www.paraview.org}. \\
More informations about VisIt is in \url{https://wci.llnl.gov/codes/visit/}. \\
More informations about Mayavi is in \url{http://mayavi.sourceforge.net/}. \\
\end{table}
%

\paragraph{Distributed VTK data}
Distributed VTK data have the following advantage and disadvantages to use:
%
\begin{itemize}
\item Advantage
\begin{itemize} 
\item Faster output
\item No external library is required
\end{itemize}
\item Disadvantage
\begin{itemize} 
\item Many data files are generated
\item Total data file size is large
\item Only ParaView supports this format
\end{itemize}
\end{itemize}
%
Distributed VTK data consist files listed in Table \ref{table:parallel_vtk}. For ParaView, all subdomain data is read by choosing \verb|[fld_prefix].[step#].pvtk| in file menu.
%
\begin{table}[htp]
\caption{List of written files for distributed VTK format}
\begin{center} 
\begin{tabular}{|c|c|}
\hline
 name &  \\ \hline \hline
\verb|[fld_prefix].[step#].[domain#].vtk|  & VTK data for each subdomain  \\ \hline
\verb|[fld_prefix].[step#].pvtk| & Subdomain file list for Paraview  \\ \hline
\end{tabular}
\end{center}
\label{table:parallel_vtk}
\end{table}
%

\paragraph{Merged VTK data}
Merged VTK data have the following advantage and disadvantages to use:
%
\begin{itemize}
\item Advantage
\begin{itemize} 
\item Merged field data is generated
\item No external library is required
\item Many applications support VTK format
\end{itemize}
\item Disadvantage
\begin{itemize} 
\item Very slow to output
\item Total data file size is large
\end{itemize}
\end{itemize}
%
Merged VTK data generate files listed in Table \ref{table:Merged_vtk}. 
%
\begin{table}[htp]
\caption{List of written files for merged VTK format}
\begin{center} 
\begin{tabular}{|c|c|}
\hline
 name &  \\ \hline \hline
\verb|[fld_prefix].[step#].vtk|  & Merged VTK data  \\ \hline
\end{tabular}
\end{center}
\label{table:Merged_vtk}
\end{table}
%

\paragraph{Merged XDMF data}
Merged XDMF data have the following advantage and disadvantages to use:
%
\begin{itemize}
\item Advantage
\begin{itemize} 
\item Fastest output
\item Merged field data is generated
\item File size is smaller than the VTK formats
\end{itemize}
\item Disadvantage
\begin{itemize} 
\item Parallel HDF5 library should be required to use
\end{itemize}
\end{itemize}
%
Merged XDMF data generate files listed in Table \ref{table:XDMF}. For ParaView, all subdomain data is read by choosing \verb|[fld_prefix].solution.xdmf| in file menu.
%
\begin{table}[htp]
\caption{List of written files for XDMF format}
\begin{center} 
\begin{tabular}{|c|c|}
\hline
 name &  \\ \hline \hline
\verb|[fld_prefix].mesh.h5|  & HDF5 file for geometry data \\ \hline
\verb|[fld_prefix].[step#].h5|  &HDF5 file for field data   \\ \hline
\verb|[fld_prefix].solution.xdmf|  & HDF5 file lists to be read  \\ \hline
\end{tabular}
\end{center}
\label{table:XDMF}
\end{table}
%
\paragraph{Calypso field data}
Calypso field data is based on the spectr data for restarting. The data is simply replaced from spherical harmonics coefficients to each component of field data in the cartesian coordinate. The file format flag \verb|[field_file_fmt_ctl]| and corresponding extensiton are showw in Table \ref{table:field_format}.
%
%
\begin{table}[htp]
\caption{Data format flag {[\tt field\_file\_fmt\_ctl]} and extensions for the field file. An example of data size and output time is also listed.}
\begin{center} 
\begin{tabular}{|c|c|c|c|c|}
\hline
File format & \verb|[field_file_fmt_ctl]|  & \verb|extension| & size (GByte) & time (sec)  \\ \hline \hline
              &  \verb|ascii| & \verb|[#].fld|  & 9.31 & 61.1 \\
              &  \verb|binary| & \verb|[#].flb|  & 2.93 & 39.4  \\
Parallel  &  \verb|gzip| &  \verb|[#].fld.gz| & 3.25 &  40.6 \\
              &  \verb|bin_gz| &\verb|[#].flb.gz| & 2.78 &  39.6\\
              &  \verb|VTK| & \verb|[#].vtk|  & 12.4 & 1030.5 \\
              &  \verb|VTK_gz| & \verb|[#].vtk.gz|  & 3.36 & 40.8 \\ \hline
              & \verb|merged|  &  \verb|.fld| & 9.53 & 39.1 \\
 Merged &  \verb|merged_bin| &  \verb|.flb| & 3.58 & 41.7  \\
              &  \verb|merged_gz| &  \verb|.fld.gz| & 2.99 &  39.4  \\
              &  \verb|merged_bin_gz| &  \verb|.flb.gz| & 2.84 &  39.2 \\
              &  \verb|merged_VTK| & \verb|.vtk|  & 11.8 & 39.3 \\
              &  \verb|merged_VTK_gz| & \verb|.vtk.gz|  & 3.13 & 39.0 \\  \hline
\end{tabular}
\end{center}
\label{table:field_format}
\end{table}
%

%
\subsection{Cross section data (Parallel Surfacing module)}
\label{section:PSF}
Calypso can output cross section data for visualization with finer time increment than the whole domain data. The cross section data consist of triangle patches with VTK format, then data can be visualized by Paraview like as the whole field data. This cross sectioning module can output arbitrary quadrature surface, but plane, sphere, and cylindrical section would be useful for the geodynamo simulations.

To output cross sectioning, increment of the surface output data should be defined by \verb|i_step_sectioning_ctl| in \verb|time_step_ctl| block. And, array block \\ \verb|cross_section_ctl| in \verb|visual_control| section is required to define cross sections. Each \verb|cross_section_ctl| block defines one cross section. Each cross section can also define by an external file by specifying external file name with \verb|file| label.
%
The sections shown in Table \ref{table:section_list} are supported in the sectioning module. These surfaces are defined in the Cartesian coordinate.
\begin{table}[htp]
\caption{Supported cross sections}
\begin{center}
\begin{tabular}{|c|c|}
\hline
Surface type & equation \\ \hline
Quadrature surface 
 & $a x^2 + b y^2 + c z^2 + d y z + e z x + f x y + g x + h y + j z + k = 0$ \\
Plane surface 
& $a \left(x-x_{0} \right) + b \left(y-y_{0} \right) + c \left(z-z_{0} \right) = 0$ \\
 Sphere 
& $\left(x-x_{0} \right)^2 + \left(y-y_{0} \right)^2 + \left(z-z_{0} \right)^2 = r^2$  \\
 Ellipsoid 
& $\displaystyle{ \left(\frac{x-x_{0}}{a} \right)^2 + \left( \frac{y-y_{0}}{b} \right)^2 + \left( \frac{z-z_{0}}{c} \right)^2} = 1$ \\
\hline
\end{tabular}
\end{center}
\label{table:section_list}
\end{table}
%
The easiest approarch is using sections defined by quadrature function with ten coefficients from $a$ to $k$ in the control array \verb|coefs_ctl|.

A plane surface is defined by a normal vector $(a, b, c)$ and one point including the surface $(x_{0}, y_{0}, z_{0})$ in arrays \verb|normal_vector| and \verb|center_position|, respectively.

A sphere surface is defined by the position of the center $(x_{0}, y_{0}, z_{0})$ and radius $r$ in array \verb|center_position| and \verb|radius|, respectively.

An Ellipsoid surface is defined the position of the center $(x_{0}, y_{0}, z_{0})$ and length of the each axis $(a, b, c)$ in arrays  \verb|center_position| and \verb|axial_length|, respectively. If one component of the \verb|axial_length| is set to 0, surfacing module generate a Ellipsoidal tube along with the axis where \verb|axial_length| is set to 0.

Area for visualization can be defined by array \verb|chosen_ele_grp_ctl| by choosing \verb|outer_core|, \verb|inner_core|, and \verb|all|. Fields to display is defined in array \verb|output_field|. In array \verb|output_field|, field type in Table \ref{table:field_type} needs to defined. The same field can be defined more than once in array \verb|output_field| to output vector field in Cartesian coordinate and radial component, for example.
%
\begin{table}[htp]
\caption{List of field type for cross sectioning and isosurface module}
\label{table:field_type}
\begin{center} 
\begin{tabular}{|c|c|}
\hline
 Definition & Field type  \\ \hline \hline
 \verb|scalar| & scalar field  \\ \hline
 \verb|vector| & Cartesian vector field \\ \hline
 \verb|x| & $x$-component  \\ \hline
 \verb|y| & $y$-component  \\ \hline
 \verb|z| & $z$-component  \\ \hline
 \verb|radial| & radial ($r$-) component  \\ \hline
 \verb|theta| & $\theta$-component  \\ \hline
 \verb|phi| & $\phi$-component  \\ \hline
 \verb|cylinder_r| & cylindrical radial ($s$-) component  \\ \hline
 \verb|magnitude| & magnitude of vector  \\ \hline
\end{tabular}
\end{center}
\end{table}
%
\paragraph{Control data} \label{section:section_control}
The format of the control file or block for cross sections is described below. The detail of each block is described in section \ref{section:def_control}.  \verb|cross_section_ctl| block can be read from an external file. 

To define the external file name, as \verb|file cross_section_ctl [file name]| in \verb|control_MHD| or \verb|control_snapshot|. \\
\\
%
Block \verb|cross_section_ctl| (Top level for sectioning)
\label{href_i:cross_section_ctl}
\begin{itemize}
	\item \hyperref[href_t:section_file_prefix]
			{\tt section\_file\_prefix    [section\_prefix]}
	\item \hyperref[href_t:psf_output_type]
			{\tt psf\_output\_type        [file\_format]}
	\item Block \hyperref[href_t:surface_define]{\tt surface\_define}
		\begin{itemize}
			\item \hyperref[href_t:section_method]
				{\tt section\_method    [METHOD]}
			\item Array \hyperref[href_t:psf_coefs_ctl]
				{\tt coefs\_ctl        [TERM]         [COEFFICIENT]}
			\item \hyperref[href_t:psf_radius]{\tt radius    [SIZE]}
			\item Array \hyperref[href_t:psf_normal_vector]
				{\tt normal\_vector    [DIRECTION]    [COMPONENT]}
			\item Array \hyperref[href_t:psf_axial_length]
                {\tt axial\_length     [DIRECTION]    [COMPONENT]}
			\item Array \hyperref[href_t:psf_center_position]
                {\tt center\_position  [DIRECTION]    [COMPONENT]}
%
			\item Array \hyperref[href_t:section_area_ctl]
				{\tt section\_area\_ctl        [AREA\_NAME]}
	\end{itemize}
%
	\item \hyperref[href_t:output_field_define]{\tt output\_field\_define}
		\begin{itemize}
			\item Array \hyperref[href_t:psf_output_field]
                {\tt output\_field     [FIELD]    [COMPONENT]}
		\end{itemize}
\end{itemize}

\paragraph{Output data format of sectioning module}
\label{sec:PSF_data}
Sectioning data are written with VTK format and VTK data compressed by zlib. Field data also output by binary format and binary compressed by zlib. The list of data format and control flag for \verb|psf_output_type| are listed in Table \ref{table:PSF_data}. In the binary data format, position data and field data are saved independently not to write the grid data for each output step. Program \hyperref[section:section_to_VTK]{\tt section\_to\_VTK} generates VTK file from the binary section data. The output data format is defined by \verb|psf_output_type|. Because the field data is written by using Cartesian coordinate $(x,y,z)$ system, $(x,y,z)$ components in ParaView corresponds to the spherical components $(r, \theta, \phi)$ or cylindrical componennts $(s,\phi, z)$ if sectioning data is writtein the spherical or cylindrical componnents. Consequently, ParaView can not draw griph or field lines for these spherical or cylindrical vectors.
%
\begin{table}[htp]
\caption{Data format  and an example of data size and output time  for sectioning data}
\label{table:PSF_data}
\begin{center} 
\begin{tabular}{|c|c|c|c|c|}
\hline
File format & \verb|[fld_format]| & \verb|extension| & size (MByte) & time (sec)  \\ \hline \hline
VTK & \verb|VTK| & \verb|.vtk| & 29.1 & 2.88 \\ \hline
Compressed VTK & \verb|VTK_gzip| & \verb|.vtk.gz| & 6.04 & 1.29 \\ \hline
Binary & \verb|PSF| & \verb|0.sgd| (grid) & 5.30 &  \\
           &        &  \verb|.sdt| (field) & 4.13 & 2.59 \\ \hline
Compressed binary & \verb|PSF_gzip| & \verb|.0.sgd.gz| (grid) & 1.17 & \\
           &        &  \verb|.sdt.gz| (field) & 3.94 & 1.21 \\ \hline
\end{tabular}
\end{center} 
\end{table} 

%
\subsection{Isosurface data}
\label{section:ISO}
%
Calypso can also output isosurface data for visualization. Generally, data size of the isosurface is much larger than the sectioning data. The isosurface data is also written as a unstructured grid data with VTK format. The isosurface also consists of triangle patches.

To output cross sectioning, increment of the surface output data should be defined by \verb|i_step_isosurface_ctl| in \verb|time_step_ctl| block. And, array block \verb|isosurface_ctl| in \verb|visual_control| section is required to define cross sections. Each \verb|isosurface_ctl| block defines one cross section. Each cross section can also define by an external file by specifying external file name with \verb|file| label.
%
\paragraph{Control data}  \label{section:isosurface_control}
The format of the control file or block for isosurfaces is described below. The detail of each block is described in section \ref{section:def_control}.  \verb|isosurface_ctl| block can be read from an external file. To define the external file name, as \verb|file isosurface_ctl [file name]| in \verb|control_MHD| or \verb|control_snapshot|. \\
\\
%
Block \verb|isosurface_ctl|  (Top lebel of the control data)
\label{href_i:isosurface_ctl}
\begin{itemize}
	\item \hyperref[href_t:isosurface_file_prefix]
			{\tt isosurface\_file\_prefix    [file\_prefix]}
	\item \hyperref[href_t:iso_output_type]
			{\tt iso\_output\_type           [file\_format]}
%
	\item Block \hyperref[href_t:isosurf_define]{\tt isosurf\_define}
		\begin{itemize}
			\item \hyperref[href_t:isosurf_field]{\tt isosurf\_field    [FIELD]}
			\item \hyperref[href_t:isosurf_component]
						{\tt isosurf\_component    [COMPONENT]}
			\item \hyperref[href_t:isosurf_value]{\tt isosurf\_value    [VALUE]}
%
			\item Array \hyperref[href_t:isosurf_area_ctl]
						{\tt isosurf\_area\_ctl  [AREA\_NAME]}
		\end{itemize}
%
	\item Block \hyperref[href_t:field_on_isosurf]{\tt field\_on\_isosurf}
		\begin{itemize}
			\item \hyperref[href_t:result_type]{\tt result\_type    [TYPE]}
			\item \hyperref[href_t:result_value]{\tt result\_value    [VALUE]}
			\item Array \hyperref[href_t:iso_output_field]
						{\tt output\_field        [FIELD]    [COMPONENT]}
		\end{itemize}
\end{itemize}

\paragraph{Output data format of isosurface module}
\label{sec:PSF_data}
Isosurface data are written with VTK format and VTK data compressed by zlib. Field data also output by binary format and binary compressed by zlib. The list of data format and control flag for \verb|iso_output_type| are listed in Table \ref{table:ISO_data}. Like as sectioning data, program \hyperref[section:section_to_VTK]{\tt section\_to\_VTK} generates VTK file from the binary section data. The output data format is defined by \verb|iso_output_type|. Because the field data is written by using Cartesian coordinate $(x,y,z)$ system, $(x,y,z)$ components in ParaView corresponds to the spherical components $(r, \theta, \phi)$ or cylindrical componennts $(s,\phi, z)$ if sectioning data is writtein the spherical or cylindrical componnents. Consequently, ParaView can not draw griph or field lines for these spherical or cylindrical vectors.
%
\begin{table}[htp]
\caption{Data format and an example of data size and output time for isosurface data}
\label{table:ISO_data}
\begin{center} 
\begin{tabular}{|c|c|c|c|c|}
\hline
File format & \verb|[fld_format]| & \verb|extension| &  Size  (MByte) & Time (sec)   \\ \hline \hline
VTK &\verb|VTK| &  \verb|.vtk| & 172 & 1.88 \\ \hline
Compressed VTK &\verb|VTK_gzip| &  \verb|.vtk.gz| & 38.8 & 2.92  \\ \hline
Binary &\verb|ISO| &  \verb|.sfm| & 55.7 & 2.22 \\ \hline
Compressed binary & \verb|ISO_gzip| & \verb|.sfm.gz| & 35.0 &  1.73  \\ \hline
\end{tabular}
\end{center} 
\end{table} 
%

\subsection{Time history data for monitoring}
Calypso also generate various time history outputs data.These files are written by the SCII format or compressed format by using zlib library. These compressed files can be expanded by {\tt gzip} commnd. 
If the time history data files exist before starting the simulation, programs append results at the end of files without checking constancy of the number of data and order of the field. If you change the configuration of data output structure, please move the existed data files to another directory before starting the programs.

\subsubsection{Layered spectrum data}
\label{section:layerd_spectr}
Spectrum data for the each radial position are written by defining {\tt layered\_pwr\_spectr\_prefix} in control file. By defining {\tt layered\_pwr\_spectr\_prefix}, following spectrum data averaged over the fluid shell is written. Data format is the same as the volume spectrum data, but radial grid point and radius of the layer is added in the list. The following files are generated. The radial points for output is listed in the array \verb|spectr_layer_ctl|. If \verb|spectr_layer_ctl| is not defined, mean square data at {\bf all} radial levels will be written. See example of \hyperref[section:dynamobench]{dynamo benchmark case 2}. Header of the layered spectrum data file consists of 
%
\begin{description}
\item{\tt  line 2: } Number of radial grid and truncation level
\item{\tt  line 4: } Radial layer ID for ICB and CMB
\item{\tt  line 6: } Number of field of data, total number of components
\item{\tt  line 8: } Number of components for each field
\end{description}
%
Field labels indicates as following:
%
\begin{description}
\item{\tt  t\_step}  Time setp number
\item{\tt  time}   Time
\item{\tt  K\_ene\_pol}  Amplitude of poloidal kinetic energy
\item{\tt  K\_ene\_tor}  Amplitude of toroidal kinetic energy
\item{\tt  K\_ene}       Amplitude of total kinetic energy
\item{\tt  M\_ene\_pol}  Amplitude of poloidal magnetic energy
\item{\tt  M\_ene\_tor}  Amplitude of toroidal magnetic energy
\item{\tt  M\_ene}       Amplitude of total magnetic energy
\item{\tt  [Field]\_pol} Mean square amplitude of poloidal component of {\tt [Field]}
\item{\tt  [Field]\_tor} Mean square amplitude of toroidal component of {\tt [Field]}
\item{\tt  [Field]}      Mean square amplitude of {\tt [Field]}
\end{description}
%
In vector fields, the kinetic energy $u^{2} / 2$ and magnetic energy $B^{2} / 2$ are calculated instead of mean square amplitude for the velocity $\bvec{u}$ and magnetic field $\bvec{B}$. Headers on the first lines indicate following data. The following mean square data is generated:

%
\begin{description}
\item{\tt [layer\_pwr\_prefix]\_s.dat[.gz]} Surface average of mean square amplitude of the fields. the mean square for the scalar field $f$ is evluated from its spherical harmonic coefficients by
%
\begin{equation}
<f^{2}(r)> = \sum_{l}\sum_{m}  \frac{1}{2l + 1} \left(f_{l}^{m}(r) \right)^2.
\end{equation}
%
And, the mean square of the vector field $\bvec{A}$ evaluated by
%
\begin{eqnarray}
<\bvec{A}_{S}^{2}(r)> &=& \sum_{l}\sum_{m} \frac{1}{2l + 1}
\left[  \left(\frac{l\left(l+1 \right) }{r^{2}} A_{Sl}^{\ m}(r) - \frac{\partial \phi_{l}^{m}}{\partial r} \right)^{2} 
\right. \nonumber \\
& & \left. +  \frac{l \left( l+1 \right)}{r^{2}} \left( \frac{\partial A_{Sl}^{\ m}}{\partial r} - \phi_{l}^{m} \right)^{2} \right],
\end{eqnarray}
%
and
%
\begin{eqnarray}
<\bvec{A}_{T}^{2}(r)> &=& \sum_{l}\sum_{m} \frac{l \left( l+1 \right)}{2l + 1} \frac{1}{r^{2}} \left( A_{Tl}^{\ m}(r) \right)^{2}.
\end{eqnarray}
Finally, the total mean square amplitude $<\bvec{A}^{2}> $ is obtained by
\begin{eqnarray}
<\bvec{A}^{2}(r)> &=& <\bvec{A}_{S}^{2}(r)>  + <\bvec{A}_{T}^{2}(r)> .
\end{eqnarray}
%
%
\item{\tt [layer\_pwr\_prefix]\_l.dat[.gz]} Surface average of mean square amplitude of the fields as a function of spherical harmonic degree $l$ and radial grid id $k$. For scalar field, the spectrum is
%
\begin{equation}
< f_{l}^{\ 2}(r,l)>  = \sum_{m=-l}^{l}  \frac{1}{2l + 1} \left(f_{l}^{m}(r) \right)^2.
\end{equation}
%
For vector field, spectrum for the poloidal and toroidal components are written by 
%
\begin{eqnarray}
<\bvec{A}_{Sl}^{\ 2}(r,l)> &=&  \sum_{m=-l}^{l}  \frac{1}{2l + 1}
\left[  \left(\frac{l\left(l+1 \right) }{r^{2}} A_{Sl}^{\ m}(r) - \frac{\partial \phi_{l}^{m}}{\partial r} \right)^{2} 
\right. \nonumber \\
& & \left. +  \frac{l \left( l+1 \right)}{r^{2}} \left( \frac{\partial A_{Sl}^{\ m}}{\partial r} - \phi_{l}^{m} \right)^{2} \right],
\end{eqnarray}
%
and
\begin{eqnarray}
<\bvec{A}_{Tl}^{\ 2}(r,l)>  &=& \sum_{m=-l}^{l} \frac{l \left( l+1 \right)}{2l + 1} \frac{1}{r^{2}} \left( A_{Tl}^{\ m}(r) \right)^{2}.
\end{eqnarray}
Finally, the total mean square amplitude $<\bvec{A}^{2}> $ is obtained by
\begin{eqnarray}
<\bvec{A}_{l}^{2}(r,l)> &=& <\bvec{A}_{Sl}^{\ 2}(r,l)>  + <\bvec{A}_{Tl}^{\ 2}(r,l)> .
\end{eqnarray}

\item{\tt [layer\_pwr\_prefix]\_m.dat} Surace average of mean square amplitude of the fields as a function of spherical harmonic order $m$ and radial grid id $k$. The zonal wave number is referred in this spectrum data. For scalar field, the spectrum is
%
\begin{equation}
<f_{m}^{2}(r,m)> = \sum_{l=m}^{L_{max}}  \frac{1}{2l + 1} \left[ \left(f_{l}^{m}(r) \right)^2 
+ \left(f_{l}^{-m}(r) \right)^2 \right]
\end{equation}
%
For vector field, spectrum for the poloidal and toroidal components are written by 
%
\begin{eqnarray}
<\bvec{A}_{Sm}^{\ 2}(r,m)> &=&   \sum_{l=m}^{L_{max}} \frac{1}{2l + 1}
\left[  \left(\frac{l\left(l+1 \right) }{r^{2}} A_{Sl}^{\ m}(r) - \frac{\partial \phi_{l}^{m}}{\partial r} \right)^{2} 
\right. \nonumber \\
& & + \left(\frac{l\left(l+1 \right) }{r^{2}} A_{Sl}^{\ -m}(r) - \frac{\partial \phi_{l}^{-m}}{\partial r} \right)^{2} 
\nonumber \\
& & +  \frac{l \left( l+1 \right)}{r^{2}} \left( \frac{\partial A_{Sl}^{\ m}}{\partial r} - \phi_{l}^{m} \right)^{2} 
\nonumber \\
& & \left. +  \frac{l \left( l+1 \right)}{r^{2}} \left( \frac{\partial A_{Sl}^{\ -m}}{\partial r} - \phi_{l}^{-m} \right)^{2} \right],
\end{eqnarray}
%
and
\begin{eqnarray}
<\bvec{A}_{Tm}^{\ 2}(r,m)> &=&  \sum_{l=m}^{L_{max}} \frac{l \left( l+1 \right)}{2l + 1} \frac{1}{r^{2}} 
\left[ \left( A_{Tl}^{\ m}(r) \right)^{2} +  \left( A_{Tl}^{\ -m}(r) \right)^{2} \right].
\end{eqnarray}
Finally, the total mean square amplitude $<\bvec{A}^{2}> $ is obtained by
\begin{eqnarray}
<\bvec{A}_{m}^{\ 2}(r,m)> &=& <\bvec{A}_{Sm}^{\ 2}(r,m)>  + <\bvec{A}_{Tm}^{\ 2}(r,m)>.
\end{eqnarray}

\item{\tt [layer\_pwr\_prefix]\_lm.dat} Surface average of mean square amplitude of the fields as a function of spherical harmonic order $n = l-m$ and radial grid id $k$. The wave number in the latitude direction is referred in this spectrum data.
For scalar field, the spectrum is
%
\begin{equation}
<f_{n}^{2}(r,n)> = \sum_{m=n}^{L_{max}} \frac{1}{2l + 1} \left[ \left(f_{l}^{l-n}(r) \right)^2 
+ \left(f_{l}^{-l+n}(r) \right)^2 \right]
\end{equation}
%
For vector field, spectrum for the poloidal and toroidal components are written by 
%
\begin{eqnarray}
<\bvec{A}_{Sn}^{\ 2}(r,n)> &=&   \sum_{m=n}^{L_{max}}\frac{1}{2l + 1}
\left[  \left(\frac{l\left(l+1 \right) }{r^{2}} A_{Sl}^{\ l-n}(r) - \frac{\partial \phi_{l}^{l-n}}{\partial r} \right)^{2} 
\right. \nonumber \\
& & + \left(\frac{l\left(l+1 \right) }{r^{2}} A_{Sl}^{\ -l+n}(r) - \frac{\partial \phi_{l}^{-l+n}}{\partial r} \right)^{2} 
\nonumber \\
& & +  \frac{l \left( l+1 \right)}{r^{2}} \left( \frac{\partial A_{Sl}^{\ l-n}}{\partial r} - \phi_{l}^{l-n} \right)^{2} 
\nonumber \\
& & \left. +  \frac{l \left( l+1 \right)}{r^{2}} \left( \frac{\partial A_{Sl}^{\ -l+n}}{\partial r} - \phi_{l}^{-l+n} \right)^{2} \right],
\end{eqnarray}
%
and
\begin{eqnarray}
<\bvec{A}_{Tn}^{\ 2}(r,n)> &=&  \sum_{m=n}^{L_{max}}\frac{l \left( l+1 \right)}{2l + 1} \frac{1}{r^{2}} 
\left[ \left( A_{Tl}^{\ l-n}(r) \right)^{2} +  \left( A_{Tl}^{\ -l+n}(r) \right)^{2} \right].
\end{eqnarray}
Finally, the total mean square amplitude $<\bvec{A}^{2}> $ is obtained by
\begin{eqnarray}
<\bvec{A}_{n}^{\ 2}(r,n)> &=& <\bvec{A}_{Sn}^{\ 2}(r,n)>  + <\bvec{A}_{Tn}^{\ 2}(r,n)> 
\end{eqnarray}
%
\item{\tt [layer\_pwr\_prefix]\_m0.dat} Surace average of mean square amplitude of the axisymmetric components of fields as a function of radial grid id $k$. The zonal wave number is referred in this spectrum data. For scalar field, the spectrum is
%
\begin{equation}
<f^{2}(r)>_{axis} = \sum_{l=0}^{L_{max}}  \frac{1}{2l + 1} \left(f_{l}^{0}(r) \right)^2
\end{equation}
%
For vector field, spectrum for the poloidal and toroidal components are written by 
%
\begin{eqnarray}
<\bvec{A}_{S}^{2}(r)>_{axis} &=&   \sum_{l=0}^{L_{max}} \frac{1}{2l + 1}
\left[  \left(\frac{l\left(l+1 \right) }{r^{2}} A_{Sl}^{\ 0}(r) - \frac{\partial \phi_{l}^{0}}{\partial r} \right)^{2} 
\right. \nonumber \\
& & \left. +  \frac{l \left( l+1 \right)}{r^{2}} \left( \frac{\partial A_{Sl}^{\ 0}}{\partial r} - \phi_{l}^{0} \right)^{2} \right],
\end{eqnarray}
%
and
\begin{eqnarray}
<\bvec{A}_{T}^{2}(r)>_{axis} &=&  \sum_{l=0}^{L_{max}} \frac{l \left( l+1 \right)}{2l + 1} \frac{1}{r^{2}} 
\left( A_{Sl}^{\ 0}(r) \right)^{2}.
\end{eqnarray}
Finally, the total mean square amplitude $<\bvec{A}^{2}> $ is obtained by
\begin{eqnarray}
<\bvec{A}^{2}(r)>_{axis} &=& <\bvec{A}_{S}^{2}(r)>_{axis}  + <\bvec{A}_{T}^{2}(r)>_{axis} .
\end{eqnarray}

\end{description}


\subsubsection{Mean square amplitude data}
\label{sec:mean_square}
This program output mean square amplitude of the fields which is marked as \verb|Monitor_ON| over the fluid shell at every \verb|[increment_monitor]| steps. The data is written in the file \verb|[vol_pwr_prefix]_s.dat[.gz]| or  \verb|sph_pwr_volume_s.dat| if \\
{\tt [vol\_pwr\_prefix]} is not defined in the control file. The mean square amplitude ifor a scalar field $f$ is defined by
%
\begin{equation}
<f^{2}> =   \frac{4\pi}{V} \int <f^{2}(r)> r^{2} dr.
\end{equation}
%
For the vector field $\bvec{A}$, the mean square of the potential component $\bvec{A}_{S}$ is included in the mean square of the poloidal component in the data if the vector field is not a solenoidal field. Consequently, $\bvec{A}_{S}$ and mean square of the toroidal component $\bvec{A}_{T}$ are evaluated by 
%
\begin{eqnarray}
<\bvec{A}_{S}^{2}> &=& \frac{4\pi}{V} \int <\bvec{A}_{S}^{2}(r)> r^{2} dr,
\nonumber \\
<\bvec{A}_{T}^{2}>  &=& \frac{4\pi}{V} \int <\bvec{A}_{T}^{2}(r)>  r^{2} dr,
\end{eqnarray}
% 
and
%
\begin{eqnarray}
<\bvec{A}^{2}> &=& \frac{4\pi}{V} \int <\bvec{A}^{2}(r)>  r^{2} dr.
\label{eq:vol_vec_msq}
\end{eqnarray}

The header in the first 12 lines is the following. 
%
\begin{description}
\item{\tt  line 2: } Number of radial grid and truncation level
\item{\tt  line 4: } Radial layer ID for ICB and CMB
\item{\tt  line 6: } Radial layer ID and radius for the inner boundary of integration
\item{\tt  line 8: } Radial layer ID and radius for the outer boundary of integration
\item{\tt  line 10: } Number of field of data, total number of components
\item{\tt  line 11: } Number of components for each field
\end{description}
%
The following is an example of the beginning of the data file:
{\scriptsize
\begin{verbatim}
Radial_layers, Truncation
             112             159
ICB_id, CMB_id
               1             112
Lower_boundary_ID, Lower_boundary_radius
               1   5.384615384615384E-001
Upper_boundary_ID, Upper_boundary_radius
             112   1.538461538461538E+000
Number_of_fields, Number_of_components
               8              16
    3    1    3    1    3    3    1    1
t_step    time    K_ene_pol    K_ene_tor    K_ene    temperature    vorticity_pol
    vorticity_tor    vorticity    pressure    M_ene_pol    M_ene_tor    M_ene
    current_density_pol    current_density_tor    current_density    buoyancy_flux    
    Lorentz_work    
             100  2.00000000000000E-004  3.08485175580558E+001  2.97052514317492E-001
  3.11455700723732E+001  8.37549401638792E-002  3.83545285358558E+001  
  4.45268846884004E+003  4.49104299737589E+003  2.20356357802801E+001  
  2.88810032145648E-006  1.76927464999397E-006  4.65737497145044E-006  
  4.45474362010442E-005  3.76310585686982E-005  8.21784947697424E-005  
  1.47655914519649E+013  1.52720673275983E-003
             200  3.99999999999999E-004  1.21246529508559E+002  4.60310202818333E+000  
  1.25849631536743E+002  8.38032562478558E-002  5.26153720206686E+002  
  1.51087784591356E+004  1.56349321793422E+004  4.32047730989212E+001  
  2.88721662147718E-006  1.76772446160287E-006  4.65494108308005E-006  
  4.49318945325754E-005  3.83578033623782E-005  8.32896978949536E-005  
  5.86224313684072E+013  6.13459599464992E-003
\end{verbatim}
}
%
\subsubsection{Volume average data}
Volume average data are written by defining {\tt volume\_average\_prefix} in control file. Volume average data are written in \verb|[vol_ave_prefix].dat.[gz]| with the same format as the men square amplitude data in Subsection \ref{sec:mean_square}. The volume average of scalar field $f$ is evaluated by the 0-th degree of the spherical harmonics coefficients $f_{0}^{0}$ as
%
\begin{equation}
<f> =   \frac{4\pi}{V} \int f_{0}^{0}(r) r^{2} dr.
\end{equation}
%
The average of the toroidal component of the vector field $<\bvec{A}_{T}>$ is always zero, and the average of poloidal component of the solenoidal vector  $<\bvec{A}_{S}>$ is also zero. If the vector is non-soleinoidal, 0-th degree of the potential field contributes the average radial vector as
%
\begin{eqnarray}
<\bvec{A}> &=& <\bvec{A}_{S}>  =  - \frac{4\pi}{V}
\int \frac{\partial \phi_{0}^{0}}{\partial r} r^{2} dr.
\end{eqnarray}
%

If you need the sphere average data for specific radial point, you can use picked spectrum data output in \hyperref[sec:pickup_spectr_ctl]{\tt pickup\_spectr\_ctl}  using $l = m = 0$ at specific radius.

\subsubsection{Volume spectrum data}
Volume spectrum data are written by defining {\tt volume\_pwr\_spectr\_prefix} in control file. By defining {\tt volume\_pwr\_spectr\_prefix}, following spectrum data averaged over the fluid shell is written. Data header format is the same as the volume mean square data, but degree $l$, order $m$, or meridional wave number $l-m$ is added in the list of data in each step. \\
%
\begin{description}
\item{\tt [vol\_pwr\_prefix\_l.dat[.gz]}  Volume average of mean square amplitude of the fields as a function of spherical harmonic degree $l$. For scalar field, the spectrum is
%
\begin{eqnarray}
<f_{l}^{2}(l)> =   \frac{4\pi}{V} \int <f_{l}^{2}(r,l)> r^{2} dr.
\end{eqnarray}
%
For vector field, spectrum for the poloidal and toroidal components are written by 
%
\begin{eqnarray}
<A_{Sl}^{\ 2} > & = &  \frac{4\pi}{V} \int <A_{Sl}^{\ 2}(r,l) >  r^{2} dr,
\nonumber \\
<A_{Tl}^{\ 2} > & = &  \frac{4\pi}{V} \int <A_{Tl}^{\ 2}(r,l) >  r^{2} dr,
\end{eqnarray}
and
\begin{eqnarray}
<A_{l}^{\ 2} > & = &  \frac{4\pi}{V} \int <A_{l}^{\ 2}(r,l) >  r^{2} dr,
\label{eq:vol_vec_msq_l}
\end{eqnarray}

\item{\tt [vol\_pwr\_prefix]\_m.dat[.gz]} Volume average of mean square amplitude of the fields as a function of spherical harmonic order $m$. The zonal wave number is referred in this spectrum data. For scalar field, the spectrum is
%
\begin{eqnarray}
<f_{m}^{2}(m)> =   \frac{4\pi}{V} \int <f_{m}^{2}(r,m)> r^{2} dr.
\end{eqnarray}
%
For vector field, spectrum for the poloidal and toroidal components are written by 
%
\begin{eqnarray}
<A_{Sm}^{\ 2} > & = &  \frac{4\pi}{V} \int <A_{Sm}^{\ 2}(r,m) >  r^{2} dr,
\nonumber \\
<A_{Tm}^{\ 2} > & = &  \frac{4\pi}{V} \int <A_{Tm}^{\ 2}(r,m) >  r^{2} dr,
\end{eqnarray}
and
\begin{eqnarray}
<A_{m}^{\ 2} > & = &  \frac{4\pi}{V} \int <A_{m}^{\ 2}(r,m) >  r^{2} dr,
\label{eq:vol_vec_msq_m}
\end{eqnarray}


\item{\tt [vol\_pwr\_prefix]\_lm.dat[.gz]} Volume average of mean square amplitude of the fields as a function of spherical harmonic order $n = l-m$. The wave number in the latitude direction is referred in this spectrum data. For scalar field, the spectrum is
%
\begin{eqnarray}
<f_{n}^{2}(n)> =   \frac{4\pi}{V} \int <f_{n}^{2}(r,n)> r^{2} dr.
\end{eqnarray}
%
For vector field, spectrum for the poloidal and toroidal components are written by 
%
\begin{eqnarray}
<A_{Sn}^{\ 2} > & = &  \frac{4\pi}{V} \int <A_{Sn}^{\ 2}(r,n) >  r^{2} dr,
\nonumber \\
<A_{Tn}^{\ 2} > & = &  \frac{4\pi}{V} \int <A_{Tn}^{\ 2}(r,n) >  r^{2} dr,
\end{eqnarray}
and
\begin{eqnarray}
<A_{n}^{\ 2} > & = &  \frac{4\pi}{V} \int <A_{n}^{\ 2}(r,n) >  r^{2} dr,
\label{eq:vol_vec_msq_lm}
\end{eqnarray}
%
\item{\tt [vol\_pwr\_prefix]\_m0.dat[.gz]} Volume average of mean square amplitude of the axisymmetric components is stored. The mean square of the axisymmetric scalar is defined by 
%
\begin{eqnarray}
<f_{m0}^{2}> 
&=&  \frac{1}{V} \sum_{l=0}^{L_{max}} \left[ N_{l} \int \left(f_{l}^{0}(r) \right)^2 r^{2} dr \right].
\end{eqnarray}
%
For vector field, spectrum for the poloidal and toroidal components are written by 
\begin{eqnarray}
<A_{Sm0}^{\ 2} >\ &=& \frac{1}{V} \sum_{l=0}^{L_{max}} N_{l}
\int \left[  \left(\frac{l\left(l+1 \right) }{r^{2}} A_{Sl}^{\ 0}(r) - \frac{\partial \phi_{l}^{0}}{\partial r} \right)^{2} 
\right. \nonumber \\
& & \left. +  \frac{l \left( l+1 \right)}{r^{2}} \left( \frac{\partial A_{Sl}^{\ 0}}{\partial r} - \phi_{l}^{0} \right)^{2} \right] r^{2} dr, \\
<A_{Tm0}^{\ 2} > &=& \frac{1}{V}   \sum_{l=0}^{L_{max}} \left( l+1 \right) N_{l}  \int \left( A_{Tl}^{\ 0}(r) \right)^{2} dr.
\end{eqnarray}
%
This data file is used toe same data format as the volume mean square data in Subsection \ref{sec:mean_square}. 
%
\end{description}

Default volume averge and mean square data is integrated over the fluid domain. Calypso also can output volume integration data between manually defined inner and outer boundary. Each volume integration parameter is defined in array block \hyperref[href_t:volume_spectrum_ctl]{\tt volume\_spectrum\_ctl}. For each integration, the inner and outer boundary radii are set by \hyperref[href_t:inner_radius_ctl]{\tt inner\_radius\_ctl} and \hyperref[href_t:outer_radius_ctl]{\tt outer\_radius\_ctl}, respectively.


\subsubsection{Gauss coefficient data {\tt [gauss\_coef\_prefix].dat}}
This program output selected Gauss coefficients of the magnetic field. Gauss coefficients is evaluated for radius defined by \verb|[gauss_coef_radius]| every \verb|[increment_monitor]| steps. Gauss coefficients are evaluated by using poloidal magnetic field at CMB $B_{Sl}^{\ m}(r_{o})$ and radius defined by \verb|[gauss_coef_radius]| $r_{e}$ as
%
\begin{eqnarray}
g_{l}^{m} &=& \frac{l}{r_{e}^2} \left(\frac{r_{o}}{r_{e}}\right)^{l} B_{Sl}^{\ m}(r_{o}),
\nonumber \\
h_{l}^{m} &=& \frac{l}{r_{e}^2} \left(\frac{r_{o}}{r_{e}}\right)^{l} B_{Sl}^{\ -m}(r_{o}).
\nonumber
\end{eqnarray}
%
The header format is the same as the volume mean square data, but each item is used for as 
%
\begin{description}
\item{\tt  line 2: } Number of radial grid and truncation level
\item{\tt  line 4: } Radial layer ID for ICB and CMB
\item{\tt  line 6: } Not used
\item{\tt  line 8: } Radius for the reference radius $r_{e}$
\item{\tt  line 10: } Nunmber of Gauss coeffifients
\item{\tt  line 11: } Number of components (All number are 1)
\end{description}
%
%
The data consists of time step, time, and Gauss coefficients for each step in one line. If the Gauss coefficients data file exist before starting the simulation, programs append Gauss coefficients at the end of files without checking constancy of the number of data and order of the field. If you change the configuration of data output structure, please move the old Gauss coefficients file to another directory before starting the programs.

\subsubsection{Spectrum monitor data \\
{\tt [picked\_sph\_prefix]\_l[degree]\_m[order].dat}}
\label{sec:pickup_spectr_ctl}
This outputs of spherical harmonics coefficients at specified spherical harmonics modes and radial points in text files. Data for each spherical harmonic mode are saved in the files named {\tt [picked\_sph\_prefix]\_l[degree]\_m[order].dat}. Consequently, the file prefix {\tt [picked\_sph\_prefix]} is recommended to be defined including subdirectory name to save under a subdirectory. Spectrum data marked \verb|[Monitor_On]| are written in one line for each radial point every \verb|[increment_monitor]| steps. If the spectrum monitor data file exist before starting the simulation, programs append spectrum data at the end of files without checking constancy of the number of data and order of the field. If you change the configuration of data output structure, please move the old spectrum monitor file to another directory before starting the programs.

If a vector field $\bvec{F}$ is not a solenoidal field, $\bvec{F}$ is described by the spherical harmonics coefficients of the poloidal $F_{Sl}^{\ m}$, toroidal $F_{Tl}^{\ m}$, and potential $\varphi_{l}^{m}$ components as
\begin{eqnarray}
\bvec{F}(r, \theta, \phi) & = &  - \frac{1}{r^{2}}\frac{\partial \varphi_{0}^{0}}{\partial r} \hat{r}
 + \sum_{l=1}^{L} \sum_{m=-l}^{l} 
\left[\nabla \times \nabla \times \left( F_{Sl}^{\ m} \hat{r} \right) +  \nabla \times \left(F_{Tl}^{\ m}\right)
 - \nabla \left(\varphi_{l}^{m} Y_{l}^{m} \right)\right].
\nonumber
\end{eqnarray}
In Calypso, the following coefficients are written for the non-solenoidal vector.
\begin{description}
\item{\tt  $\verb|[field_name]_pol|$ : }
 $\left\{\begin{array}{ccr}
\displaystyle{
F_{Sl}^{\ m} - \frac{r^{2}}{l \left(l+1\right)} \frac{\partial \varphi_{l}^{m}}{\partial r} }
& \mbox{for} & \left (l \ne 0 \right)\\
\displaystyle{
 -r^{2} \frac{\partial \varphi_{0}^{0}}{\partial r}
} & \mbox{for} & \left (l = 0 \right)
\end{array}
\right.$
\item{\tt  $\verb|[field_name]_dpdr|$ : } 
$
\left\{
\begin{array}{ccr}
\displaystyle{
\frac{\partial F_{Sl}^{\ m}}{\partial r} - \varphi_{l}^{m}}
 & \mbox{for} & \left (l \ne 0 \right)\\
 0 & \mbox{for} & \left (l = 0 \right)
\end{array}
\right. $
\item{\tt  $\verb|[field_name]_tor|$ : }  $F_{Tl}^{\ m}$
\end{description}


\subsubsection{Nusselt number data {\tt [nusselt\_number\_prefix].dat}}
{\bf CAUTION: Nusselt number is not evaluated if heat source is exsist.}
The Nusselt number Nu at CMB and ICB is written for each step in one line. The Nusselt number is evaluated by
%
\begin{eqnarray*}
Nu = \frac{<\partial T / \partial r>}{\partial T_{diff}/ \partial r},
\end{eqnarray*}
where, $<\partial T / \partial r>$ and $T_{diff}$ are the horizontal average of the temperature gradient at ICB and CMB and diffusive temperature profile, respectively. $T_{diff}$ is evaluated without heat source, as
\begin{eqnarray*}
T_{diff} = \frac{r_{o}T_{o} - r_{i}T_{i}}{r_{o} - r_{i}}
    +  \frac{r_{o}r_{i}\left(T_{i} - T_{o}\right)}{r_{o} - r_{i}} \frac{1}{r}.
\end{eqnarray*}
%
This diffusive temperature profile is for the case without heat source in the fluid. If simulation is performed including the heat source, this data file does not written.

The header format is the same as the volume mean square data, but number of components are 2 for the Nusselt number for the inner and outer boundaries.

\subsubsection{Dipolrity data {\tt [dipolarity\_file\_prefix].dat}}
The dipolarity $f_{\rm dip}$ is evaluated the poloidal magnetic field at CMB is written for each step in one line. The dipolarity represents the relative strength of the axial dipole magnetic field, which is defined by the ration of the magnetic energy of the dipole component to the total magnetic energy at the CMB as
%
\begin{equation}
f_{\rm dip} = 
\left(
\frac{E_{{\rm B}\ 1}^{\ 0} (r=r_o)}
     {\sum_{l=1}^{L_{\rm dip}}
      \sum_{m=-l}^l E_{{\rm B}\ l}^{\ m} (r=r_o)}
\right)^{1/2}.
\label{eq:f_dip}
\end{equation}
%
The magnetic energy at the CMB, $E_{\rm B} (r=r_o)$, is calculated as
%
\begin{equation}
E_{{\rm B}\ l}^{\ m} (r=r_o) = 
  \frac{1}{2S_o} \int_{S_o} \left[
  \left(\bvec{B}_{l}^{m} \right)^2 + \left(\bvec{B}_{l}^{-m} \right)^2 \right] dS,
\end{equation}
%
where $S_o = 4\pi r_o^2$ is the surface area of the outer core.

\subsubsection{Typical length scale data {\tt [typical\_scale\_prefix].dat}}
The typical length scale is evaluated from kinetic and magnetic energy spectra as a wave number for the spherical harmonic degree $l$, order $m$, and difference between harmonic degree and order $n = l-m$. Each value is evaluated by
%
\begin{eqnarray}
k_{l} & = & \frac{\sum_{l=1}^{L_{max}} l <\bvec{A}_{l}^{\ 2} >}{<\bvec{A}^{2} >},
\\
k_{m} & = & \frac{\sum_{m=0}^{L_{max}} m <\bvec{A}_{m}^{\ 2} >}{<\bvec{A}^{2} >}, 
\\
k_{l-m} & = & \frac{\sum_{n=0}^{L_{max}} (l-m) <\bvec{A}_{n}^{\ 2} >}{<\bvec{A}^{2} >},
\end{eqnarray}
%
where $\bvec{A}$ is the velocity $\bvec{u}$ or magnetic field $\bvec{B}$, and volume mean squares $<\bvec{A}^{2}>$, $<\bvec{A}_{l}^{\ 2}>$, $<\bvec{A}_{m}^{\ 2}>$, and $<\bvec{A}_{n}^{\ 2}>$, are defined by equations (\ref{eq:vol_vec_msq}), (\ref{eq:vol_vec_msq_l}), (\ref{eq:vol_vec_msq_m}) , and (\ref{eq:vol_vec_msq_lm}), respectively.

The header format is the same as the volume mean square data, and the field name is defined by {\tt truncation\_}$L_{dip}$. The labels of each length scales are in Table \ref{table:scale_name}.
%
\begin{table}[htp]
\caption{Data names for typical length scles.}
\begin{center}
\begin{tabular}{|c|c|c|}
\hline
 & \mbox{Velocity} & \mbox{Magnetic field} \\ \hline
$k_{l}$ & {\tt lscale\_flow\_degree} & {\tt lscale\_magnetic\_degree} \\
$k_{m}$ & {\tt lscale\_flow\_order} & {\tt lscale\_magnetic\_order} \\
$k_{l-m}$ & {\tt lscale\_flow\_diff\_lm} & {\tt lscale\_magnetic\_diff\_lm} 
\\ \hline 
\end{tabular}
\end{center}
\label{table:scale_name}
\end{table}%
%

\subsubsection{Data output for dynamo benchmark}
\label{section:check_bench}
This program is only used to check solution for dynamo benchmark by Christensen {\it et. al}. The following files are used for this program.

\begin{table}[htp]
\caption{List of files for dynamo benchmark check {\tt sph\_dynamobench} }
\begin{center} 
\begin{tabular}{|c|c|c|}
\hline
 name & Parallelization & I/O \\ \hline \hline
\verb|control_snapshot| & Serial & Input \\ \hline
\verb|[sph_prefix].[rj_extension]|  & - & Input \\
\verb|[sph_prefix].[rlm_extension]| & - & Input \\
\verb|[sph_prefix].[rtm_extension]| & - & Input \\
\verb|[sph_prefix].[rtp_extension]| & - & Input \\ \hline
\verb|[rst_prefix].[step#].[rst_extension]| &  - & Input  \\ \hline
\verb|dynamobench.dat| & Single & Output \\ \hline
\end{tabular}
\end{center}
\label{table:sph_dynamobench}
\end{table}

\paragraph{Dynamo benchmark data {\tt dynamobench.dat}}
 In benchmark test by Christensen {\it et. al.}, both global values and local values are checked. As global results, Kinetic energy 
 $\displaystyle{ \frac{1}{V} \int \frac{1}{2} u^{2} dV}$ in the fluid shell, magnetic energy in the fluid shell 
 $\displaystyle{ \frac{1}{V} \frac{1}{E Pm} \int \frac{1}{2} B^{2} dV}$ (for case 1 and 2), and magnetic energy in the solid inner sphere 
 $\displaystyle{ \frac{1}{V_{i}} \frac{1}{E Pm} \int \frac{1}{2} B^{2} dV_{i}}$ (for case 2 only). Benchmark also requests 
 By increasing number of grid point at mid-dpeth of the fluid shell in the equatorial plane by \hyperref[href_t:nphi_mid_eq_ctl]{{\tt nphi\_mid\_eq\_ctl}}, program can find accurate solution for the point where $u_{r} = 0$ and $\partial u_{r} / \partial \phi > 0$. Angular frequency of the field pattern with respect to the $\phi$ direction is also required. The benchmark test also requires temperature and $\theta$ component of velocity. In the text file {\tt [dynamo\_benchmark\_file\_prefix].dat}, the following data are written in one line for every \verb|[i_step_rst_ctl]| step. If {\tt [dynamo\_benchmark\_file\_format]} is set to be {\tt gzip}, the data file can be commpressed by gzip format.
%
\begin{description}
\item{\tt t\_step:  }  Time step number
\item{\tt time:     }  Time
\item{\tt KE\_total: }  Total kinetic energy
\item{\tt ME\_total: }  Total magnetic energy  (Case 1 and 2)
\item{\tt ME\_total\_icore: }  Total magnetic energy in inner core (Case 2)
\item{\tt omega\_ic\_z: } Angular velocity of inner core rotation (Case 2)
\item{\tt MAG\_torque\_ic\_z: }  Magnetic torque integrated over the inner core (Case 2)
\item{\tt B\_theta: } $\Theta$ component of magnetic field at requested point.
\item{\tt v\_phi: } $\phi$ component of velocity at requested point.
\item{\tt temperature: } Temperature at requested point.
\item{\tt Average\_drift\_vr:} Average drift frequency evaluated by the position at $u_{r} = 0$ and $d u_{r} / d \phi > 0$.
\item{\tt omega\_vp44:} Drift frequency evaluated by $V_{S4}^{\ 4}$ component
\item{\tt omega\_vt54:} Drift frequency evaluated by $V_{T5}^{\ 4}$ component


\end{description}

{\small 
\begin{verbatim}
radial_layers, truncation
              64              47
ICB_id, CMB_id
               1              64
Not_used, Not_used
               1   5.384615384615384E-001
Upper_boundary_ID, Upper_boundary_radius
              64   1.538461538461538E+000
Number_of_field, Number_of_components
               8               8
    1    1    1    1    1    1    1    1
t_step    time    KE_total    ME_total    v_phi    B_theta    temperature 
   Average_drift_vr    omega_vp44    omega_vt54    
          200000   9.999999999983364E+000   3.113789741746705E+001   3.177
285062627831E+000  -7.640579505325297E+000  -4.978517354899571E+000   3.73
2175546577078E-001  -3.143091532770634E+000  -3.145607506083934E+000  -3.1
45609151616705E+000
 ...
\end{verbatim}
}
%
\subsubsection{Data output for Field along with a circle}
Calypso can output field and Fourier coefficients along with a circle as a function of longiture $\phi$. The circle is defined by \hyperref[href_t:pick_cylindrical_radius_ctl]{{\tt pick\_cylindrical\_radius\_ctl}} \\
and \hyperref[href_t:pick_vertical_position_ctl]{{\tt pick\_vertical\_position\_ctl}}, respectively. The coordinate system of the field can be chosen from {\tt spherical} or {\tt cylindrical} coordinate by \hyperref[href_t:pick_circle_coord_ctl]{{\tt pick\_circle\_coord\_ctl}}.
When \hyperref[href_t:field_on_circle_prefix]{{\tt field\_on\_circle\_prefix}} is defined, the field data $f(\phi)$ is written in \\
{\tt field\_on\_circle\_prefix.dat[.gz]}, and the data format is the same as the layerd mean square data.
The spectrum data $f(m)$ is written in {\tt spectr\_on\_circle\_prefix.dat[.gz]} \\
if \hyperref[href_t:spectr_on_circle_prefix]{{\tt spectr\_on\_circle\_prefix}} is defined The data format is the same as the volume spectra monitor data.
When \hyperref[href_t:field_on_circle_format]{{\tt field\_on\_circle\_format}}, is set to be {\tt gzip}, data file can be compressed by gzip format.


\newpage

\section{Utility programs}
Calypso includes some utility programs. These programs are useful to data analysis and debugging the simulation programs.

\subsection{Data transform program ({\tt sph\_snapshot})}
\label{section:sph_snapshot}
%
\begin{figure}[htbp]
\begin{center}
\includegraphics*[width=130mm]{Images/flow_3}
\end{center}
\caption{Data flow for data transform program.}
\label{fig:flow_3}
\end{figure}
%
Simulation program outputs spectrum data as a whole field data. This program is made from simulation program by replacing from time integration routines to restart data input routine. Consequently, Input/Output files in Table \ref{table:sph_mhd} are the same for {\tt sph\_snapshot}, except for the required input restart data \verb|[rst_prefix].[step #].[rst_extension]|.
%
This program requires control file \verb|control_snapshot| insteacd of \verb|control_mhd|. File format of the control file is same as the control field for simulation \hyperref[href_i:MHD_control]{\tt control\_MHD}.

The same files as the simulation program are read in this program, and field data are generated from the snapshots of spectrum data. The monitoring data for snapshots can also be generated. \verb|[step #]| is added in the file name, and the \verb|[step #]| is calculated by \verb|time step|/\verb|[ISTEP_FIELD]|.

We recommend to output cross section data at $y = 0$ by using sectioning module (see \ref{section:PSF}) for zonal mean snapshot program \verb|sph_zm_snapshot| to reduce data size.

\newpage
\subsection{Initial field generation program \\
({\tt sph\_initial\_field})}
\label{sec:sph_initial_field}
%
\begin{figure}[htbp]
\begin{center}
\includegraphics*[width=130mm]{Images/flow_ini}
\end{center}
\caption{Data flow for initial field generation program.}
\label{fig:flow_ini}
\end{figure}
%
 The initial fields for dynamo benchmark can set in the simulation program by setting \verb|[INITIAL_TYPE]| flag. This program is used to generate initial field by user.  The heat source $q_{T}$ and light element source $q_{C}$ are also defined by this program because $q_{T}$ and $q_{C}$ are defined as scalar fields. Spherical harmonics indexing data files are also generated by using information in \verb|spherical_shell_ctl| block if these indexing data files do not exist. The Fortran source file to define initial field \\
 \verb|const_sph_initial_spectr.f90| is saved in \verb|src/programs/data_utilities| \\
 \verb|/INITIAL_FIELD/| directory,  and please compile again after modifying this module. This program also needs the files listed in Table \ref{table:inital_fld}.
%
\begin{table}[htp]
\caption{List of files for simulation {\tt sph\_initial\_field} }
\begin{center} 
\begin{tabular}{|c|c|c|}
\hline
 name & Parallelization & I/O \\ \hline \hline
\verb|control_MHD| & Serial & Input \\ \hline
\verb|[sph_prefix].[rj_extension]|  & - & Input/(Output) \\
\verb|[sph_prefix].[rlm_extension]| & - & Input/(Output) \\
\verb|[sph_prefix].[rtm_extension]| & - & Input/(Output) \\
\verb|[sph_prefix].[rtp_extension]| & - & Input/(Output) \\ \hline
\verb|[rst_prefix].[step #].[rst_extension]| & - & Input/Output  \\ \hline
\end{tabular}
\end{center}
(Output): Marked files are generated if files do not exist.
\label{table:inital_fld}
\end{table}
%
This program generates the spectrum data files \verb|[rst_prefix].[step #].[rst_extension]|. To use generated initial data file, please set 
\hyperref[href_t:i_step_init_ctl]{{\tt [ISTEP\_START]}} to be 0 and \hyperref[href_t:restart_file_ctl]{{\tt [INITIAL\_TYPE]}} to be \\
\hyperref[href_t:restart_file_ctl]{{\tt start\_from\_rst\_file}}.

\subsubsection{Definition of the initial field}
\label{sec:def_initial}
To construct Initial field data, you need to edit the source code \verb|const_sph_initial_spectr.f90| in \verb|src/programs/data_utilities/INITIAL_FIELD/| directory. The module \verb|const_sph_initial_spectr| consists of the following subroutines:
%
\begin{description}
\item{\verb|sph_initial_spectrum|:}    Top subroutine to construct initial field.
\item{\verb|set_initial_velocity|:}        Routine to construct initial velocity.
\item{\verb|set_initial_temperature|:} Routine to construct initial temperature.
\item{\verb|set_initial_composition|:} Routine to construct initial composition.
\item{\verb|set_initial_magne_sph|:} Routine to construct initial magnetic field.
\item{\verb|set_initial_heat_source_sph|:} Routine to construct heat source.
\item{\verb|set_initial_light_source_sph|:}  Routine to construct composition source.
\end{description}
%
The construction routine for each field are called from the top routine \\
\verb|const_sph_initial_spectr.f90|. If lines to call subroutines are commented out, corresponding initial fields are set to 0. In addition, the initial fields to be constructed need to be defined by \verb|nod_value_ctl| array in the \verb|control_MHD|.
%
\begin{table}[htp]
\caption{Field name and corresponding field id in Calypso}
\begin{center}
\begin{tabular}{|c|c|cc|}
\hline
field name & scalar & poloidal  & toroidal  \\ \hline
Velocity & - & \verb|ipol%i_velo| &   \verb|itor%i_velo| \\ 
Magnetic field & - & \verb|ipol%i_magne| &  \verb|itor%i_magne| \\ 
Current density & - & \verb|ipol%i_current| &  \verb|itor%i_current| \\ 
Temperature & \verb|ipol%i_temp| & - & - \\ 
Composition & \verb|ipol%i_light| & - & - \\ 
Heat source & \verb|ipol%i_heat_source| & - & - \\ 
Composition source & \verb|ipol%i_light_source| & - & - \\ \hline
\end{tabular}
\end{center}
\label{table:field_point}
\end{table}%
% 

Initial fields need to be defined by the spherical harmonics coefficients at each radial points as array \verb|d_rj(i,i_field)|, where \verb|i| and \verb|i_field| are the local address of the spectrum data and field id, respectively. The address of the fields are listed in Table \ref{table:field_point}.

In Calypso, local data address for each MPI process is used for the spectrum data address \verb|i|. To find the local address \verb|i|, two functions are required. \\
First, \verb|j = find_local_sph_mode_address(l,m)| returns the local spherical  harmonics address \verb|j| from aa spherical harmonics mode $Y_{l}^{m}$. If process does not have the data for $Y_{l}^{m}$, \verb|j| is set to 0. Second, \verb|i = local_sph_data_address(k,j)| returns the local data address \verb|i| from radial grid number \verb|k| and local spherical harmonics id \verb|j|. For do loops in the radial direction, the total number of radial grid points, radial address for ICB, and radial address for CMB are defined as \verb|nidx_rj(1)|, \verb|nlayer_ICB|, and \verb|nlayer_CMB|, respectively. The radius for the \verb|k|-th grid points can be obtained by \verb|r = radius_1d_rj_r(k)|. The subroutines to define initial temperature for the dynamo benchmark Case 1 is shown below as an example.

After updating the source code, the program \verb|sph_initial_field| needs to be updated. To update the program, move to the work directory \verb|[CALYPSO_HOME]/work| and run make command as
% 
\begin{verbatim}
% cd \verb|[CALYPSO_HOME]/work|
% make
\end{verbatim}
%
Then, the program  \verb|sph_initial_field| and  \verb|sph_add_initial_field| are updated.

%
{\small
\begin{verbatim}
!
      subroutine set_initial_temperature
!
      use m_sph_spectr_data
!
      integer ( kind = kint) :: inod, k, jj
      real (kind = kreal) :: pi, rr, xr, shell
      real(kind = kreal), parameter :: A_temp = 0.1d0
!
!
!$omp parallel do
      do inod = 1, nnod_rj
        d_rj(inod,ipol%i_temp) = zero
      end do
!$omp end parallel do
!
      pi = four * atan(one)
      shell = r_CMB - r_ICB
!
!   search address for (l = m = 0)
      jj = find_local_sph_mode_address(0, 0)
!
!   set reference temperature if (l = m = 0) mode is there
      if (jj .gt. 0) then
        do k = 1, nlayer_ICB-1
          inod = local_sph_data_address(k,jj)
          d_rj(inod,ipol%i_temp) = 1.0d0
        end do
        do k = nlayer_ICB, nlayer_CMB
          inod = local_sph_data_address(k,jj)
          d_rj(inod,ipol%i_temp) = (ar_1d_rj(k,1) * 20.d0/13.0d0        &
     &                              - 1.0d0 ) * 7.0d0 / 13.0d0
        end do
      end if
!
!
!    Find local addrtess for (l,m) = (4,4)
      jj =  find_local_sph_mode_address(4, 4)
!      jj =  find_local_sph_mode_address(5, 5)
!
!    If data for (l,m) = (4,4) is there, set initial temperature
      if (jj .gt. 0) then
!    Set initial field from ICB to CMB
        do k = nlayer_ICB, nlayer_CMB
!
!    Set radius data
          rr = radius_1d_rj_r(k)
!    Set 1d address to substitute at (Nr, j)
          inod = local_sph_data_address(k,jj)
!
!    set initial temperature
          xr = two * rr - one * (r_CMB+r_ICB) / shell
          d_rj(inod,ipol%i_temp) = (one-three*xr**2+three*xr**4-xr**6)  &
     &                            * A_temp * three / (sqrt(two*pi))
        end do
      end if
!
!    Center
      if(inod_rj_center .gt. 0) then
        jj = find_local_sph_mode_address(0, 0)
        inod = local_sph_data_address(1,jj)
        d_rj(inod_rj_center,ipol%i_temp) = d_rj(inod,ipol%i_temp)
      end if
!
      end subroutine set_initial_temperature
!
\end{verbatim}
}
%

\subsection{Initial field modification program \\
({\tt sph\_add\_initial\_field})}
\label{sec:add_initial_field}
%
\begin{figure}[htbp]
\begin{center}
\includegraphics*[width=130mm]{Images/flow_ini}
\end{center}
\caption{Data flow for initial field modification program.}
\label{fig:flow_add_ini}
\end{figure}
%
{\bf Caution: This program overwrites existing initial field data. Please run it after taking a backup.} \\

 This program modifies or adds new data to an initial field file. It could be used to start a new geodynamo simulation by adding seed magnetic field or source terms to a non-magnetic convection simulation. The initial fields to be added are also defined in \verb|const_sph_initial_spectr.f90|. \verb|data_utilities/INITIAL_FIELD/| directory. This program also needs the files listed in Table \ref{table:add_inital_fld}.
%
\begin{table}[htp]
\caption{List of files for simulation {\tt sph\_add\_initial\_field} }
\begin{center} 
\begin{tabular}{|c|c|c|}
\hline
 name & Parallelization & I/O \\ \hline \hline
\verb|control_MHD| & Serial & Input \\ \hline
\verb|[sph_prefix].[rj_extension]|  & - & Input / (Output) \\
\verb|[sph_prefix].[rlm_extension]| & - & Input / (Output)  \\
\verb|[sph_prefix].[rtm_extension]| & - & Input / (Output)  \\
\verb|[sph_prefix].[rtp_extension]| & - & Input / (Output)  \\ \hline
\verb|[rst_prefix].[step #].[rst_extension]| & - & Input/Output  \\ \hline
\end{tabular}
\end{center}
(Output): Marked files are generated if files do not exist.
\label{table:add_inital_fld}
\end{table}
%
This program generates the spectrum data files \verb|[rst_prefix].[step#].[rst_extension]|. To use generated initial data file, set 
 \hyperref[href_t:i_step_init_ctl]{{\tt [ISTEP\_START]}} and \verb|[ISTEP_RESTART]| to be appropriate time step and increment, respectively.
To read the original initial field data, \hyperref[href_t:restart_file_ctl]{{\tt [INITIAL\_TYPE]}} is set to be \hyperref[href_t:restart_file_ctl]{{\tt start\_from\_rst\_file}} in \verb|control_MHD|. In other words, the \verb|[step #]| in the file name, \hyperref[href_t:i_step_init_ctl]{{\tt [ISTEP\_START]}}, and \verb|[ISTEP_RESTART]| in the control file should be the consistent.

This program also uses the module file \verb|const_sph_initial_spectr.f90| to define the initial field. The initial fields are defined as following the previous section \ref{sec:def_initial}. After updating the source code, the program \verb|sph_initial_field| needs to be updated. After modifying  \verb|const_sph_initial_spectr.f90|, the program is build by make command in  the work directory \verb|[CALYPSO_HOME]/work|.

\subsection{Sectioning program ({\tt sectioning})} \label{sec:sectioning}
This program generates cross sections and isosurfaces from FEM mesh data and field data using the sectioning and isosurface module in the simulation program {\tt sph\_mhd}. The data for this program is listed in Table \ref{table:sectioning}. This program run on the parallel environment, and needs to use the same number of MPI processes as the number of processes which is used for the simulation program. VTK and compressed VTK data is not supported for the input field data.
%
\begin{figure}[htbp]
\begin{center}
\includegraphics*[width=130mm]{Images/flow_5}
\end{center}
\caption{Data flow for sectioning program.}
\label{fig:flow_add_ini}
\end{figure}
%
\begin{table}[htp]
\caption{List of files for sectioning {\tt sectioning} }
\begin{center} 
\begin{tabular}{|c|c|c|}
\hline
name & Parallelization & I/O \\ \hline \hline
\verb|control_viz| & Serial & Input \\ \hline
\verb|[mesh_prefix].[fem_extension]| & - & Input \\ \hline
\verb|[fld_prefix].[step#].[domain#].[extension]| & - & Input  \\ \hline
\verb|[section_prefix].[step#].[extension]| &  Single & Output  \\
\verb|[isosurface_prefix].[step#].[extension]| &  Single & Output  \\ \hline
\end{tabular}
\end{center}
\label{table:sectioning}
\end{table}
%

\subsubsection{Control file}
The format of the control file \verb|control_viz| is described below. The detail of each block is described in section \ref{section:def_control}. You can jump to detailed description by clicking each item". \\
\\
%
Block \verb|visualizer|  (Top block of the control file)
\label{href_i:visualizer}
%
\begin{itemize}
\item Block \hyperref[href_t:data_files_def]{\tt data\_files\_def}
	\label{href_i:data_files_def_v}
%
	\begin{itemize}
	\item \hyperref[href_t:num_subdomain_ctl]
			{\tt num\_subdomain\_ctl    [Num\_PE]}
	\item \hyperref[href_t:num_smp_ctl]
			{\tt num\_smp\_ctl    [Num\_Threads]}
	\item \hyperref[href_t:mesh_file_prefix]
			{\tt mesh\_file\_prefix    [mesh\_prefix]}
	\item \hyperref[href_t:field_file_prefix]
			{\tt field\_file\_prefix    [fld\_prefix]}
%
	\item \hyperref[href_t:mesh_file_fmt_ctl]
			{\tt mesh\_file\_fmt\_ctl    [mesh\_format]}
	\item \hyperref[href_t:field_file_fmt_ctl]
			{\tt field\_file\_fmt\_ctl    [fld\_format]}
	\end{itemize}
%
\item Block \hyperref[href_t:time_step_ctl]{\tt time\_step\_ctl}
	\begin{itemize} \label{href_i:time_step_ctl_v}
	\item \hyperref[href_t:i_step_init_ctl]
		{\tt i\_step\_init\_ctl        [ISTEP\_START]}
	\item \hyperref[href_t:i_step_finish_ctl]
		{\tt i\_step\_finish\_ctl      [ISTEP\_FINISH]}
	\item \hyperref[href_t:i_step_field_ctl]
		{\tt i\_step\_field\_ctl       [ISTEP\_FIELD]}
	\item \hyperref[href_t:i_step_sectioning_ctl]
		{\tt i\_step\_sectioning\_ctl  [ISTEP\_SECTION]}
	\item \hyperref[href_t:i_step_isosurface_ctl]
		{\tt i\_step\_isosurface\_ctl  [ISTEP\_ISOSURFACE]}
	\end{itemize}
%
\item Block \hyperref[href_t:visual_control]{\tt visual\_control}
    \begin{itemize} \label{href_i:visual_control_v}
    \item \hyperref[href_t:i_step_sectioning_ctl]
        {\tt i\_step\_sectioning\_ctl  [ISTEP\_SECTION]}
    \item Array \hyperref[href_t:cross_section_ctl]{\tt cross\_section\_ctl}
		\begin{itemize}
        \item File or Block {\tt cross\_section\_ctl} \\
                            {\tt [section\_control\_file]} \\
								(See section \ref{section:section_control})
		\end{itemize}
%
    \item \hyperref[href_t:i_step_isosurface_ctl]
		{\tt i\_step\_isosurface\_ctl  [ISTEP\_ISOSURFACE]}
    \item Array \hyperref[href_t:isosurface_ctl]{\tt isosurface\_ctl}
		\begin{itemize}
		\item File or Block {\tt isosurface\_ctl} \\
                            {\tt [isosurface\_control\_file]} \\
								(See section \ref{section:isosurface_control})
		\end{itemize}
    \end{itemize}
\end{itemize}
%

\subsection{Field data converter program ({\tt field\_to\_VTK})}
\label{sec:field_to_VTK}
This program generates VTK data from FEM mesh data and field data. The data for this program is listed in Table \ref{table:fld_to_vtk}. This program run on the parallel environment, and needs to use the same number of MPI processes as the number of processes which is used for the simulation program.
%
\begin{table}[htp]
\caption{List of files for sectioning {\tt sectioning} }
\begin{center} 
\begin{tabular}{|c|c|c|}
\hline
name & Parallelization & I/O \\ \hline \hline
\verb|control_viz| & Serial & Input \\ \hline
\verb|[mesh_prefix].[fem_extension]| & - & Input \\ \hline
\verb|[fld_prefix].[step#].[domain#].[extension]| & - & Input  \\ \hline
\verb|[fld_prefix].[step#].[domain#].[vtk]| or [vtk.gz] & - & Output  \\ \hline
\end{tabular}
\end{center}
\label{table:sectioning}
\end{table}
%

\newpage
\subsubsection{Control file}
The format of the control file \verb|control_viz| is described below. The detail of each block is described in section \ref{section:def_control}. You can jump to detailed description by clicking each item". \\
\\
%
Block \verb|visualizer|  (Top block of the control file)
\label{href_i:visualizer}
%
\begin{itemize}
\item Block \hyperref[href_t:data_files_def]{\tt data\_files\_def}
	\label{href_i:data_files_def_f}
%
	\begin{itemize}
	\item \hyperref[href_t:num_subdomain_ctl]
			{\tt num\_subdomain\_ctl    [Num\_PE]}
	\item \hyperref[href_t:num_smp_ctl]
			{\tt num\_smp\_ctl    [Num\_Threads]}
	\item \hyperref[href_t:mesh_file_prefix]
			{\tt mesh\_file\_prefix    [mesh\_prefix]}
	\item \hyperref[href_t:field_file_prefix]
			{\tt field\_file\_prefix    [fld\_prefix]}
%
	\item \hyperref[href_t:mesh_file_fmt_ctl]
			{\tt mesh\_file\_fmt\_ctl    [mesh\_format]}
	\item \hyperref[href_t:field_file_fmt_ctl]
			{\tt field\_file\_fmt\_ctl    [fld\_format]}
	\end{itemize}
%
\item Block \hyperref[href_t:time_step_ctl]{\tt time\_step\_ctl}
	\begin{itemize} \label{href_i:time_step_ctl_f}
	\item \hyperref[href_t:i_step_init_ctl]
		{\tt i\_step\_init\_ctl        [ISTEP\_START]}
	\item \hyperref[href_t:i_step_finish_ctl]
		{\tt i\_step\_finish\_ctl      [ISTEP\_FINISH]}
	\item \hyperref[href_t:i_step_field_ctl]
		{\tt i\_step\_field\_ctl       [ISTEP\_FIELD]}
	\end{itemize}
%
\item Block \hyperref[href_t:visual_control]{\tt visual\_control}
    \begin{itemize} \label{href_i:visual_control_v}
    \item \hyperref[href_t:output_field_file_fmt_ctl]
		{\tt output\_field\_file\_fmt\_ctl  [VTK\_format]}
    \end{itemize}
\end{itemize}
%

\subsection{Section and isosurface data converter program ({\tt section\_to\_VTK})} 
\label{section:section_to_VTK}
This program generates VTK data from bindary sectioning and isosurface data. This program run on a single processor, and needs interactive input. The following is the console output of the program.

{\small
\begin{verbatim}
% /usr/local/Calypso/bin/section_to_VTK
Input file prefix
zm_y0									<- Input file prefix
Input file extension from following:
vtk, vtk.gz, vtd, vtd.gz, inp, inp.gz, udt, udt.gz, psf, psf.gz, sdt, sdt.gz  
sdt.gz									<- Input extension
ifmt_input          23
Input start, end, and increment of file step
{\color{red} 2004 2000 1								<- Input start, end, and increment of file step} 
Write ascii VTK file: zm_y0.2000.vtk
\end{verbatim}
}

\subsection{Restart data assemble program ({\tt assemble\_sph})}
\label{section:assemble_sph}
%
\begin{figure}[htbp]
\begin{center}
\includegraphics*[width=130mm]{Images/flow_4}
\end{center}
\caption{Data flow for spectrum data assemble program}
\label{fig:flow_4}
\end{figure}
%
Calypso uses distributed data files for simulations. This program is to generate new spectrum data for restarting with different spatial resolution or parallel configuration. This program organizes new spectral data by using specter indexing data using different domain decomposition. The following files used for data IO. If radial resolution is changed from the original data, the program makes new spectrum data by linear interpolation. If new data have smaller or larger truncation degree, the program fills zero to the new spectrum data or truncates the data to fit the new spatial resolution, respectively. This program can perform with any number of MPI processes, but we recommend to run the program with {\bf one} process or the same number of processes as the number of subdomains for the target configuration which is defined by \verb|num_new_domain_ctl|. Data files for the program are shown In Table \ref{table:assemble_newsph}. The time and number of time step can also be changed by this program. The new time and time step are defined by the parameters in \verb|new_time_step_ctl| block. The step number of the restart data will be \verb|i_step_init_ctl| / \verb|i_step_rst_ctl| in  \verb|new_time_step_ctl|. If \verb|new_time_step_ctl| block is not defined, time and time step informations are carried from the original restart data.

%
\begin{table}[htp]
\caption{List of files for {\tt assemble\_sph} }
\begin{center} 
\begin{tabular}{|c|c|c|}
\hline
 extension & Distributed? & I/O \\ \hline
\verb|control_assemble_sph| & Serial & Input \\ \hline
\verb|[sph_prefix].[rj_extension]|  & - & Input \\  \hline
\verb|[new_sph_prefix].[domain#].rj| &  Distributed & Input \\ \hline
\verb|[rst_prefix].[step#].[rst_extension]| & - & Input  \\
\verb|[new_rst_prefix].[step#].[domain#].fst| &  Distributed & Output \\ \hline
\end{tabular}
\end{center}
\label{table:assemble_newsph}
\end{table}
%

\subsubsection{Format of control file}
Control file consists the following groups. \\
%
Block \verb|assemble_control| \label{href_i:assemble_control} (Top lebel of the block)
\begin{itemize}
\item Block \verb|data_files_def|
	\hyperref[href_t:data_files_def]{(Detail)}
	\begin{itemize}
	\item \hyperref[href_t:num_subdomain_ctl]
			{\tt num\_subdomain\_ctl    [Num\_PE]}
	\item \hyperref[href_t:sph_file_prefix]
			{\tt sph\_file\_prefix      [sph\_prefix]}
	\item \hyperref[href_t:restart_file_prefix]
            {\tt restart\_file\_prefix  [rst\_prefix])}
%
	\item \hyperref[href_t:sph_file_fmt_ctl]
			{\tt sph\_file\_fmt\_ctl    [sph\_format]}
	\item \hyperref[href_t:restart_file_fmt_ctl]
			{\tt restart\_file\_fmt\_ctl    [rst\_format]}
	\end{itemize}
%
\item Block \verb|new_data_files_def|
	\label{href_i:new_data_files_def}
	\hyperref[href_t:new_data_files_def]{(Detail)}
	\begin{itemize}
	\item \hyperref[href_t:num_subdomain_ctl]
			{\tt num\_subdomain\_ctl    [Num\_PE]}
	\item \hyperref[href_t:sph_file_prefix]
			{\tt sph\_file\_prefix      [sph\_prefix]}
	\item \hyperref[href_t:restart_file_prefix]
            {\tt restart\_file\_prefix  [rst\_prefix])}
%
	\item \hyperref[href_t:sph_file_fmt_ctl]
			{\tt sph\_file\_fmt\_ctl    [sph\_format]}
	\item \hyperref[href_t:restart_file_fmt_ctl]
			{\tt restart\_file\_fmt\_ctl    [rst\_format]}
%
	\item \hyperref[href_t:delete_original_data_flag]
			{\tt delete\_original\_data\_flag    [YES or NO]}
	\end{itemize}
%
\item Block \verb|control|
	\begin{itemize}
	\item Block \hyperref[href_t:time_step_ctl]{\tt time\_step\_ctl}
		\begin{itemize} \label{href_i:time_step_ctl2}
		\item \hyperref[href_t:i_step_init_ctl]
			{\tt i\_step\_init\_ctl        [ISTEP\_START]}
		\item \hyperref[href_t:i_step_finish_ctl]
			{\tt i\_step\_finish\_ctl      [ISTEP\_FINISH]}
		\item \hyperref[href_t:i_step_rst_ctl]
			{\tt i\_step\_rst\_ctl         [ISTEP\_RESTART]}
		\end{itemize}
%
	\item  Block \hyperref[href_t:i_step_init_ctl]{\tt new\_time\_step\_ctl}
		\begin{itemize} \label{href_i:new_time_step_ctl}
		\item \hyperref[href_t:i_step_init_ctl_a]
			{\tt i\_step\_init\_ctl        [ISTEP\_START]}
		\item \hyperref[href_t:i_step_rst_ctl_a]
			{\tt i\_step\_rst\_ctl         [ISTEP\_RESTART]}
		\item \hyperref[href_t:time_init_ctl_a]
			{\tt time\_init\_ctl           [INITIAL\_TIME]}
		\end{itemize}
	\end{itemize}
%
\item Block \hyperref[href_t:newrst_magne_ctl]{\tt newrst\_magne\_ctl}
	\begin{itemize} \label{href_i:newrst_magne_ctl}
	\item \hyperref[href_t:magnetic_field_ratio_ctl]
		{\tt magnetic\_field\_ratio\_ctl    [ratio]}
	\end{itemize}
\end{itemize}

\subsection{Time averaging program  ({\tt t\_ave\_monitor\_data})}
\label{sec:ave_mean_square}
This program generate time average and standard deviation of monitoring data defined in {\tt sph\_monitor\_ctl} block. This program read a control file {\tt control\_sph\_time\_average}. the control parameters are the following:
%
Block \verb|time_averaging_sph_monitor|  (Top block of the control file)
\label{href_i:time_averaging_sph_monitor}
%
\begin{itemize}
\item \hyperref[href_t:tave_start_time_ctl]
		{\tt start\_time\_ctl    [TIME]}
\item \hyperref[href_t:tave_end_time_ctl]
		{\tt end\_time\_ctl      [TIME]}
\item Block {\tt monitor\_data\_list\_ctl}
				\label{href_i:data_files_def}
	\begin{itemize}
	\item array \hyperref[href_t:volume_integrate_prefix]
				{\tt volume\_integrate\_prefix    [File\_Prefix]}
				\label{href_i:volume_integrate_prefix}
	\item array \hyperref[href_t:volume_sph_spectr_prefix]
				{\tt volume\_sph\_spectr\_prefix   [File\_Prefix]}
				\label{href_i:volume_sph_spectr_prefix}
	\item array \hyperref[href_t:sphere_integrate_prefix]
				{\tt sphere\_integrate\_prefix     [File\_Prefix]}
				\label{href_i:sphere_integrate_prefix}
	\item array \hyperref[href_t:layer_sph_spectr_prefix]
				{\tt layer\_sph\_spectr\_prefix    [File\_Prefix]}
				\label{href_i:layer_sph_spectr_prefix}
	\item array \hyperref[href_t:picked_sph_prefix]
				{\tt picked\_sph\_prefix           [File\_Prefix]}
				\label{href_i:picked_sph_prefix}
	\end{itemize}
%
\end{itemize}
%
There are five types of the monitor data to be averaged. The corresponding monitor data prefixes for each type are listed in Tables \ref{table:volume_integrate_prefix} to \ref{table:picked_sph_prefix}. If data is end before the end time, the program will finish at the end of file. \verb|t_ave| and \verb|t_sigma| are added at the beginning of the input file name for the time average and standard deviation data file, respectively.

%
\begin{table}[htp]
\caption{List of avaiable monitor data for {\tt [volume\_integrate\_prefix]}}
\begin{center} 
\begin{tabular}{|c|}
\hline
 name  \\ \hline \hline
\verb|[vol_pwr_prefix]_s|   \\
\verb|[vol_pwr_prefix]_m0|   \\
\verb|[vol_ave_prefix]|   \\
\verb|[gauss_coef_prefix]|     \\
\verb|[nusselt_number_prefix]|     \\
\verb|[dipolarity_file_prefix]|     \\
\verb|[dynamo_benchmark_data_ctl]|     \\ \hline
\end{tabular}
\end{center}
\label{table:volume_integrate_prefix}
\end{table}
%
\begin{table}[htp]
\caption{List of avaiable monitor data for {\tt [volume\_sph\_spectr\_prefix]}}
\begin{center} 
\begin{tabular}{|c|}
\hline
 name  \\ \hline \hline
\verb|[vol_pwr_prefix]_l|   \\
\verb|[vol_pwr_prefix]_m|   \\
\verb|[vol_pwr_prefix]_lm|    \\ \hline
\end{tabular}
\end{center}
\label{table:volume_sph_spectr_prefix}
\end{table}
%
\begin{table}[htp]
\caption{List of avaiable monitor data for {\tt [sphere\_integrate\_prefix]}}
\begin{center} 
\begin{tabular}{|c|}
\hline
 name  \\ \hline \hline
\verb|[layer_pwr_prefix]_s|   \\
\verb|[layer_pwr_prefix]_m0|   \\
\verb|[field_on_circle_prefix]|     \\ \hline
\end{tabular}
\end{center}
\label{table:sphere_integrate_prefix}
\end{table}
%
\begin{table}[htp]
\caption{List of avaiable monitor data for {\tt [layer\_sph\_spectr\_prefix]}}
\begin{center} 
\begin{tabular}{|c|}
\hline
 name  \\ \hline \hline
\verb|[layer_pwr_prefix]_l|   \\
\verb|[layer_pwr_prefix]_m|   \\
\verb|[layer_pwr_prefix]_lm|   \\
\verb|[spectr_on_circle_prefix]|     \\ \hline
\end{tabular}
\end{center}
\label{table:layer_sph_spectr_prefix}
\end{table}
%
%
\begin{table}[htp]
\caption{List of avaiable monitor data for {\tt [picked\_sph\_prefix]}}
\begin{center} 
\begin{tabular}{|c|}
\hline
 name  \\ \hline \hline
\verb|[picked_sph_prefix]| \\ \hline
\end{tabular}
\end{center}
\label{table:picked_sph_prefix}
\end{table}
%

\subsection{Module dependency program ({\tt make\_f90depends})}
\label{sec:make_f90depends}
This program is only used to generate Makefile in {\tt work} directory. Most of case, Fortran 90 modules have to compiled prior to be referred by another fortran90 routines. This program is generates dependency lists in Makefile. To use this program, the following limitation is required.
\begin{itemize}
\item One source code has to consist of one module.
\item The module name should be the same as the file name.
\end{itemize}
%

\subsection{Visualization using field data}
\label{sec:paraview}
The field data is written by XDMF or VTK data format using Cartesian coordinate. In this section we briefly introduce how to display the radial magnetic field using ParaView as an example.
%
\begin{figure}[htbp]
\begin{center}
\includegraphics*[width=130mm]{Images/paraview_open}
\caption{File open window for ParaView}
\label{fig:paraview_load}
\end{center}
\end{figure}
%

After the starting Paraview, the file to be read is chosen in the file menu, and press "apply", button. Then, Paraview load the data from files (see Figure \ref{fig:paraview_load}). 
Because the magnetic field is saved by the Cartesian coordinate, the radial magnetic field is obtained by the calculator tool. The procedure is as following (see Figure \ref{fig:paraview_gen_Br})
%
\begin{enumerate}
\item Push calculator button.
\item Choose "Point Data" in Attribute menu
\item Input data name for radial magnetic field ("B\_r" in  Figure \ref{fig:paraview_gen_Br})
\item Enter the equation to evaluate radial mantic field $B_{r} = \bvec{B} \cdot \bvec{r} / |r|$.
\item Finally, push "Apply" button.
\end{enumerate}
%
%
\begin{figure}[htbp]
\begin{center}
\includegraphics*[width=100mm]{Images/paraview_calc}
\caption{File open window for ParaView}
\label{fig:paraview_gen_Br}
\end{center}
\end{figure}
%
After obtaining the radial mantric field, the image in figure \ref{fig:paraview_br} is obtained by using "slice" and  "Contour" tools with appropriate color mapping.
%
\begin{figure}[htbp]
\begin{center}
\includegraphics*[width=100mm]{Images/Paraview_Br}
\end{center}
\caption{Visualization of radial magnetic field by Paraview.}
\label{fig:paraview_br}
\end{figure}
%

\newpage
\section{Deprecated features}
There are some deprecated features and programs from the Calypso Ver. 1.x. These programs are still helpful for development, debug, and to convert dat from Ver.1.x.

\subsection{Preprocessing program ({\tt gen\_sph\_grid})}
\label{section:gen_sph_grid}
%
From Ver. 2, the spherical harmonic indices data is not necessary if the parameters for the spherical shell described below is included in the control for the simulation. However, This program \verb|gen_sph_grid| is still useful for debug and I will describe how to set up the parallelization in this subsection.
%
\begin{figure}[htbp]
\begin{center}
\includegraphics*[width=130mm]{Images/flow_1}
\end{center}
\caption{Generated files by preprocessing program in Data flow.}
\label{fig:gen_sph_grid}
\end{figure}
%
This program generates index table and a communication table for parallel spherical harmonics, table of integrals for Coriolis term, and FEM mesh information to generate visualization data (see Figure \ref{fig:gen_sph_grid}). This program needs control file for input. This program can perform with {\bf any} number of MPI processes less than the number of subdomains, but is required to run with the SAME number of subdomains to generate MERGED data . The output files include the indexing tables. 

%
\begin{table}[htp]
\caption{List of files for {\tt gen\_sph\_grid} }
\begin{center} 
\begin{tabular}{|c|c|c|c|}
\hline
Files & extension & Parallelization & I/O \\ \hline \hline
Control file & \verb|control_sph_grid| & Single & Input \\ \hline
Index for $(r,j)$ & \verb|[sph_prefix].[rj_extension]| & - & Output \\
Index for $(r,l,m)$ & \verb|[sph_prefix].[rlm_extension]| & - & Output \\
Index for $(r,t,m)$ & \verb|[sph_prefix].[rtm_extension]| & - & Output \\
Index for $(r,t,p)$ & \verb|[sph_prefix].[rtp_extension]| & - & Output \\ \hline
FEM mesh & \verb|[sph_prefix].[fem_extension]| & - & Output \\ \hline
Radial point list & \verb|radial_info.dat| & Single & Output \\ \hline
\end{tabular}
\end{center}
\label{table:gen_sph_grid}
\end{table}
%
%
\begin{table}[htp]
\caption{data format flag {[\tt sph\_file\_fmt\_ctl]} and extensions.}
\begin{center} 
\begin{tabular}{|c||c|c|c|c|}
\hline
  \multicolumn{5}{|c|}{Distributed files} \\ \hline
  \verb|[sph_file_fmt_ctl]| &  \verb|ascii| & \verb|binary| & \verb|gzip| & \verb|bin_gz| \\ \hline
\verb|[rj_extension]|  & \verb|[#].rj|  & \verb|[#].brj| & \verb|[#].rj.gz|  & \verb|[#].brj.gz| \\
\verb|[rlm_extension]| & \verb|[#].rlm| & \verb|[#].blm| & \verb|[#].rlm.gz| & \verb|[#].blm.gz| \\
\verb|[rtm_extension]| & \verb|[#].rtm| & \verb|[#].btm| & \verb|[#].rtm.gz| & \verb|[#].btm.gz| \\
\verb|[rtp_extension]| & \verb|[#].rtp| & \verb|[#].btp| & \verb|[#].rtp.gz| & \verb|[#].btp.gz| \\ \hline
\verb|[fem_extension]| & \verb|[#].gfm| & \verb|[#].gfb| & \verb|[#].gfm.gz| & \verb|[#].gfb.gz| \\ \hline \hline
  \multicolumn{5}{|c|}{Single file}  \\ \hline
  \verb|[sph_file_fmt_ctl]| & \verb|merged| & \verb|merged_bin| & \verb|merged_gz| & \verb|merged_bin_gz| \\ \hline
\verb|[rj_extension]|   & \verb|.rj|  & \verb|.brj| & \verb|.rj.gz|  & \verb|.brj.gz| \\
\verb|[rlm_extension]| & \verb|.rlm| & \verb|.blm| & \verb|.rlm.gz| & \verb|.blm.gz| \\
\verb|[rtm_extension]| & \verb|.rtm| & \verb|.btm| & \verb|.rtm.gz| & \verb|.btm.gz| \\
\verb|[rtp_extension]| & \verb|.rtp| & \verb|.btp| & \verb|.rtp.gz| & \verb|.btp.gz| \\ \hline
\verb|[fem_extension]| & \verb|.gfm| & \verb|.gfb| & \verb|.gfm.gz| & \verb|.gfb.gz| \\ \hline \hline
  \multicolumn{5}{c}{{\tt [\#]} is the domain or process number} \\
\end{tabular}
\end{center}
\label{table:mesh_format}
\end{table}
%
\subsubsection{Control file (\tt{control\_sph\_shell})}
\label{section:control_sph_shell}
Control file ({\tt control\_sph\_shell}) consists the following items. Detailed description for each item can be checked by clicking each item.
\\
%
\verb|spherical_shell_ctl|
\label{href_i:spherical_shell_ctl}
\\
\\
%
Block {\tt MHD\_control} (Top block of the control file)
	\begin{itemize}
	\item Block \hyperref[href_t:data_files_def]{\tt data\_files\_def}
	\item File \hyperref[href_t:spherical_shell_ctl]{\tt spherical\_shell\_ctl    [resolution\_control]}
	\item or Block \hyperref[href_t:spherical_shell_ctl]{\tt spherical\_shell\_ctl}
	\end{itemize}

If \verb|num_radial_domain_ctl| and \verb|num_horizontal_domain_ctl| are defined, the following arrays \verb|num_domain_sph_grid|, \verb|num_domain_legendre|, and \verb|num_domain_spectr| are not necessary. \\
(see \hyperref[href_t:gen_w_innercore]{example} \verb|spherical_shell/with_inner_core|)
%
\subsubsection{Spectrum index data}
\verb|gen_sph_grid| generates indexing table of the spherical transform. To perform spherical harmonics transform with distributed memory computers, data communication table is also included in these files. Calypso needs four indexing data for the spherical transform.
%
\begin{description}
\item{\verb|[sph_prefix].[rj_extension]|} Indexing table for spectrum data $f(r,l,m)$ to calculate linear terms. In program,  spherical harmonics modes $(l,m)$ is indexed by $j = l(l+1) + m$. The spectrum data are decomposed by spherical harmonics modes $j$. Data communication table for Legendre transform is included. The data also have the radial index of the ICB and CMB. Extension \verb|[rj_extension]| is listed in Table \ref{table:mesh_format}.
\item{\verb|[sph_prefix].[rlm_extension]|} Indexing table for spectrum data $f(r,l,m)$ for Legendre transform. The spectrum data are decomposed by radial direction $r$ and spherical harmonics order $m$. Data communication table to caricurate liner terms is included. Extension \verb|[rlm_extension]| is listed in Table \ref{table:mesh_format}.
\item{\verb|[sph_prefix].[rtm_extension]|} Indexing table for data $f(r,\theta,m)$ for Legendre transform. The data are decomposed by radial direction $r$ and spherical harmonics order $m$. Data communication table for backward Fourier transform is included. Extension \verb|[rtm_extension]| is listed in Table \ref{table:mesh_format}.
\item{\verb|[sph_prefix].[rtp_extension]|} Indexing table for data $f(r,\theta,m)$ for Fourier transform and field data $f(r,\theta,\phi)$. The data are decomposed by radial direction $r$ and meridional direction $\theta$. Data communication table for forward Legendre transform is included. Extension \verb|[rtp_extension]| is listed in Table \ref{table:mesh_format}.
\end{description}
%

\subsubsection{Finite element mesh data (optional)}
Calypso generates field data for visualization with XDMF or VTK format. To generate field data file, the preprocessing program generates FEM mesh data for each subdomain of spherical grid $(r,\theta,\phi)$ under the Cartesian coordinate $(x,y,z)$. The mesh data file is written based on GeoFEM (\url{http://geofem.tokyo.rist.or.jp}) mesh data format, which consists of each subdomain mesh and communication table among overlapped nodes. The extension of the mesh file is listed in Table \ref{table:mesh_format}. This mesh data is only used in the programs \hyperref[sec:sectioning]{\tt sectioning} and \verb|field_to_VTK|.


\subsection{Volume rendering data (Parallel volume rendering module)}

Calypso includes parallel volume rendering (PVR) module for scalar field visualization. The PVR module generates volume rendering data during time integration without writing large size volume data.

To output volume rendering data, increment of the volume rendering needs to be defined at \verb|i_step_sectioning_ctl| in \verb|time_step_ctl| block. Parameters foe each rendering are defined in array block \verb|volume_rendering| in \verb|visual_control|. Each  \verb|volume_rendering| block can be stored in an external file, and the external file can be defined as \verb|file volume_rendering [File_Name]|.

\paragraph{Control data}
The control parameters for PVR is the following:
\\
%
Block \verb|volume_rendering| (Top level for volume rendering)
\\
\label{href_i:volume_rendering}
\begin{itemize}
	\item \hyperref[href_t:updated_sign]
			{\tt updated\_sign    [SIGNAL]}
	\item \hyperref[href_t:pvr_file_prefix]
			{\tt pvr\_file\_prefix    [File\_Prefix]}
	\item \hyperref[href_t:pvr_output_format]
			{\tt pvr\_output\_format    [File\_Format]}
	\item \hyperref[href_t:monitoring_mode]
			{\tt monitoring\_mode       [ON/OFF]}
%
	\item \hyperref[href_t:stereo_imaging]
			{\tt stereo\_imaging          [ON/OFF]}
	\item \hyperref[href_t:anaglyph_switch]
			{\tt anaglyph\_switch         [ON/OFF]}
	\item \hyperref[href_t:quilt_3d_imaging]
			{\tt quilt\_3d\_imaging       [ON/OFF]}
%
	\item \hyperref[href_t:output_field]
			{\tt output\_field            [Field\_Name]}
	\item \hyperref[href_t:output_component]
			{\tt output\_component       [Component\_Name]}
%
    	\item File or Block \verb|view_transform_ctl|
				\label{href_i:view_transform_ctl}
		\begin{itemize}
    			\item Block \verb|image_size_ctl|
						\label{href_i:image_size_ctl}
				\begin{itemize}
					\item \hyperref[href_t:x_pixel_ctl]
							{\tt x\_pixel\_ctl   [\# of PIXELS]}
					\item \hyperref[href_t:y_pixel_ctl]
							{\tt y\_pixel\_ctl   [\# of PIXELS]}
				\end{itemize}
%
			\item Array \hyperref[href_t:modelview_matrix_ctl]
					{\tt modelview\_matrix\_ctl   [X/Y/X/W] [X/Y/Z/W]  [VALUE]}
			\item Array \hyperref[href_t:look_at_point_ctl]
					{\tt look\_at\_point\_ctl   [X/Y/Z]    [VALUE]}
			\item Array \hyperref[href_t:eye_position_ctl]
					{\tt eye\_position\_ctl   [X/Y/Z]     [VALUE]}
			\item Array \hyperref[href_t:up_direction_ctl]
					{\tt up\_direction\_ctl   [X/Y/Z]     [VALUE]}
			\item Array \hyperref[href_t:view_rotation_vec_ctl]
					{\tt view\_rotation\_vec\_ctl   [X/Y/Z]     [VALUE]}
			\item \hyperref[href_t:view_rotation_deg_ctl]
					{\tt view\_rotation\_deg\_ctl     [DEGREE]}
			\item \hyperref[href_t:scale_factor_ctl]
					{\tt scale\_factor\_ctl     [SCALE]}
			\item Array \hyperref[href_t:scale_factor_vec_ctl]
					{\tt scale\_factor\_vec\_ctl  [X/Y/Z]     [SCALE]}
			\item Array \hyperref[href_t:eye_position_in_viewer_ctl]
					{\tt eye\_position\_in\_viewer\_ctl   [X/Y/Z]    [VALUE]}
%
			\item \hyperref[href_t:projection_type_ctl]
					{\tt projection\_type\_ctl   [TYPE]}
%
   			\item Block \verb|projection_matrix_ctl|
						\label{href_i:projection_matrix_ctl}
				\begin{itemize}
					\item \hyperref[href_t:perspective_angle_ctl]
							{\tt perspective\_angle\_ctl   [DEGREE]}
					\item \hyperref[href_t:perspective_xy_ratio_ctl]
							{\tt perspective\_xy\_ratio\_ctl   [ASPECT]}
					\item \hyperref[href_t:perspective_near_ctl]
							{\tt perspective\_near\_ctl   [NEAR\_DISTANCE]}
					\item \hyperref[href_t:perspective_far_ctl]
							{\tt perspective\_far\_ctl   [FAR\_DISTANCE]}
%
%					\item \hyperref[href_t:horizontal_range_ctl]
%							{\tt horizontal\_range\_ctl   [LEFT]  [RIGHT[}
%					\item \hyperref[href_t:vertical_range_ctl]
%							{\tt vertical\_range\_ctl   [BOTTOM]  [TOP]}
				\end{itemize}
%
    			\item Block \verb|stereo_view_parameter_ctl|
						\label{href_i:stereo_view_parameter_ctl}
				\begin{itemize}
					\item \hyperref[href_t:focal_distance_ctl]
							{\tt focal\_distance\_ctl   [DISTANCE]}
					\item \hyperref[href_t:eye_separation_ctl]
							{\tt eye\_separation\_ctl   [SEPARATION]}
					\item \hyperref[href_t:eye_separation_angle]
							{\tt eye\_separation\_angle   [DEGREE]}
					\item \hyperref[href_t:eye_separation_step_by_angle]
							{\tt eye\_separation\_step\_by\_angle   [ON/OFF]}
				\end{itemize}
%
		\end{itemize}
%
	\item Block \verb|plot_area_ctl|
				\label{href_i:plot_area_ctl}
		\begin{itemize}
			\item \hyperref[href_t:chosen_ele_grp_ctl]
					{\tt chosen\_ele\_grp\_ctl    [AREA\_NAME]}
			\item \hyperref[href_t:surface_enhanse_ctl]
					{\tt surface\_enhanse\_ctl    [SURFACE\_NAME]  [TYPE]  [OPACITY]}
		\end{itemize}
%
	\item File or Block \verb|pvr_color_ctl|
			\label{href_i:pvr_color_ctl}
		\begin{itemize}
    			\item Block \verb|colormap_ctl|
 			\item Block \verb|colorbar_ctl|
		\end{itemize}
%
	\item Block \verb|colormap_ctl|
			\label{href_i:colormap_ctl}
		\begin{itemize}
			\item \hyperref[href_t:colormap_mode_ctl]
					{\tt colormap\_mode\_ctl    [MODE]}
			\item \hyperref[href_t:background_color_ctl]
					{\tt background\_color\_ctl    [MODE]}
%
%			\item \hyperref[href_t:LIC_color_field]
%					{\tt LIC\_color\_field    [Field\_Name]}
%			\item \hyperref[href_t:LIC_color_componenet]
%					{\tt LIC\_color\_componenet    [Component\_Name]}
%			\item \hyperref[href_t:LIC_transparent_field]
%					{\tt LIC\_transparent\_field    [Field\_Name]}
%			\item \hyperref[href_t:LIC_transparent_componenet]
%					{\tt LIC\_transparent\_componenet    [Component\_Name]}
%
			\item \hyperref[href_t:data_mapping_ctl]
					{\tt data\_mapping\_ctl    [TYPE]}
			\item \hyperref[href_t:range_min_ctl]
					{\tt range\_min\_ctl    []MIN\_VALUE}
			\item \hyperref[href_t:range_max_ctl]
					{\tt range\_max\_ctl    [MAX\_VALUE]}
			\item \hyperref[href_t:color_table_ctl]
					Array {\tt color\_table\_ctl    [VALUE]  [COLOR\_VAL]}
			\item \hyperref[href_t:opacity_style_ctl]
					{\tt opacity\_style\_ctl    [TYPE]}
			\item \hyperref[href_t:constant_opacity_ctl]
					{\tt constant\_opacity\_ctl    [OPACITY]}
			\item \hyperref[href_t:linear_opacity_ctl]
					Array {\tt linear\_opacity\_ctl    [VALUE]  [OPACITY]}
		\end{itemize}
%
	\item Block \verb|colorbar_ctl|
			\label{href_i:colorbar_ctl}
		\begin{itemize}
			\item \hyperref[href_t:colorbar_switch_ctl]
					{\tt colorbar\_switch\_ctl    [ON/OFF]}
			\item \hyperref[href_t:colorbar_scale_ctl]
					{\tt colorbar\_scale\_ctl    [ON/OFF]}
			\item \hyperref[href_t:font_size_ctl]
					{\tt font\_size\_ctl    [SIZE]}
			\item \hyperref[href_t:num_grid_ctl]
					{\tt num\_grid\_ctl    [\# of grid]}
			\item \hyperref[href_t:zeromarker_switch]
					{\tt zeromarker\_switch    [ON/OFF]}
			\item \hyperref[href_t:colorbar_range]
					{\tt colorbar\_range    [MIN]  [MAX]}
			\item \hyperref[href_t:axis_label_switch]
					{\tt axis\_label\_switch    [ON/OFF]}
			\item \hyperref[href_t:time_label_switch]
					{\tt time\_label\_switch    [ON/OFF]}
			\item \hyperref[href_t:map_grid_switch]
					{\tt map\_grid\_switch    [ON/OFF]}
		\end{itemize}
%
	\item File or Block \verb|lighting_ctl  [File_Name]|
				\label{href_i:lighting_ctl}
		\begin{itemize}
			\item \hyperref[href_t:ambient_coef_ctl]
					{\tt ambient\_coef\_ctl     [VALUE]}
			\item \hyperref[href_t:diffuse_coef_ctl]
					{\tt diffuse\_coef\_ctl     [VALUE]}
			\item \hyperref[href_t:specular_coef_ctl]
					{\tt specular\_coef\_ctl     [VALUE]}
			\item \hyperref[href_t:position_of_lights]
					Array {\tt position\_of\_lights     [X]  [Y]  [Z]}
			\item \hyperref[href_t:sph_position_of_lights]
					Array {\tt sph\_position\_of\_lights     [R]  [THETA]  [PHI]}
		\end{itemize}
%
	\item Array Block \verb|section_ctl|
				\label{href_i:section_ctl}
		\begin{itemize}
   			 \item File or Block \hyperref[href_t:surface_define]
			 		{\tt surface\_define  [File\_Name]}
			\item \hyperref[href_t:opacity_ctl]
					{\tt opacity\_ctl     [VALUE]}
			\item \hyperref[href_t:zeroline_switch_ctl]
				{\tt zeroline\_switch\_ctl     [ON/OFF]}
		\end{itemize}
%
	\item Array Block \verb|isosurface_ctl|
				\label{href_i:isosurface_ctl}
		\begin{itemize}
			\item \hyperref[href_t:isosurf_value]
					{\tt isosurf\_value     [VALUE]}
			\item \hyperref[href_t:opacity_ctl]
					{\tt opacity\_ctl     [VALUE]}
			\item \hyperref[href_t:surface_direction]
					{\tt surface\_direction     [DIRECTION]}
		\end{itemize}
%
	\item Block        \verb|quilt_image_ctl|
			\label{href_i:quilt_image_ctl}
		\begin{itemize}
			\item \hyperref[href_t:num_column_row_ctl]
					{\tt num\_column\_row\_ctl     [\# of column] [\# of row]}
			\item \hyperref[href_t:num_row_column_ctl]
					{\tt num\_row\_column\_ctl     [\# of row] [\# of column]}
	    	\item Array Block \verb|view_transform_ctl|
					\label{href_i:quilt_view_transform_ctl}
		\end{itemize}
%
	\item Block        \verb|snapshot_movie_ctl|
				\label{href_i:snapshot_movie_ctl}
		\begin{itemize}
			\item \hyperref[href_t:movie_mode_ctl]
					{\tt movie\_mode\_ctl       [Mode]}
			\item \hyperref[href_t:num_frames_ctl]
					{\tt num\_frames\_ctl       [\# of Flame]}
			\item \hyperref[href_t:rotation_axis_ctl]
					{\tt rotation\_axis\_ctl    [X/Y/Z]}
			\item \hyperref[href_t:angle_range]
					{\tt angle\_range    [START] [END]}
			\item \hyperref[href_t:apature_range]
					{\tt apature\_range  [START] [END]}
			\item \hyperref[href_t:LIC_kernel_peak_range]
					{\tt LIC\_kernel\_peak\_range  [START] [END]}
%
	    		\item File or Block \verb|start_view_control [File_Name]|
					\label{href_i:start_view_control}
	   	 	\item File or Block \verb|end_view_control  [File_Name]|
					\label{href_i:end_view_control}
	  	  	\item Array File or Block \verb|view_transform_ctl  [File_Name] |
					\label{href_i:movie_view_transform_ctl}
		\end{itemize}
\end{itemize}


\subsection{Map projection data (Parallel volume rendering module)}
Map projection module generate visualization image using map projection through surfacing module. Currently, contour plots for scalar fields can only be generated, and Aitoff projection is only supported in this module.

To output map projection data, increment of the map projection needs to be defined at \verb|i_step_map_projection_ctl| in \verb|time_step_ctl| block. Parameters foe each rendering are defined in array block \verb|map_rendering_ctl| in \verb|visual_control|. Same the other visualizatiom modules, each  \verb|map_rendering_ctl| block can be stored in an external file, and the external file can be defined as \\
 \verb|file map_rendering_ctl [File_Name]|.
%
\paragraph{Control data}
The control parameters for map projection module is the following:
\\
%
Block \verb|map_rendering_ctl| (Top level for map prodection)
\\
\label{href_i:map_rendering_ctl}
\begin{itemize}
	\item \hyperref[href_t:map_image_prefix]
			{\tt map\_image\_prefix    [File\_Prefix]}
	\item \hyperref[href_t:map_image_format]
			{\tt map\_image\_format    [File\_Format]}
%
	\item \hyperref[href_t:output_field]
			{\tt output\_field            [Field\_Name]}
	\item \hyperref[href_t:output_component]
			{\tt output\_component       [Component\_Name]}
	\item \hyperref[href_t:isoline_field]
			{\tt isoline\_field                [Field\_Name]}
	\item \hyperref[href_t:isoline_component]
			{\tt isoline\_component     [Component\_Name]}
%
	\item Block \verb|section_ctl|
			\label{href_i:map_section_ctl}
		\begin{itemize}
    			\item File or Block \verb|surface_define|
						\label{href_i:map_surface_define}
%
			\item \hyperref[href_t:zeroline_switch_ctl]
					{\tt zeroline\_switch\_ctl     [ON/OFF]}
			\item \hyperref[href_t:isoline_color_mode]
					{\tt isoline\_color\_mode     [MODE]}
			\item \hyperref[href_t:isoline_number_ctl]
					{\tt isoline\_number\_ctl     [\# of LINES]}
			\item \hyperref[href_t:isoline_range_ctl]
					{\tt isoline\_range\_ctl     [MIN\_VALUE]  [MAX\_VALUE]}
			\item \hyperref[href_t:isoline_width_ctl]
					{\tt isoline\_width\_ctl     [WIDTH]}
			\item \hyperref[href_t:grid_width_ctl]
					{\tt grid\_width\_ctl     [WIDTH]}
			\item \hyperref[href_t:tangent_cylinder_switch_ctl]
					{\tt tangent\_cylinder\_switch\_ctl     [ON/OFF]}
			\item \hyperref[href_t:inner_radius_ctl]
					{\tt inner\_radius\_ctl     [RADIUS]}
			\item \hyperref[href_t:outer_radius_ctl]
					{\tt outer\_radius\_ctl     [RADIUS]}
		\end{itemize}
%
	\item File or Block \verb|map_projection_ctl|
				\label{href_i:map_projection_ctl}
		\begin{itemize}
    			\item Block \verb|image_size_ctl|
						\label{href_i:image_size_ctl}
				\begin{itemize}
					\item \hyperref[href_t:x_pixel_ctl]
							{\tt x\_pixel\_ctl   [\# of PIXELS]}
					\item \hyperref[href_t:y_pixel_ctl]
							{\tt y\_pixel\_ctl   [\# of PIXELS]}
				\end{itemize}
%
%			\item Array \hyperref[href_t:modelview_matrix_ctl]
%					{\tt modelview\_matrix\_ctl   [X/Y/X/W] [X/Y/Z/W]  [VALUE]}
%			\item Array \hyperref[href_t:look_at_point_ctl]
%					{\tt look\_at\_point\_ctl   [X/Y/Z]    [VALUE]}
%			\item Array \hyperref[href_t:eye_position_ctl]
%					{\tt eye\_position\_ctl   [X/Y/Z]     [VALUE]}
%			\item Array \hyperref[href_t:up_direction_ctl]
%					{\tt up\_direction\_ctl   [X/Y/Z]     [VALUE]}
%			\item Array \hyperref[href_t:view_rotation_vec_ctl]
%					{\tt view\_rotation\_vec\_ctl   [X/Y/Z]     [VALUE]}
%			\item \hyperref[href_t:view_rotation_deg_ctl]
%					{\tt view\_rotation\_deg\_ctl     [DEGREE]}
%			\item \hyperref[href_t:scale_factor_ctl]
%					{\tt scale\_factor\_ctl     [SCALE]}
%			\item Array \hyperref[href_t:scale_factor_vec_ctl]
%					{\tt scale\_factor\_vec\_ctl  [X/Y/Z]     [SCALE]}
%			\item Array \hyperref[href_t:eye_position_in_viewer_ctl]
%					{\tt eye\_position\_in\_viewer\_ctl   [X/Y/Z]    [VALUE]}
%
			\item \hyperref[href_t:projection_type_ctl]
					{\tt projection\_type\_ctl   [TYPE]}
%
   			\item Block \verb|projection_matrix_ctl|
						\label{href_i:projection_matrix_ctl}
				\begin{itemize}
%					\item \hyperref[href_t:perspective_angle_ctl]
%							{\tt perspective\_angle\_ctl   [DEGREE]}
%					\item \hyperref[href_t:perspective_xy_ratio_ctl]
%							{\tt perspective\_xy\_ratio\_ctl   [ASPECT]}
%					\item \hyperref[href_t:perspective_near_ctl]
%							{\tt perspective\_near\_ctl   [NEAR\_DISTANCE]}
%					\item \hyperref[href_t:perspective_far_ctl]
%							{\tt perspective\_far\_ctl   [FAR\_DISTANCE]}
%
					\item \hyperref[href_t:horizontal_range_ctl]
							{\tt horizontal\_range\_ctl   [LEFT]  [RIGHT[}
					\item \hyperref[href_t:vertical_range_ctl]
							{\tt vertical\_range\_ctl   [BOTTOM]  [TOP]}
				\end{itemize}
		\end{itemize}
%
	\item File \verb|map_color_ctl|
				\label{href_i:map_color_ctl}
		\begin{itemize}
    			\item Block \verb|colormap_ctl|
 			\item Block \verb|colorbar_ctl|
		\end{itemize}
%
	\item Block \verb|colormap_ctl|
				\label{href_i:colormap_ctl}
		\begin{itemize}
			\item \hyperref[href_t:colormap_mode_ctl]
					{\tt colormap\_mode\_ctl    [MODE]}
			\item \hyperref[href_t:background_color_ctl]
					{\tt background\_color\_ctl    [MODE]}
%
			\item \hyperref[href_t:data_mapping_ctl]
					{\tt data\_mapping\_ctl    [TYPE]}
			\item \hyperref[href_t:range_min_ctl]
					{\tt range\_min\_ctl    []MIN\_VALUE}
			\item \hyperref[href_t:range_max_ctl]
					{\tt range\_max\_ctl    [MAX\_VALUE]}
			\item \hyperref[href_t:color_table_ctl]
					Array {\tt color\_table\_ctl    [VALUE]  [COLOR\_VAL]}
		\end{itemize}
%
	\item Block \verb|colorbar_ctl|
			\label{href_i:colorbar_ctl}
		\begin{itemize}
			\item \hyperref[href_t:colorbar_switch_ctl]
					{\tt colorbar\_switch\_ctl    [ON/OFF]}
			\item \hyperref[href_t:colorbar_scale_ctl]
					{\tt colorbar\_scale\_ctl    [ON/OFF]}
			\item \hyperref[href_t:font_size_ctl]
					{\tt font\_size\_ctl    [SIZE]}
			\item \hyperref[href_t:num_grid_ctl]
					{\tt num\_grid\_ctl    [\# of grid]}
			\item \hyperref[href_t:zeromarker_switch]
					{\tt zeromarker\_switch    [ON/OFF]}
			\item \hyperref[href_t:colorbar_range]
					{\tt colorbar\_range    [MIN]  [MAX]}
			\item \hyperref[href_t:time_label_switch]
					{\tt time\_label\_switch    [ON/OFF]}
			\item \hyperref[href_t:map_grid_switch]
					{\tt map\_grid\_switch    [ON/OFF]}
		\end{itemize}
\end{itemize}



\subsubsection{\tt volume\_rendering}
\label{href_t:volume_rendering}
%
\paragraph{\tt updated\_sign}
\label{href_t:updated_sign}  
\verb|[SIGNAL]| \\
The program will read again PVR parameter files and updates PVR parameters if the text \verb|[SIGNAL]| is changed. 
%
\paragraph{\tt pvr\_file\_prefix}
\label{href_t:pvr_file_prefix}  
\verb|[File_Prefix]| \\
File prefix of the output image file  \verb|[File_Prefix]| is defined by text. 
%
\paragraph{\tt pvr\_output\_format}
\label{href_t:pvr_output_format}  
\verb|[File_Format]| \\
File format of the output image file  \verb|[File_Format]| is defined by text.  The following format can be defined.
\begin{description}
\item{\tt BMP: }               Bitmap format
\item{\tt PNG: }               PNG format  (zlib or libpng is required to build)
\item{\tt QUILT: }             BMP Quilt format for holograms
\item{\tt QUILT\_GZ: }      Compressed BMP Quilt format for holograms
\end{description}
%
\paragraph{\tt monitoring\_mode}
\label{href_t:monitoring_mode}  
\verb|[ON/OFF]| \\
When the monitoring mode is turned on, the program outputs two same image files. One file is named with step number, and another file is named without step number. Consequently, the image file without step number is overwritten every PVR data output. During the simulation, we can check the latest image.
%
\paragraph{\tt stereo\_imaging}
\label{href_t:stereo_imaging}  
\verb|[ON/OFF]| \\
When stereo imaging switch is turned on, Stereo image is generated. In stereo imaing mode, the following \hyperref[href_t:anaglyph_switch]{\tt anaglyph\_switch} or \hyperref[href_t:quilt_3d_imaging]{\tt quilt\_3d\_imaging} needs to be turned on.
%
\paragraph{\tt anaglyph\_switch}
\label{href_t:anaglyph_switch}  
\verb|[ON/OFF]| \\
When anaglyph switch is turned on, stereo anaglyph image is generated. To look anaglyph, red and blue glass is required.
%
\paragraph{\tt quilt\_3d\_imaging}
\label{href_t:quilt_3d_imaging}  
\verb|[ON/OFF]| \\
When anaglyph switch is turned on, stereo image for looking glass is generated. This switch is also turned on if stereo image with side-by-side is generated.
%
\paragraph{\tt output\_field}
\label{href_t:output_field}  
\verb|[Field_Name]| \\
Field name \verb|[Field_Name]| for rendering is defined by text.
%
\paragraph{\tt output\_component}
\label{href_t:output_component}  
\verb|[Component_Name]| \\
Component name \verb|[Component_Name]| for rendering is defined by text.
%
\subsubsection{\tt view\_transform\_ctl}
\label{href_t:view_transform_ctl}
Parameters for view point and direction are defined in this block. This block can be saved into an external file. (Go to \hyperref[href_i:view_transform_ctl] {\tt view\_transform\_ctl})

\subsubsection*{\tt image\_size\_ctl}
\label{href_t:image_size_ctl}
Image size (number of pixels) are defined in this block. (Go to \hyperref[href_i:image_size_ctl] {\tt image\_size\_ctl})
%
\paragraph{\tt x\_pixel\_ctl}
\label{href_t:x_pixel_ctl}  
\verb|[\# of PIXELS]| \\
Number of pixels in the horizontal direction \verb|[\# of PIXELS]| is defined by integer.
%
\paragraph{\tt y\_pixel\_ctl}
\label{href_t:y_pixel_ctl}  
\verb|[\# of PIXELS]| \\
Number of pixels in the vertical direction \verb|[\# of PIXELS]| is defined by integer.
%
\paragraph{\tt modelview\_matrix\_ctl}
\label{href_t:modelview_matrix_ctl}  
\verb|[X/Y/X/W] [X/Y/Z/W]  [VALUE]| \\
Modelview matrix $A_{ij}$ is defined by array. Directions of row and columns are defined in the first and second texts, respectively, and the value of the each matrix component is defined by the third real value.
%
\paragraph{\tt look\_at\_point\_ctl}
\label{href_t:look_at_point_ctl}  
\verb|[X/Y/X] [VALUE]| \\
Position to look at is defined by array. Directions are defined in the first text, and the value is defined by the third real value.
%
\paragraph{\tt eye\_position\_ctl}
\label{href_t:eye_position_ctl}  
\verb|[X/Y/X] [VALUE]| \\
Position of eye (or camera) is defined by array. Directions are defined in the first text, and the value is defined by the third real value.
%
%

%
\subsubsection{\tt map\_rendering\_ctl}
\label{href_t:map_rendering_ctl}
%




\newpage
\begin{thebibliography}{10}

\bibitem{Bullard:54} Bullard, E. C. and Gellman, H., Homogeneous dynamos and terrestrial magnetism, {\it Proc. of the Roy. Soc. of London}, {\bf A247}, 213--278, 1954.
\bibitem{Uli:2001} Christensen, U.R., Aubert, J., Cardin, P., Dormy, E., Gibbons, S., Glatzmaier, G. A., Grote, E., Honkura, H., Jones, C., Kono, M., Matsushima, M., Sakuraba, A., Takahashi, F., Tilgner, A., Wicht, J. and Zhang, K., A numerical dynamo benchmark, {\it Physics of the Earth and Planetary Interiors}, {\bf 128}, 25--34, 2001.

\end{thebibliography}


\newpage
\begin{appendices}
\section{Definition of parameters for control files}
\label{section:def_control}

\subsection{Block {\tt data\_files\_def}}
\label{href_t:data_files_def}
File names and number of processes and threads are defined in this block. \\
\hyperref[href_i:MHD_control]{(Back to {\tt control\_MHD})} \\
\hyperref[href_i:spherical_shell_ctl]{(Back to {\tt control\_sph\_shell})} \\
\hyperref[href_i:assemble_control]{(Back to {\tt control\_assemble\_sph})}

\paragraph{\tt num\_subdomain\_ctl}
\label{href_t:num_subdomain_ctl}
\verb|[Num_PE]| \\
Number of subdomain for the MPI program \verb|[Num_PE]| is defined by integer. If number of processes in \verb| mpirun -np | is different from number of subdomains, program will be stopped with message.

\paragraph{\tt num\_smp\_ctl}
\label{href_t:num_smp_ctl}
\verb|[Num_Threads]| \\
Number of SMP threads for OpenMP \verb|[Num_Threads]| is defined by integer. You can set larger number than the actual umber of thread to be used. If actual number of thread is less than this number, number of threads is set to the number which is defined in this field.

\paragraph{\tt sph\_file\_prefix}
\label{href_t:sph_file_prefix}
\verb|[sph_prefix]| \\
File prefix of spherical harmonics indexing and FEM mesh file \verb|[sph_prefix]| is defined by text. Process ID and extension are added after this file prefix.

\paragraph{\tt mesh\_file\_prefix}
\label{href_t:mesh_file_prefix}
\verb|[mesh_prefix]| \\
File prefix of FEM mesh file \verb|[mesh_prefix]| is defined by text. Process ID and extension are added after this file prefix. This flag is only used for the sectioning program (\hyperref[sec:sectioning]{\tt sectioning}) and data converter to VTK (\hyperref[sec:field_to_VTK]{\tt field\_to\_VTK}).

\paragraph{\tt boundary\_data\_file\_name}
\label{href_t:boundary_data_file_name}
\verb|[boundary_data_name]| \\
File name of boundary condition data file \verb|[File_Name]| is defined by text. 

\paragraph{\tt radial\_field\_file\_name}
\label{href_t:radial_field_file_name}
\verb|[radial_field_file_name]| \\
File name of radial field data file \verb|[File_Name]| is defined by text. 

\paragraph{\tt restart\_file\_prefix}
\label{href_t:restart_file_prefix}
\verb|[rst_prefix]| \\
File prefix of spectrum data for restarting and snapshots \verb|[rst_prefix]| is defined by text. Step number, process ID, and extension are added after this file prefix.

\paragraph{\tt field\_file\_prefix}
\label{href_t:field_file_prefix}
\verb|[fld_prefix]| \\
File prefix of field data for visualize snapshots \verb|[fld_prefix]| is defined by text. Step number and file extension are  added after this file prefix.

\paragraph{\tt sph\_file\_fmt\_ctl}
\label{href_t:sph_file_fmt_ctl}
\verb|[sph_formayt]| \\
File format of spherical harmonics indexing and FEM mesh file \verb|[sph_format]| is defined by text. Following data formats can be defined. Extensions of each data format is listed in Table \ref{table:mesh_format}.
%
\begin{description}
\item{\tt ascii: }   Distributed ASCII data
\item{\tt binary: }  Distributed binary data
\item{\tt merged: }  Merged ASCII data
\item{\tt merged\_bin: }   Merged binary data
\item{\tt gzip: }            Compressed distributed ASCII data
\item{\tt binary\_gz: }      Compressed distributed binary data
\item{\tt merged\_gz: }      Compressed merged ASCII data
\item{\tt merged\_bin\_gz: } Compressed merged binary data
\end{description}
%

\paragraph{\tt mesh\_file\_fmt\_ctl}
\label{href_t:mesh_file_fmt_ctl}
\verb|[mesh_formayt]| \\
File format of FEM mesh file \verb|[mesh_format]| is defined by text. Data formats can be defined the same as {\tt sph\_file\_fmt\_ctl}. Extensions of each data format is listed in Table \ref{table:mesh_format}. This flag is only used for the sectioning program (\hyperref[sec:sectioning]{\tt sectioning}) and data converter to VTK (\hyperref[sec:field_to_VTK]{\tt field\_to\_VTK}).

\paragraph{\tt restart\_file\_fmt\_ctl}
\label{href_t:restart_file_fmt_ctl}
\verb|[rst_format]| \\
File format of restart files \verb|[rst_format]| is defined by text. Following data formats can be defined. Extensions of each data format is listed in Table \ref{table:restart_format}.
%
\begin{description}
\item{\tt ascii: }   Distributed ASCII data
\item{\tt binary: }  Distributed binary data
\item{\tt merged: }  Merged ASCII data
\item{\tt merged\_bin: }   Merged binary data
\item{\tt gzip: }            Compressed distributed ASCII data
\item{\tt binary\_gz: }      Compressed distributed binary data
\item{\tt merged\_gz: }      Compressed merged ASCII data
\item{\tt merged\_bin\_gz: } Compressed merged binary data
\end{description}
%

\paragraph{\tt field\_file\_fmt\_ctl}
\label{href_t:field_file_fmt_ctl}
\verb|[fld_format]| \\
Field data field format for visualize snapshots \verb|[fld_format]| is defined by text. The following formats are currently supported.
%
\begin{description}
\item{\tt single\_HDF5: }  Merged HDF5 file (Available if HDF5 library is linked)
\item{\tt single\_VTK: }   Merged VTK file (Default)
\item{\tt VTK: }           Distributed VTK file
\item{\tt single\_VTK\_gz: }   Compressed merged VTK file (Available if zlib library is linked)
\item{\tt VTK\_gz: }           Compressed distributed VTK file (Available if zlib library is linked)
\end{description}
%
%
%
%

\subsection{\tt spherical\_shell\_ctl}
\label{href_t:spherical_shell_ctl}
Configuration of the spherical shell and parallelization are defined by in this block. This block can be stored in an external file.
%

\subsubsection{\tt FEM\_mesh\_ctl}
\label{href_t:FEM_mesh_ctl}
Configuration of the FEM mesh is defined in this block. This block is optional.
\hyperref[href_i:FEM_mesh_ctl]{(Back to {\tt control\_sph\_shell})}

\paragraph{\tt FEM\_mesh\_output\_switch}
\label{href_t:FEM_mesh_output_switch}
\verb|[ON or OFF]| \\
Set \verb|ON| if FEM mesh data need to be written.
%

\subsubsection{\tt num\_domain\_ctl}
\label{href_t:num_domain_ctl}
Parallelization is defined in this block. Domain decomposition is defined for spectrum data, field data, and Legendre transform. \\
\hyperref[href_i:num_domain_ctl]{(Back to {\tt control\_sph\_shell})}

\paragraph{\tt ordering\_set\_ctl}
\label{href_t:ordering_set_ctl}
\verb|[ORDERING_SET]| \\
 Ordering set of spherical harmonics and grid data is defined here.  The following text parameter set \verb|[ORDERING_SET]| is available:
 \begin{description}
 \item[{\tt Ver\_2}]  Optimized ordering for Ver. 2.
 \item[{\tt Ver\_1}] Original data ordering for Ver. 1.x
 \end{description} 

If \verb|[ORDERING_SET]| is not defined, ordering set for \verb|Ver_2| is chosen.

\paragraph{\tt num\_radial\_domain\_ctl}
\label{href_t:num_radial_domain_ctl}
\verb|[Ndomain]| \\
Number of subdomains in the radial direction for the spherical grid $(r, \theta, \phi)$ and spherical transforms $(r, \theta, m)$ and $(r, l, m)$.

\paragraph{\tt num\_horizontal\_domain\_ctl}
\label{href_t:num_horizontal_domain_ctl} 
\verb|[Ndomain]| \\
Number of subdomains in the horizontal direction. The number will be the number of subdomains for the meridional directios for the spherical grid $(r, \theta, \phi)$ and Fourier transform $(r, \theta, m)$. For Legendre transform $(r, \theta, m)$ and $(r, l, m)$, the number will be the number of subdomains for the h.armonics ordedr $m$.


\paragraph{\color{magenta} \tt num\_domain\_sph\_grid    [Direction]    [Ndomain]}
\label{href_t:num_domain_sph_grid} 
{\color{magenta} (Depricated)}\\
 Definition of number of subdomains for physical data in spherical coordinate $(r, \theta, \phi)$. Direction {\tt  radial} or {\tt meridional} is set in \verb|[Direction]|, and number of subdomains \verb|[Ndomain]| are defined in the integer field.

\paragraph{\color{magenta} \tt num\_domain\_legendre    [Direction]    [Ndomain]}
\label{href_t:num_domain_legendre}
{\color{magenta} (Depricated)}\\
 Definition of number of subdomains for Legendre transform between $(r, \theta, m)$ and $(r, l, m)$. Direction {\tt  radial} or {\tt zonal} is set in \verb|[Direction]|, and number of subdomains \verb|[Ndomain]| are defined in the integer field.

\paragraph{\color{magenta} \tt num\_domain\_spectr    [Direction]    [Ndomain]}
\label{href_t:num_domain_spectr}
{\color{magenta} (Depricated)}\\
Definition of number of subdomains for spectrum data in $(r, l, m)$. Direction {\tt  modes} is set in the \verb|[Direction]| field, and number of subdomains \verb|[Ndomain]| are defined in the integer field.


\subsubsection{\tt num\_grid\_sph}
\label{href_t:num_grid_sph}
Spatial resolution of the spherical shell is defined in this block. \\
\hyperref[href_i:num_grid_sph]{(Back to {\tt control\_sph\_shell})}

\paragraph{\tt truncation\_level\_ctl}
\label{href_t:truncation_level_ctl}
\verb|[Lmax]| \\
Truncation level $L$ is defined by integer. Spherical harmonics is truncated by triangular $0 \le l \le L$ and $0 <m < l$.

\paragraph{\tt ngrid\_meridonal\_ctl}
\label{href_t:ngrid_meridonal_ctl}
\verb|[Ntheta]| \\
Number of grid in the meridional direction \verb|[Ntheta]| is defined by integer
\paragraph{\tt ngrid\_zonal\_ctl}
\label{href_t:ngrid_zonal_ctl}
\verb|[Nphi]| \\
Number of grid in the zonal direction \verb|[Nphi]| is defined by integer.

\paragraph{\tt raidal\_grid\_type\_ctl}
\label{href_t:radial_grid_type_ctl}
\verb|[explicit, Chebyshev, or equi_distance]| \\
Type of the radial grid spacing is defined by text. The following types are supported in Calypso.
%
\begin{description}
	\item{\tt explicit}  Equi-distance grid
	\item{\tt Chebyshev} Chebyshev collocation points
	\item{\tt equi\_distance} Set explicitly by \verb|r_layer| array
\end{description}
%

\paragraph{\tt num\_fluid\_grid\_ctl}
\label{href_t:num_fluid_grid_ctl}
\verb|[Nr_shell]| \\
(This option works with \verb|radial_grid_type_ctl| is {\tt explicit} or {\tt Chebyshev}.)
Number of layer in the fluid shell \verb|[Nr_shell]| is defined by integer. Number of grids including CMB and ICB will be (\verb|[Nr_shell]| + 1).

\paragraph{\tt fluid\_core\_size\_ctl}
\label{href_t:fluid_core_size_ctl}
\verb|[Length]| \\
(This option works with \verb|radial_grid_type_ctl| is {\tt explicit} or {\tt Chebyshev}.)
Size of the outer core \verb|[Length]| ($ = r_{o}-r_{i}$) is defined by real.

\paragraph{\tt ICB\_to\_CMB\_ratio\_ctl}
\label{href_t:ICB_to_CMB_ratio_ctl} 
\verb|[R_ratio]| \\
(This option works with \verb|radial_grid_type_ctl| is {\tt explicit} or {\tt Chebyshev}.)
Ratio of the inner core radius to outer core \verb|[R_ratio]| ($ = r_{i} / r_{o}$) is defined by real.

\paragraph{\tt Min\_radius\_ctl}
\label{href_t:Min_radius_ctl}
\verb|[Rmin]| \\
(This option works with \verb|radial_grid_type_ctl| is {\tt explicit} or {\tt Chebyshev}.)
Minimum radius of the domains \verb|[Rmin]| is defined by real. If this value is not defined, ICB becomes inner boundary of the domain.

\paragraph{\tt Max\_radius\_ctl }
\label{href_t:Max_radius_ctl} 
\verb|[Rmax]| \\
(This option works with \verb|radial_grid_type_ctl| is {\tt explicit} or {\tt Chebyshev}.)
Maximum radius of the domains \verb|[Rmax]| is defined by real. If this value is not defined, CMB becomes outer boundary of the domain.

\paragraph{\tt r\_layer}
\label{href_t:r_layer}
\verb|[Layer #]   [Radius]| \\
(This option works with \verb|[radial_grid_type_ctl]| is {\tt explicit}.)
List of the radial grid points in the simulation domain. Index of the radial point \verb|[Layer #]| is defined by integer, and radius \verb|[Radius]| is defined by real.

\paragraph{\tt array boundaries\_ctl}
\verb|[Boundary_name]  [Layer #]| \\
\label{href_t:boundaries_ctl} 
(This option works with \verb|[radial_grid_type_ctl]| is {\tt explicit}.)
Boundaries of the simulation domain is defined by \verb|[Layer #]| in \verb|[r_layer]| array. The following boundary name can be defined for \verb|[Boundary_name]|.
%
\begin{description}
	\item{\tt to\_Center} Inner boundary of the domain to fill the center.
	\item{\tt ICB} ICB
	\item{\tt CMB} CMB
\end{description}
%
%
%

\subsection{\tt phys\_values\_ctl}
\label{href_t:phys_values_ctl}
Fields for the simulation are defined in this block. \\
\hyperref[href_i:phys_values_ctl]{(Back to {\tt control\_MHD})}
%
\paragraph{\tt array nod\_value\_ctl}
\label{href_t:nod_value_ctl}
\verb|[Field] [Viz_flag] [Monitor_flag]| \\
Fields name \verb|[Field]| for the simulation are listed in this array. If required fields for simulation are not in the list, simulation program adds required field in the list, but does not output any field data and monitoring data. \verb|[Viz_flag]| is set to output of the field data for visualization by
%
\begin{description}
\item{\tt VIz\_On}  Write field data to VTK file
\item{\tt VIz\_Off} Do not write field data to VTK file.
\end{description}
%
In the \verb|[Monitor_flag]|, output in the monitoring data is defined by
%
\begin{description}
\item{\tt Monitor\_On}  Write spectrum into monitoring data
\item{\tt Monitor\_Off} Do not write spectrum into monitoring data
\end{description}
%
Supported field in the present version is listed in Table from \ref{table:fields} to \ref{table:energy_flux_sym} 
%
\begin{table}[htp]
\caption{List of field name}
\begin{center}
\begin{tabular}{|c|c|c|}
\hline
\tt [Name] & field name & Description \\ \hline \hline
\tt velocity &    Velocity &  \bvec{u} \\
\tt vorticity &   Vorticity & $\bvec{\omega} = \nabla \times \bvec{u} $ \\
\tt pressure &    Pressure & $P$ \\
\hline
\tt temperature & Temperature & $T$ \\
\tt perturbation\_temp & Perturbation of temperature
& $\Theta = T - T_{0}$ \\
\tt heat\_source & Heat source
& $q_{T}$ \\
\hline
\tt composition & Composition variation & $C$ \\
\tt composition\_source & Composition source & $q_{C}$ \\
\hline
\tt magnetic\_field &  Magnetic field  & $\bvec{B}$ \\
\tt current\_density & Current density & $\bvec{J} = \nabla \times \bvec{B} $ \\
\tt electric\_field & Electric field & $\bvec{E} = \sigma \left(\bvec{J} - \bvec{u} \times \bvec{B}\right) $ \\
\tt truncated\_magnetic\_field & Truncated Magnetic field at $Lt$ & $\sum_{l=1}^{Lt} \bvec{B}_{l}^{m}$ \\
 &  See \hyperref[href_t:crustal_filtering_ctl]{\tt truncation\_degree\_ctl} &  \\
\hline \hline
\tt viscous\_diffusion & Viscous diffusion
& $-\nu \nabla \times \nabla \times \bvec{u}$ \\
\tt inertia & Inertia term &  $ \bvec{\omega} \times \bvec{u} $ \\
\tt buoyancy                   & Thermal buoyancy &  $ -\alpha_{T} T \bvec{g}  $ \\
\tt composite\_buoyancy & Compositional buoyancy &  $ -\alpha_{C} C\bvec{g}  $\\
\tt Lorentz\_force & Lorentz force &  $ \bvec{J} \times \bvec{B} $ \\
\tt Coriolis\_force & Coriolis force &  $ -2 \Omega \hat{z} \times \bvec{u} $ \\
\tt pressure\_gradient & Pressure gradient &  $  - \nabla P  $ \\
\tt rest\_of\_geostrophic & Rest of geostrophic balance &  $ - \nabla P - 2 \Omega \hat{z} \times \bvec{u} $ \\
\hline
\tt thermal\_diffusion & Termal diffusion & $ \kappa_{T} \nabla^{2} T $ \\
\tt grad\_temp & Temperature gradient & $ \nabla T$ \\
\tt heat\_flux & Advective heat flux & $ \bvec{u} T$ \\
\tt heat\_advect & Heat advection & $ \bvec{u} \cdot \nabla T = \nabla \cdot \left(  \bvec{u} T \right) $ \\
\hline
\tt composition\_diffusion & Compositional diffusion & $ \kappa_{C} \nabla^{2} C $ \\
\tt grad\_composition & Composition gradient & $ \nabla C$ \\
\tt composite\_flux & Advective composition flux & $ \bvec{u} C$ \\
\tt composition\_advect & Compositional advection & $ \bvec{u} \cdot \nabla C = \nabla \cdot \left(  \bvec{u} C \right) $ \\
\hline
\tt magnetic\_diffusion & Magnetic diffusion
& $-\eta \nabla \times \nabla \times \bvec{B}$ \\
\tt vecp\_induction & Induction for the vector potential &  $ \bvec{u} \times \bvec{B} $ \\
\tt magnetic\_induction & Magnetic induction &  $ \nabla \times \left(\bvec{u} \times \bvec{B}\right) $ \\
\tt poynting\_flux & Poynting flux &  $ \bvec{E} \times \bvec{B} $ \\
\hline
\end{tabular}
\end{center}
\label{table:fields}
\end{table}
%
%
\begin{table}[htp]
\caption{List of field name (Continued)}
\begin{center}
\begin{tabular}{|c|c|c|}
\hline
\tt [Name] & field name & Description \\ \hline \hline
\tt rot\_inertia & Curl of inertia &  $ \nabla \times \left(\bvec{\omega} \times \bvec{u}\right) $ \\
\tt rot\_Lorentz\_force & Curl of Lorentz force &  $ \nabla \times \left(\bvec{J} \times \bvec{B}\right) $ \\
\tt rot\_Coriolis\_force & Curl of Coriolis force &  $ -2 \Omega \nabla \times \left(\hat{z} \times \bvec{u} \right) $ \\
\tt rot\_buoyancy                   & Curl of thermal buoyancy &  $ - \nabla \times \left(\alpha_{T} T \bvec{g}\right)  $ \\
\tt rot\_composite\_buoyancy & Curl of compositional buoyancy &  $ - \nabla \times \left(\alpha_{C} C\bvec{g}\right)  $\\
\hline
\tt Lorentz\_work & Work of Lorentz force
 & $\bvec{u}\cdot \left( \bvec{J} \times \bvec{B} \right) $ \\ 
\tt work\_against\_Lorentz & Work against Lorentz force
 & $- \bvec{u}\cdot \left( \bvec{J} \times \bvec{B} \right) $ \\ 
\tt buoyancy\_flux & Thermal buoyancy flux & $ -\alpha_{T} T \bvec{g} \cdot \bvec{u} $ \\
\tt composite\_buoyancy\_flux & Compositional buoyancy flux & $ -\alpha_{c} C \bvec{g} \cdot \bvec{u} $ \\
\tt magnetic\_ene\_generation & Energy production by magnetic induction
 & $  \bvec{B} \cdot \left( \bvec{u} \times \bvec{B} \right) $ \\ 
\hline
\tt kinetic\_helicity & Kinetic helicity & $  \bvec{u} \cdot \bvec{\omega} $ \\
\tt current\_helicity & Current helicity & $  \bvec{B} \cdot \bvec{J} $ \\
\tt cross\_helicity & Cross helicity$^{*}$ & $  \bvec{u} \cdot \bvec{B} $ \\
\hline
\end{tabular}
\end{center}
*: Magnetic helicity $\bvec{A} \cdot \bvec{B} $ is not implimented.
\label{table:fields2}
\end{table}
%
\begin{table}[htp]
\caption{List of field names decomposed by equatorial symmetry}
\begin{center}
\begin{tabular}{|cc|cc|}
\hline
\multicolumn{2}{|c|}{Symmetric} & \multicolumn{2}{c|}{Anti-symmetric} \\
 {\tt [Name]}  & expression  & {\tt [Name]}  &  expression \\ \hline \hline
\tt sym\_velocity &  $\bvec{u}_{sym}$ &  \tt asym\_velocity &  $\bvec{u}_{asym}$  \\
\tt sym\_vorticity & $\bvec{\omega}_{sym} = \nabla \times \bvec{u}_{asym} $
     & \tt asym\_vorticity & $\bvec{\omega}_{asym} = \nabla \times \bvec{u}_{sym} $ \\
\tt sym\_pressure  & $P_{sym}$  & \tt asym\_pressure  & $P_{a}$ \\ \hline
\tt sym\_temperature & $T_{sym}$ & \tt asym\_temperature  & $T_{asym}$ \\
\tt sym\_composition & $C_{sym}$ & \tt asym\_composition & $C_{asym}$ \\ \hline
\tt sym\_magnetic\_field & $\bvec{B}_{sym}$ & \tt asym\_magnetic\_field  & $\bvec{B}_{asym}$ \\
\tt sym\_current\_density & $\bvec{J}_{sym} = \nabla \times \bvec{B}_{asym} $
     & \tt asym\_current\_density  & $\bvec{J}_{asym} = \nabla \times \bvec{B}_{sym} $ \\ \hline
\end{tabular}
\end{center}
\label{table:fields_sym}
\end{table}
%
%
\begin{table}[htp]
\caption{List of force and nonlinear term names decomposed by equatorial symmetry}
\begin{center}
\begin{tabular}{|c|c|}
\hline
{\tt [Name]} & expression  \\ \hline
\tt sym\_thermal\_buoyancy &  $ -\alpha_{T} T_{sym} \bvec{g}  $ \\
\tt asym\_thermal\_buoyancy &  $ -\alpha_{T} T_{asym} \bvec{g}  $ \\ \hline
\tt sym\_composite\_buoyancy &  $ -\alpha_{C} C_{sym} \bvec{g}  $ \\
\tt asym\_composite\_buoyancy &  $ -\alpha_{C} C_{asym} \bvec{g}  $ \\ \hline
%
\tt wsym\_x\_usym  &  $ \bvec{\omega}_{sym} \times \bvec{u}_{sym} $ \\
\tt wasym\_x\_uasym  &  $ \bvec{\omega}_{asym} \times \bvec{u}_{asym} $ \\
\tt wsym\_x\_uasym  &  $ \bvec{\omega}_{sym} \times \bvec{u}_{asym} $ \\
\tt wasym\_x\_usym  &  $ \bvec{\omega}_{asym} \times \bvec{u}_{sym} $ \\ \hline
%
\tt Jsym\_x\_Bsym  &  $ \bvec{J}_{sym} \times \bvec{B}_{sym} $ \\
\tt Jasym\_x\_Basym  &  $ \bvec{J}_{asym} \times \bvec{B}_{asym} $ \\
\tt Jsym\_x\_Basym  &  $ \bvec{J}_{sym} \times \bvec{B}_{asym} $ \\
\tt Jasym\_x\_Bsym  &  $ \bvec{J}_{asym} \times \bvec{B}_{sym} $ \\ \hline
%
\tt usym\_x\_Bsym  &  $ \bvec{u}_{sym} \times \bvec{B}_{sym} $ \\
\tt uasym\_x\_Basym  &  $ \bvec{u}_{asym} \times \bvec{B}_{asym} $ \\
\tt usym\_x\_Basym  &  $ \bvec{u}_{sym} \times \bvec{B}_{asym} $ \\
\tt uasym\_x\_Bsym  &  $ \bvec{u}_{asym} \times \bvec{B}_{sym} $ \\ \hline
\end{tabular}
\end{center}
\label{table:fields_sym2}
\end{table}
%
%
\begin{table}[htp]
\caption{List of energy flux names decomposed by equatorial symmetry}
\begin{center}
\begin{tabular}{|c|c|}
\hline
{\tt [Name]} & expression  \\ \hline
\tt sym\_buoyancy\_flux &  $ - \bvec{u}_{sym}  \cdot  \alpha_{T} T_{sym} \bvec{g}  $ \\
\tt asym\_buoyancy\_flux  &  $ - \bvec{u}_{asym}  \cdot  \alpha_{T} T_{asym} \bvec{g}  $ \\ \hline
% \tt sym\_composite\_buoyancy &  $ -\alpha_{C}  \cdot   C_{sym} \bvec{g}  $ \\
% \tt asym\_composite\_buoyancy &  $ -\alpha_{C}  \cdot   C_{asym} \bvec{g}  $ \\ \hline
%
\tt -ua\_d\_ws\_x\_us &  $  \bvec{u}_{asym} \cdot \left( \bvec{\omega}_{sym} \times \bvec{u}_{sym} \right)$ \\
\tt -ua\_d\_wa\_x\_ua  &  $ \bvec{u}_{asym} \cdot  \left( \bvec{\omega}_{asym} \times \bvec{u}_{asym} \right)$ \\
\tt -us\_d\_ws\_x\_ua  &  $ \bvec{u}_{sym} \cdot  \left( \bvec{\omega}_{sym} \times \bvec{u}_{asym} \right)$ \\
\tt -us\_d\_wa\_x\_us &  $ \bvec{u}_{sym} \cdot  \left( \bvec{\omega}_{asym} \times \bvec{u}_{sym} \right)$ \\ \hline
%
\tt ua\_d\_js\_x\_bs  &  $ \bvec{u}_{asym} \cdot \left(\bvec{J}_{sym} \times \bvec{B}_{sym} \right) $ \\
\tt ua\_d\_ja\_x\_ba  &  $ \bvec{u}_{asym}  \cdot  \left(\bvec{J}_{asym} \times \bvec{B}_{asym} \right) $ \\
\tt us\_d\_js\_x\_ba  &  $ \bvec{u}_{sym}  \cdot  \left(\bvec{J}_{sym} \times \bvec{B}_{asym} \right) $ \\
\tt us\_d\_ja\_x\_bs  &  $ \bvec{u}_{sym}  \cdot  \left(\bvec{J}_{asym} \times \bvec{B}_{sym} \right) $ \\ \hline
%
% \tt usym\_x\_Bsym  &  $ \bvec{u}_{sym} \times \bvec{B}_{sym} $ \\
% \tt uasym\_x\_Basym  &  $ \bvec{u}_{asym} \times \bvec{B}_{asym} $ \\
% \tt usym\_x\_Basym  &  $ \bvec{u}_{sym} \times \bvec{B}_{asym} $ \\
% \tt uasym\_x\_Bsym  &  $ \bvec{u}_{asym} \times \bvec{B}_{sym} $ \\ \hline
\end{tabular}
\end{center}
\label{table:energy_flux_sym}
\end{table}
%
%

\subsection{\tt time\_evolution\_ctl}
\label{href_t:time_evolution_ctl}
Fields for time evolution are defined in this block. \\
\hyperref[href_i:time_evolution_ctl]{(Back to {\tt control\_MHD})}

\paragraph{\tt array time\_evo\_ctl}
\label{href_t:time_evo_ctl}
\verb|[Field]| \\
Fields name for time evolution are listed in this array in \verb|[Field]| by text.
Available fields are listed in Table \ref{table:evolution_field}.
%
\begin{table}[htp]
\caption{List of field name for time evolution}
\begin{center}
\begin{tabular}{|c|c|c|}
\hline
 label & field name & Description \\ \hline
\verb|velocity| &    Velocity &  \bvec{u} \\
\verb|temperature| & Temperature & $T$ \\
\verb|composition| & Composition variation & $C$ \\
\verb|magnetic_field| &  Magnetic field  & $\bvec{B}$ \\ \hline
\end{tabular}
\end{center}
\label{table:evolution_field}
\end{table}

\subsection{\tt boundary\_condition}
\label{href_t:boundary_condition}
Boundary condition are defined in this block. \\
\hyperref[href_i:boundary_condition]{(Back to {\tt control\_MHD})}

\paragraph{\tt array bc\_temperature}
\label{href_t:bc_temperature}
\verb|[Group]  [Type]  [Value]| \\
Boundary conditions for temperature are defined by this array. Position of boundary is defined in \verb|[Group]| column by {\tt ICB} or {\tt CMB}. The following type of boundary conditions are available for temperature in \verb|[Type]| column.
%
\begin{description}
\item{\tt fixed}			Fixed homogeneous temperature on the boundary. The fixed value is defined in \verb|[Value]| by real.
\item{\tt fixed\_file}			Fixed temperature defined by external file. \verb|[Value]| in this line is ignored. See section \ref{sec:boundary_file}.
\item{\tt fixed\_flux}	Fixed homogeneous heat flux on the boundary. The value is defined in \verb|[Value]| by real. Positive value indicates outward flux from fluid shell. ({\it e.g.} Flux to center at ICB and Flux to mantle at CMB are positive.)
\item{\tt fixed\_flux\_file}			Fixed heat flux defined by external file. \verb|[Value]| in this line is ignored.  See section \ref{sec:boundary_file}.
\end{description}
%

\paragraph{\tt array bc\_velocity}
\label{href_t:bc_velocity}
\verb|[Group]  [Type]  [Value]| \\
Boundary conditions for velocity are defined by this array. Position of boundary is defined in \verb|[Group]| by {\tt ICB} or {\tt CMB}. The following boundary conditions are available for velocity in \verb|[Type]| column.
%
\begin{description}
\item{\tt non\_slip\_sph}	Non-slip boundary is applied to the boundary defined in \verb|[Group]|. Real value is required in \verb|[Value]|, but they value is not used in the program.
\item{\tt free\_slip\_sph}	Free-slip boundary is applied to the boundary defined in \verb|[Group]|. Real value is required in \verb|[Value]|, but they value is not used in the program.
\item{\tt rot\_inner\_core} If this condition is set, inner core ($r < r_{i}$) rotation is solved by using viscous torque and Lorentz torque. This boundary condition can be used for {\tt ICB}, and grid is filled to center. Real value is required in \verb|[Value]|, but they value is not used in the program.

\item{\tt rot\_x} Set constant rotation around $x$-axis in \verb|[Value]| by real. Rotation vector can be defined with {\tt rot\_y} and {\tt rot\_z}.
\item{\tt rot\_y} Set constant rotation around $y$-axis in \verb|[Value]| by real. Rotation vector can be defined with {\tt rot\_z} and {\tt rot\_x}.
\item{\tt rot\_z} Set constant rotation around $z$-axis in \verb|[Value]| by real. Rotation vector can be defined with {\tt rot\_x} and {\tt rot\_y}.
\end{description}
%

\paragraph{\tt array bc\_magnetic\_field}
\label{href_t:bc_magnetic_field}
\verb|[Group]  [Type]  [Value]| \\
Boundary conditions for magnetic field are defined by this array. Position of boundary is defined in \verb|[Group]| by {\tt to\_Center}, {\tt ICB}, or {\tt CMB}. The following boundary conditions are available for magnetic field in \verb|[Type]| column.
%
\begin{description}
\item{\tt insulator}	Magnetic field is connected to potential field at boundary defined in [Group]. real value is required at \verb|[Value]|, but they value is not used in the program.
\item{\tt sph\_to\_center}	 If this condition is set, magnetic field in conductive inner core ($r < r_{i}$) is solved. This boundary condition can be used for {\tt ICB}, and grid is filled to center. The value at \verb|[Value]| does not used.
\end{description}
%

\paragraph{\tt array bc\_composition}
\label{href_t:bc_composition}
\verb|[Group]  [Type]  [Value]| \\
Boundary conditions for composition variation are defined by this array. Position of boundary is defined in \verb|[Group]| by {\tt ICB} or {\tt CMB}. The following boundary conditions are available for composition variation in \verb|[Type]| column.
%
\begin{description}
\item{\tt fixed}			Fixed homogeneous composition on the boundary. The fixed value is defined in \verb|[Value]| by real.
\item{\tt fixed\_file}			Fixed composition defined by external file. \verb|[Value]| in this line is ignored. See section \ref{sec:boundary_file}.
\item{\tt fixed\_flux}	Fixed homogeneous compositional flux on the boundary. The value is defined in \verb|[Value]| by real. Positive value indicates outward flux from fluid shell. ({\it e.g.} Flux to center at ICB and Flux to mantle at CMB are positive.)
\item{\tt fixed\_flux\_file}			Fixed compositional flux defined by external file. \verb|[Value]| in this line is ignored. See section \ref{sec:boundary_file}.
\end{description}
%

\subsection{\tt forces\_define}
\label{href_t:forces_define}
Forces for the momentum equation are defined in this block. \\
\hyperref[href_i:forces_define]{(Back to {\tt control\_MHD})}

\paragraph{\tt array force\_ctl}
\label{href_t:force_ctl}
\verb|[Force]| \\
Name of forces for momentum equation are listed in \verb|[Force]| by text.
The following fields are available.
%
\begin{table}[htp]
\caption{List of force}
\begin{center}
\begin{tabular}{|c|c|c|}
\hline
 Label & Field name & Equation \\ \hline
\verb|Coriolis| & Coriolis force & $-2\Omega \hat{z} \times \bvec{u} $ \\
\verb|Lorentz| & Lorentz force &  $\bvec{J} \times \bvec{B} $ \\
\verb|gravity| & Thermal buoyancy & $-\alpha_{T} T \bvec{g}$ \\
\verb|Composite_gravity| & Compositional buoyancy  & $-\alpha_{C} C \bvec{g}$\\ \hline
\end{tabular}
\end{center}
\label{table:forces}
\end{table}
%

\subsection{\tt dimensionless\_ctl}
\label{href_t:dimensionless_ctl}
Dimensionless numbers are defined in this block. \\
\hyperref[href_i:dimensionless_ctl]{(Back to {\tt control\_MHD})}

\paragraph{\tt array dimless\_ctl}
\label{href_t:dimless_ctl}
\verb|[Name] [Value]| \\
Dimensionless are listed in this array. The name is defined in \verb|[Name]| by text, and value is defined in \verb|[Value]| by real. These name of the dimensionless numbers are used to construct coefficients for each terms in governing equations. The following names can not be used because of reserved name in the program.
%
\begin{table}[htp]
\caption{List of reserved name of dimensionless numbers}
\begin{center}
\begin{tabular}{|c|c|c|}
\hline
 label & field name & value \\ \hline
\verb|Zero| & zero & 0.0 \\
\verb|One| &  one &  1.0 \\
\verb|Two| &  two &  2.0 \\
\verb|Radial_35| & Ratio of outer core thickness to whole core & $0.65 = 1 - 0.35$ \\ \hline
\end{tabular}
\end{center}
\label{table:reserved_params}
\end{table}
%

\subsection{\tt coefficients\_ctl}
\label{href_t:coefficients_ctl}
Coefficients of each term in governing equations are defined in this block.
Each coefficients are defined by list of name of dimensionless number \verb|[Name]| and its power \verb|[Power]|. For example, coefficient for Coriolis term for the dynamo benchmark $ 2E^{-1}$ is defined as
%
\begin{verbatim}
        array coef_4_Coriolis_ctl   2
          coef_4_Coriolis_ctl       Two            1.0
          coef_4_Coriolis_ctl       Ekman_number  -1.0
        end array coef_4_Coriolis_ctl
\end{verbatim}
%
\hyperref[href_i:coefficients_ctl]{(Back to {\tt control\_MHD})}

\subsubsection{\tt thermal}
\label{href_t:thermal}
Coefficients of each term in heat equation are defined in this block. \\
\hyperref[href_i:thermal]{(Back to {\tt control\_MHD})}

\paragraph{\tt coef\_4\_termal\_ctl}
\label{href_t:coef_4_termal_ctl}
\verb|[Name] [Power]| \\
Coefficient for evolution of temperature $\displaystyle \frac{\partial T}{\partial t}$ and advection of heat $\left(\bvec{u} \cdot \nabla \right) T$ is defined by this array.

\paragraph{\tt coef\_4\_t\_diffuse\_ctl}
\label{href_t:coef_4_t_diffuse_ctl}
\verb|[Name] [Power]| \\
Coefficient for thermal diffusion $\displaystyle \kappa_{T} \nabla^{2} T$ is defined by this array.

\paragraph{\tt coef\_4\_heat\_source\_ctll}
\label{href_t:coef_4_heat_source_ctl}
\verb|[Name] [Power]| \\
Coefficient for heat source $\displaystyle q_{T}$ is defined by this array.

\subsubsection{\tt momentum}
\label{href_t:momentum}
Coefficients of each term in momentum equation are defined in this block. \\
\hyperref[href_i:momentum]{(Back to {\tt control\_MHD})}

\paragraph{\tt coef\_4\_velocity\_ctl}
\label{href_t:coef_4_velocity_ctl}
\verb|[Name] [Power]| \\
Coefficient for evolution of velocity $\displaystyle \frac{\partial \bvec{u}}{\partial t}$ (or $\displaystyle \frac{\partial \bvec{\omega}}{\partial t}$ for the vorticity equation) and advection $-\bvec{\omega} \times \bvec{u}$ (or $- \nabla \times \left(\bvec{\omega} \times \bvec{u} \right)$ for the vorticity equation) is defined by this array.

\paragraph{\tt coef\_4\_press\_ctl}
\label{href_t:coef_4_press_ctl}
\verb|[Name] [Power]| \\
Coefficient for pressure gradient $-\nabla P$ is defined by this array. Pressure does not appear the vorticity equation which is used for the time integration. But this coefficient is used to evaluate pressure field.

\paragraph{\tt coef\_4\_v\_diffuse\_ctl}
\label{href_t:coef_4_v_diffuse_ctl}
\verb|[Name] [Power]| \\
Coefficient for viscous diffusion $- \nu \nabla \times \nabla \times \bvec{u}$ is defined by this array.

\paragraph{\tt coef\_4\_buoyancy\_ctl}
\label{href_t:coef_4_buoyancy_ctl}
\verb|[Name] [Power]| \\
Coefficient for buoyancy $- \alpha_{T} T \bvec{g}$ is defined by this array.

\paragraph{\tt coef\_4\_Coriolis\_ctl}
\label{href_t:coef_4_Coriolis_ctl}
\verb|[Name] [Power]| \\
Coefficient for Coriolis force $-2 \Omega \hat{z} \times \bvec{u}$ is defined by this array.

\paragraph{\tt coef\_4\_Lorentz\_ctl}
\label{href_t:coef_4_Lorentz_ctl}
\verb|[Name] [Power]| \\
Coefficient for Lorentz force $ \rho_{0}^{-1} \bvec{J} \times \bvec{B}$ is defined by this array.

\paragraph{\tt coef\_4\_composit\_buoyancy\_ctl}
\label{href_t:coef_4_composit_buoyancy_ctl}
\verb|[Name] [Power]| \\
Coefficient for compositional buoyancy $ -\alpha_{C} C \bvec{g}$ is defined by this array.

\subsubsection{\tt induction}
\label{href_t:induction}
Coefficients of each term in magnetic induction equation are defined in this block. \\
\hyperref[href_i:induction]{(Back to {\tt control\_MHD})}

\paragraph{\tt coef\_4\_magnetic\_ctl}
\label{href_t:coef_4_magnetic_ctl}
\verb|[Name] [Power]| \\
Coefficient for evolution of temperature $\displaystyle \frac{\partial \bvec{B}}{\partial t}$ is defined by this array.

\paragraph{\tt coef\_4\_m\_diffuse\_ctl}
\label{href_t:coef_4_m_diffuse_ctl}
\verb|[Name] [Power]| \\
Coefficient for magnetic diffusion $ -\eta \nabla \times \nabla \times \bvec{B}$ is defined by this array.

\paragraph{\tt coef\_4\_induction\_ctl}
\label{href_t:coef_4_induction_ctl}
\verb|[Name] [Power]| \\
Coefficient for magnetic induction $\nabla \times \left(\bvec{u} \times \bvec{B} \right)$ is defined by this array.

\subsubsection{\tt composition}
\label{href_t:composition}
Coefficients of each term in composition equation are defined in this block. \\
\hyperref[href_i:composition]{(Back to {\tt control\_MHD})}

\paragraph{\tt coef\_4\_composition\_ctl}
\label{href_t:coef_4_composition_ctl}
\verb|[Name] [Power]| \\
Coefficient for evolution of composition variation $\displaystyle \frac{\partial C}{\partial t}$ and advection of heat $\left(\bvec{u} \cdot \nabla \right) C$ is defined by this array.

\paragraph{\tt coef\_4\_c\_diffuse\_ctl}
\label{href_t:coef_4_c_diffuse_ctl}
\verb|[Name] [Power]| \\
Coefficient for compositional diffusion $\displaystyle \kappa_{C} \nabla^{2} C$ is defined by this array.

\paragraph{\tt coef\_4\_composition\_source\_ctll}
\label{href_t:coef_4_composition_source_ctl}
\verb|[Name] [Power]| \\
Coefficient for composition source $\displaystyle q_{C}$ is defined by this array.

% \subsection{\tt gravity\_define}
% \label{href_t:gravity_define}
% Gravity (buoyancy) vector is defined in this block \\
% \hyperref[href_i:gravity_define]{(Back to {\tt control\_MHD})} 
%
% \paragraph{\tt gravity\_type\_ctl}
% \label{href_t:gravity_type_ctl}
% \verb|[Direction]  [Value]| \\
% Gravity (buoyancy) type is defined by text. The following setting is available.
% \begin{description}
% \item{\tt radial} Gravity vector goes to center and is proportional to the radius $\bvec{g} = -\bvec{r}$. This model generally used to the geodyanmo simulations. 
% \item{\tt constant\_radial} Gravity vector goes to center and has constants amplitude  $\bvec{g} = -\bvec{r} / r$. This model generally used to the geodyanmo simulations. 
% \end{description}
%
% \subsection{\tt Coriolis\_define}
% \label{href_t:Coriolis_define}
% Rotation of the system for Coriolis force is defined in this block. \\
% \hyperref[href_i:Coriolis_define]{(Back to {\tt control\_MHD})}
%
% \paragraph{\tt array rotation\_vec}
% \label{href_t:rotation_vec}
% \verb|[Direction]  [Value]| \\
% Rotation vector of the system is defined by array. {\tt x}, {\tt y}, or {\tt z} is set in \verb|[Direction]|, and each component of the rotation vector is set in the \verb|[Value]| as real. In this program, the rotation vector does NOT normalized.

\subsection{\tt temperature\_define}
\label{href_t:temperature_define}
Reference of temperature $T_{0}$ is defined in this block. If reference of temperature is defined, perturbation of temperature $\Theta = T - T_{0}$ is used for time evolution and buoyancy. \\
\hyperref[href_i:temperature_define]{(Back to {\tt control\_MHD})}

\paragraph{\tt ref\_temp\_ctl}
\label{href_t:ref_temp_ctl}
\verb|[REFERENCE_TYPE]| \\
Type of reference temperature is defined by text. The following options are available for \verb|[REFERENCE_TYPE]|.
%
\begin{description}
\item{\tt none: }   \\
Reference of temperature is not defined. Temperature $T$ is used to time evolution and thermal buoyancy.
\item{\tt file: }      \\
Read reference temperature from data file defined by \hyperref[href_t:ref_field_file_name]{\tt [ref\_field\_file\_name]}
\item{\tt numerical\_solution: } \\
Reference temperature obtained from the given boundary condition and radial heat source by
\begin{eqnarray}
 0 = \nabla^{2} T_{0} + q_{0}
 \nonumber
\end{eqnarray}
%

\item{\tt spherical\_shell:} Reference of temperature is set by
\begin{eqnarray}
 T_{0} = \frac{1}{\left(r_{h}-r_{l} \right)} \left[
          r_{l}T_{l} - r_{h}T_{h} + \frac{r_{l} r_{h}}{r} \left(T_{h}-T_{l}\right) \right].
\nonumber
\end{eqnarray}
\item{\tt file} Reference temperature data is read from file defeind at {\tt [radial\_field\_file\_name]}. The reference temperature is interpolated into the radial grid in the simulation.
\end{description}
%
\paragraph{\tt ref\_field\_file\_name [File\_Name]}
\label{href_t:ref_field_file_name}
 [File\_Name] for the reference temperature is defined as text. 

\paragraph{\tt low\_temp\_ctl}
\label{href_t:low_temp_ctl}
Amplitude of low reference temperature $T_{l}$ and its radius $r_{l}$ (Generally $r_{l} = r_{o}$) are defined in this block.

\paragraph{\tt high\_temp\_ctl}
\label{href_t:high_temp_ctl}
Amplitude of high reference temperature $T_{h}$ and its radius $r_{h}$ (Generally $r_{h} = r_{i}$) are defined in this block.

\paragraph{\tt depth}
\label{href_t:depth}
\verb|[RADIUS]| \\
Radius for reference temperature is defined by real.

\paragraph{\tt temperature}
\label{href_t:temperature}
\verb|[TEMPERATURE]| \\
Temperature for reference temperature is defined by real.

\subsection{\tt composition\_define}
\label{href_t:composition_define}
Reference of light element $C_{0}$ is defined in this block. If reference of light element amount is defined, perturbation of temperature $\Theta_{C} = C - C_{0}$ is used for time evolution and buoyancy. \\
\hyperref[href_i:composition_define]{(Back to {\tt control\_MHD})}

\paragraph{\tt ref\_comp\_ctl}
\label{href_t:ref_comp_ctl}
\verb|[REFERENCE_TYPE]| \\
Type of reference temperature is defined by text. See  \hyperref[href_t:ref_temp_ctl]{\tt [ref\_temp\_ctl]} for the available options for \verb|[REFERENCE_TYPE]| .
%
\\paragraph{\tt low\_comp\_ctl}
\label{href_t:low_comp_ctl}
Amplitude of low reference composition $C_{l}$ and its radius $r_{l}$ (Generally $r_{l} = r_{o}$) are defined in this block.

\paragraph{\tt high\_comp\_ctl}
\label{href_t:high_comp_ctl}
Amplitude of high reference composition $C_{h}$ and its radius $r_{h}$ (Generally $r_{h} = r_{i}$) are defined in this block.

\paragraph{\tt composition}
\label{href_t:composition}
\verb|[COMPOSITION]| \\
Composition for the reference composition is defined by real.


\subsection{\tt time\_step\_ctl}
\label{href_t:time_step_ctl}
Time stepping parameters are defined in this block. \\
\hyperref[href_i:time_step_ctl]{(Back to {\tt control\_MHD})} \\
\hyperref[href_i:time_step_ctl2]{(Back to {\tt control\_assemble\_sph)}}

\paragraph{\tt elapsed\_time\_ctl}
\label{href_t:elapsed_time_ctl}
\verb|[ELAPSED_TIME]| \\
Elapsed (wall clock) time (second) for simulation \verb|[ELAPSED_TIME]| is defined by real. 
This parameter varies if end step \verb|[ISTEP_FINISH]| is defined to {\tt -1}. If simulation runs for given time, program output spectrum data  \verb|[rst_prefix].elaps.[process #].fst| immediately, and finish the simulation.

\paragraph{\tt i\_step\_init\_ctl}
\label{href_t:i_step_init_ctl}
\verb|[ISTEP_START]| \\
Start step of simulation \verb|[ISTEP_START]| is defined by integer. if \verb|[ISTEP_START]| is set to {\tt -1} and \verb|[INITIAL_TYPE]| is set to \verb|start_from_rst_file|, program read spectrum data file \verb|[rst_prefix].elaps.[process #].fst| and start the simulation.

\paragraph{\tt i\_step\_finish\_ctl}
\label{href_t:i_step_finish_ctl}
\verb|[ISTEP_FINISH]| \\
End step of simulation \verb|[ISTEP_FINISH]| is defined by integer. If this value is set to  {\tt -1}, simulation stops when elapsed time reaches to \verb|[ELAPSED_TIME]|.

\paragraph{\tt i\_step\_check\_ctl}
\label{href_t:i_step_check_ctl}
\verb|[ISTEP_MONITOR]| \\
Increment of time step for monitoring data \verb|[ISTEP_MONITOR]| is defined by integer.

\paragraph{\tt i\_step\_rst\_ctl}
\label{href_t:i_step_rst_ctl}
\verb|[ISTEP_RESTART]| \\
Increment of time step to output spectrum data for restarting \verb|[ISTEP_RESTART]| is defined by integer.

\paragraph{\tt i\_step\_field\_ctl}
\label{href_t:i_step_field_ctl}
\verb|[ISTEP_FIELD]| \\
Increment of time step to output field data for visualization \verb|[ISTEP_FIELD]| is defined by integer. If \verb|[ISTEP_FIELD]| is set to be 0, no field data are written.

\paragraph{\tt i\_step\_sectioning\_ctl}
\label{href_t:i_step_sectioning_ctl}
\verb|[ISTEP_SECTION]| \\
Increment of time step to output cross section data for visualization \verb|[ISTEP_SECTION]| is defined by integer. If \verb|[ISTEP_SECTION]| is set to be 0, no cross section data are written. If \verb|[ISTEP_SECTION]| is set in the block \hyperref[href_i:visual_control]{\tt visual\_control}, The value in {\tt visual\_control} is used.

\paragraph{\tt i\_step\_isosurface\_ctl}
\label{href_t:i_step_isosurface_ctl}
\verb|[ISTEP_ISOSURFACE]| \\
Increment of time step to output isosurface data for visualization \verb|[ISTEP_ISOSURFACE]| is defined by integer. If \verb|[ISTEP_ISOSURFACE]| is set to be 0, no isosurface data are written.If \verb|[ISTEP_ISOSURFACE]| is set in the block \hyperref[href_i:visual_control]{\tt visual\_control}, The value in {\tt visual\_control} is used.

\paragraph{\tt dt\_ctl}
\label{href_t:dt_ctl}
\verb|[DELTA_TIME]| \\
Length of time step $\Delta t$ is defined by real value.

\paragraph{\tt time\_init\_ctl}
\label{href_t:time_init_ctl}
\verb|[INITIAL_TIME]| \\
Initial time $t_{0}$ is defined by real value. This value is ignored if simulation starts from restart data.

\subsection{\tt new\_time\_step\_ctl}
\label{href_t:new_time_step_ctl}
Time stepping parameters to update initial data are defined in this block. Items in this block is the same as \hyperref[href_t:i_step_field_ctl]{\tt i\_step\_field\_ctl}.
\hyperref[href_i:new_time_step_ctl]{(Back to {\tt control\_assemble\_sph)}}


\subsection{\tt restart\_file\_ctl}
\label{href_t:restart_file_ctl}
Initial field for simulation is defined in this block.\\
\hyperref[href_t:restart_file_prefix]{(Back to {\tt control\_MHD})}

\paragraph{\tt rst\_ctl}
\label{href_t:rst_ctl}
\verb|[INITIAL_TYPE]| \\
Type of Initial field is defined by text. The following parameters are available for \verb|[INITIAL_TYPE]|.
%
\begin{description}
\item{\tt No\_data}  No initial data file. Small temperature perturbation and seed magnetic field are set as an initial field.
\item{\tt start\_from\_rst\_file} Initial field is read from spectrum data file. File prefix is defined by \hyperref[href_t:restart_file_prefix]{$\mbox{\tt restart\_file\_prefix}$}.
\item{\tt Dynamo\_benchmark\_0}   Generate initial field for dynamo benchmark case 0
\item{\tt Dynamo\_benchmark\_1}   Generate initial field for dynamo benchmark case 1
\item{\tt Dynamo\_benchmark\_2}   Generate initial field for dynamo benchmark case 2
\item{\tt Pseudo\_vacuum\_benchmark} Generate initial field for pseudo vacuum dynamo benchmark
\end{description}
%

\subsection{\tt time\_loop\_ctl}
\label{href_t:time_loop_ctl}
Time evolution scheme is defined in this block. \\
\hyperref[href_i:time_loop_ctl]{(Back to {\tt control\_MHD})}

\paragraph{\tt scheme\_ctl}
\label{href_t:scheme_ctl}
\verb|[EVOLUTION_SCHEME]| \\
Time evolution scheme is defined by text. Currently, Crank-Nicolson scheme is only available for diffusion terms.
%
\begin{description}
\item{\tt Crank\_Nicolson} Crank-Nicolson scheme for diffusion terms and second order Adams-Bashforth scheme the other terms.
% \item{\tt 2nd\_Adams\_Bashforth}  Second order Adams-Bashforth scheme
% \item{\tt explicit\_Euler} First order Euler scheme.
\end{description}
%

\paragraph{\tt coef\_imp\_v\_ctl}
\label{href_t:coef_imp_v_ctl}
\verb|[COEF_INP_U]| \\
Coefficients for the implicit parts of the Crank-Nicolson scheme for viscous diffusion \verb|[COEF_INP_U]| is defined by real.

\paragraph{\tt coef\_imp\_t\_ctl}
\label{href_t:coef_imp_t_ctl}
\verb|[COEF_INP_T]| \\
Coefficients for the implicit parts of the Crank-Nicolson scheme for thermal diffusion \verb|[COEF_INP_T]| is defined by real.

\paragraph{\tt coef\_imp\_b\_ctl}
\label{href_t:coef_imp_b_ctl}
\verb|[COEF_INP_B]| \\
Coefficients for the implicit parts of the Crank-Nicolson scheme for magnetic diffusion \verb|[COEF_INP_B]| is defined by real.

\paragraph{\tt coef\_imp\_c\_ctl}
\label{href_t:coef_imp_c_ctl}
\verb|[COEF_INP_C]| \\
Coefficients for the implicit parts of the Crank-Nicolson scheme for compositional diffusion \verb|[COEF_INP_C]| is defined by real. 


\paragraph{\tt FFT\_library\_ctl}
\label{href_t:FFT_library_ctl}
\verb|[FFT_Name]| \\
FFT library name for Fourier transform is defined by text. The following libraries are available for \verb|[FFT_Name]|. 
If this flag is not defined, program searches the fastest library in the initialization process.
%
\begin{description}
\item{\tt FFTW}		Use FFTW
\item{\tt FFTPACK}	Use FFTPACK
% \item{\tt ISPACK}	Use ISPACK
\end{description}
%

\paragraph{\tt Legendre\_trans\_loop\_ctl}
\label{href_t:Legendre_trans_loop_ctl}
\verb|[FFT_Name]| \\
Loop configuration for Legendre transform is defined by text. The following settings are available for \verb|[Leg_Loop]|. 
If this flag is not defined, program searches the fastest approarch in the initialization process.
%
\begin{description}
\item{\tt Inner\_radial\_loop}	Loop for the radial grids is set as the innermost loop
\item{\tt Outer\_radial\_loop}	Loop for the radial grids is set as the outermost loop
\item{\tt Long\_loop}	        Long one-dimentional loop is used
\end{description}
%

%
\subsection{\tt sph\_monitor\_ctl}
\label{href_t:sph_monitor_ctl}
Monitoring data is defined in this block. Monitoring data output (mean square, average, Gauss coefficients, or specific components of spectrum data) are flagged by {\tt Monitor\_On} in \hyperref[href_t:nod_value_ctl]{ {\tt nod\_value\_ctl} array}. \\
\hyperref[href_i:sph_monitor_ctl]{(Back to {\tt control\_MHD})}

\paragraph{\tt volume\_average\_prefix}
\label{href_t:volume_average_prefix}
\verb|[vol_ave_prefix]| \\
File prefix for volume average data \verb|[vol_ave_prefix]| is defined by Text. Program add {\tt .dat} or {\tt .dat.gz} extension after this file prefix. If this file prefix is not defined, volume average data are not generated. 

\paragraph{\tt volume\_pwr\_spectr\_prefix}
\label{href_t:volume_pwr_spectr_prefix}
\verb|[vol_pwr_prefix]| \\
File prefix for mean square spectrum data averaged over the fluid shell \verb|[vol_pwr_prefix]| is defined by Text. 

Spectrum as a function of degree {l} is written in \verb|[vol_pwr_prefix])_l.dat|, spectrum as a function of order {m} is written in \verb|[vol_pwr_prefix]_m.dat|, and spectrum as a function of $(l-m)$ is written in \verb|[vol_pwr_prefix]_lm.dat|. This prefix is also used for the file name of the volume mean square data as \verb|[vol_pwr_prefix]_s.dat|.
If this file prefix is not defined, volume spectrum data are not generated and volume mean square data is written as \verb|sph_pwr_volume_s.dat|.

\paragraph{\tt volume\_pwr\_spectr\_format}
\label{href_t:volume_pwr_spectr_format}
\verb|[file_format]| \\
File format for mean square spectrum data averaged over the fluid shell. If \verb|[file_format]| is \verb|gzip|, volume average, mean square and spectrum data are compressed by zlib.  If the mean square data file exist, this setting is ignored and append data in the existing file.

\paragraph{\tt nusselt\_number\_prefix}
\label{href_t:nusselt_number_prefix}
\verb|[nusselt_number_prefix]| \\
File prefix for Nusselt number data at ICB and CMB \verb|[nusselt_number_prefix]| is defined by Text. Program add {\tt .dat} or {\tt .dat.gz} extension after this file prefix. If this file prefix is not defined, Nusselt number data are not generated. \\
{\bf CAUTION: Nusselt number is not evaluated if heat source exsists.}

\paragraph{\tt nusselt\_number\_format}
\label{href_t:nusselt_number_format}
\verb|[file_format]| \\
File format for Nusselt number data \verb|[nusselt_number_prefix].dat|. If \verb|[file_format]| is \verb|gzip|, Nusselt number data is compressed by zlib.  If the Nusselt number data file exist, this setting is ignored and append data in the existing file.

%
\subsubsection{\tt volume\_spectrum\_ctl}
\label{href_t:volume_spectrum_ctl}
Volume average of power spectrum and mean square data between any radius range are defined in this block.

\paragraph{\tt degree\_spectra\_switch}
\label{href_t:degree_spectra_switch}
\verb|[ON/OFF]| \\
Switch to output the power spectrum data as a function of spherical harmonic degree $l$. Default value is {\tt ON}.

\paragraph{\tt order\_spectra\_switch}
\label{href_t:order_spectra_switch}
\verb|[ON/OFF]| \\
Switch to output the power spectrum data as a function of spherical harmonic order $m$. Default value is {\tt ON}.

\paragraph{\tt diff\_lm\_spectra\_switch}
\label{href_t:diff_lm_spectra_switch}
\verb|[ON/OFF]| \\
Switch to output the power spectrum data as a function of difference of spherical harmonic order from degree $l-m$. Default value is {\tt ON}.

\paragraph{\tt axisymmetric\_power\_switch}
\label{href_t:axisymmetric_power_switch}
\verb|[ON/OFF]| \\
Switch to output the mean square data for the axisymmetric component. Default value is {\tt ON}.

\paragraph{\tt Inner\_radius\_ctl}
\label{href_t:inner_radius_ctl}
\verb|[radius]| \\
Inner boundary of the volume average \verb|[radius]| is defined. The closest radial grid point is chosen as a inner boundary of averaging.

\paragraph{\tt outer\_radius\_ctl}
\label{href_t:outer_radius_ctl}
\verb|[radius]| \\
Outer boundary of the volume average \verb|[radius]| is defined. The closest radial grid point is chosen as the outer boundary of averaging.

%
\subsubsection{\tt layered\_spectrum\_ctl}
\label{href_t:layered_spectrum_ctl}
Sphere average of power spectrum and mean square data are defined in this block.

\paragraph{\tt layered\_pwr\_spectr\_prefix}
\label{href_t:layered_pwr_spectr_prefix}
\verb|[layer_pwr_prefix]| \\
File prefix for mean square spectrum data averaged over each sphere surface \verb|[layer_pwr_prefix]| is defined by Text.

Spectrum as a function of degree {l} is written in \verb|[layer_pwr_prefix]_l.dat|, spectrum as a function of order {m} is written in \verb|[layer_pwr_prefix]_m.dat|, and spectrum as a function of $(l-m)$ is written in \verb|[layer_pwr_prefix]_lm.dat|. If this file prefix is not defined, sphere averaged spectrum data are not generated. 

\paragraph{\tt layered\_pwr\_spectr\_format}
\label{href_t:layered_pwr_spectr_format}
\verb|[file_format]| \\
File format for mean square spectrum data averaged over the fluid shell. If \verb|[file_format]| is \verb|gzip|, the mean square and spectrum data are compressed by zlib. Otherwise, data file is written by text format. If the mean square data file exist, this setting is ignored and append data in the existing file.

\paragraph{\tt array spectr\_radius\_ctl}
\label{href_t:spectr_radius_ctl}
\verb|[Radius]|
List of radii \verb|[Radius]| to output power spectrum data by real value.
\paragraph{\tt array spectr\_layer\_ctl}
\label{href_t:spectr_layer_ctl}
\verb|[Layer #]|
List of radial grid point number \verb|[Layer #]| to output power spectrum data by integer. If this array nor array {\tt spectr\_radius\_ctl} is not defined, layered mean square data are written for all radial grid points.

%
\subsubsection{\tt gauss\_coefficient\_ctl}
\label{href_t:gauss_coefficient_ctl}
Gauss coefficients data at specified radius are defined in this block.

\paragraph{\tt gauss\_coefs\_prefix}
\label{href_t:gauss_coefs_prefix}
\verb|[gauss_coef_prefix]| \\
File prefix for Gauss coefficients \verb|[gauss_coef_prefix]| is defined by Text. Program add {\tt .dat} or {\tt .dat.gz} extension after this file prefix. If this file prefix is not defined, Gauss coefficients data are not generated. 

\paragraph{\tt gauss\_coefs\_format}
\label{href_t:gauss_coefs_format}
\verb|[file_format]| \\
File format for the Gauss coefficients data file. If \verb|[file_format]| is \verb|gzip|, volume mean square and spectrum data is compressed by zlib. Otherwise, data file is written by text format. If the Gauss coefficients data file exist, this setting is ignored and append data in the existing file.

\paragraph{\tt gauss\_coefs\_radius\_ctl}
\label{href_t:gauss_coefs_radius_ctl}
\verb|[gauss_coef_radius]| \\
Normalized radius to obtain Gauss coefficients \verb|[gauss_coef_radius]| is defined by real. Gauss coefficients are evaluated from the poloidal magnetic field at CMB by assuming electrically insulated mantle. Do not set \verb|[gauss_coef_radius]| less than the outer core radius $r_{o}$.

\paragraph{\tt array pick\_gauss\_coefs\_ctl}
\label{href_t:pick_gauss_coefs_ctl}
\verb|[Degree]  [Order]| \\
List of spherical harmonics mode $l$ and $m$ of Gauss coefficients to output. \verb|[Degree]| and \verb| [Order]| are defined by integer.

\paragraph{\tt array pick\_gauss\_coef\_degree\_ctl}
\label{href_t:pick_gauss_coef_degree_ctl}
\verb|[Degree]| \\
Degrees $l$ to output Gauss coefficients are listed in \verb|[Degree]| by integer. All Gauss coefficients with listed $l$ is output in file.

\paragraph{\tt array pick\_gauss\_coef\_order\_ctl}
\label{href_t:pick_gauss_coef_order_ctl}
\verb|[Order]| \\
Orders $m$ to output Gauss coefficients are listed in \verb|[Order]| by integer. All Gauss coefficients with listed order $m$ is output in file.

%
\subsubsection{\tt pickup\_spectr\_ctl}
\label{href_t:pickup_spectr_ctl}
Spherical harmonic coefficients data output is defined in this block.

\paragraph{\tt picked\_sph\_prefix}
\label{href_t:picked_sph_prefix}
\verb|[picked_sph_prefix]| \\
File prefix for picked spectrum data files \verb|[picked_sph_prefix]| is defined by Text. Program add {\tt [picked\_sph\_prefix]\_l[degree]\_m[order][c/s].dat} extension after this file prefix and save each mode of spherical harmonics coefficients data file. If this file prefix is not defined, picked spectrum data are not generated. Because data files are generated for each spherical harmonics mode, we recommend to save these files in a sub directory.

\paragraph{\tt picked\_sph\_format}
\label{href_t:picked_sph_format}
\verb|[file_format]| \\
File format for picked spectrum data files. If \verb|[file_format]| is \verb|gzip|, picked spectrum data files are compressed by zlib. Otherwise, data files are written by text format. If the picked spectrum data files exist, this setting is ignored and  data is appended in the existing file.

\paragraph{\tt array pick\_radius\_ctl}
\label{href_t:pick_radius_ctl}
\verb|[Layer #]|
List of radii \verb|[Radius]| to output picked spectrum data by integer.
\paragraph{\tt array pick\_layer\_ctl}
\label{href_t:pick_layer_ctl}
\verb|[Layer #]|
List of radial grid point number \verb|[Layer #]| to output picked spectrum data by integer. If this array nor array {\tt array pick\_radius\_ctl} is not defined, picked spectrum data are written for all radial grid points.

\paragraph{\tt array pick\_sph\_spectr\_ctl}
\label{href_t:pick_sph_spectr_ctl}
\verb|[Degree]  [Order]| \\
List of spherical harmonics mode $l$ and $m$ of spectrum data to output. \verb|[Degree]| and \verb| [Order]| are defined by integer.

\paragraph{\tt array pick\_sph\_degree\_ctl}
\label{href_t:pick_sph_degree_ctl}
\verb|[Degree]| \\
Degrees $l$ to output spectrum data are listed in \verb|[Degree]| by integer. All spectrum data with listed degree $l$ is output in file.

\paragraph{\tt array pick\_sph\_order\_ctl}
\label{href_t:pick_sph_order_ctl}
\verb|[Order]| \\
Order $m$ to output spectrum data are listed in \verb|[Order]| by integer. All spectrum data with listed order $m$ is output in file.

%
\subsubsection{\tt sph\_dipolarity\_ctl}
\label{href_t:sph_dipolarity_ctl}
Dipolrity data output is defined in this block.

\paragraph{\tt dipolarity\_file\_prefix}
\label{href_t:dipolarity_file_prefix}
\verb|[dipolarity_file_prefix]| \\
File prefix for dipolarity data files \verb|[dipolarity_file_prefix]| is defined by Text. Program add {\tt .dat} or {\tt .dat.gz} extension after this file prefix and save dipolarity data. If this file prefix is not defined, dipolarity data are not generated.

\paragraph{\tt dipolarity\_file\_format}
\label{href_t:dipolarity_file_format}
\verb|[file_format]| \\
File format for dipolarity data files. If \verb|[file_format]| is \verb|gzip|, dipolarity data files are compressed by zlib. Otherwise, the data file is written by text format. If the dipolrity data file exist, this setting is ignored and data is appended in the existing file.

\paragraph{\tt array dipolarity\_truncation\_ctl}
\label{href_t:dipolarity_truncation_ctl}
\verb|[Degree]| \\
Truncation degrees $l$ to evaluate dipolarity is defined in \verb|[Degree]| by integer. More than 1 truncations can be defined in this array. The dipolarity with the truncation degree of the simulation $L_{max}$ is automatically added in the dta output.
%

\subsubsection{\tt dynamo\_benchmark\_data\_ctl}
\label{href_t:dynamo_benchmark_data_ctl}
Parameters to generate dynamo benchmark data.

\paragraph{\tt dynamo\_benchmark\_file\_prefix}
\label{href_t:dynamo_benchmark_file_prefix}
\verb|[File\_Prefix]| \\
File prefix for dynamo benchmark data output.

\paragraph{\tt dynamo\_benchmark\_file\_format}
\label{href_t:dynamo_benchmark_file_format}
\verb|[ASCII or gzip]| \\
File format of the data files. {\tt ASCII} or {\tt gzip} are available.

\paragraph{\tt nphi\_mid\_eq\_ctl}
\label{href_t:nphi_mid_eq_ctl}
\verb|[Nphi_mid_equator]| \\
Number of grid points \verb|[Nphi_mid_equator]|in longitudinal direction to evaluate mid-depth of the shell in the equatorial plane for dynamo benchmark is defined as integer. If \verb|[Nphi_mid_equator]| is not defined or less than zero, \verb|[Nphi_mid_equator]| is set set number grid as the input spherical transform data. 
%

\subsubsection{\tt fields\_on\_circle\_ctl}
\label{href_t:fields_on_circle_ctl}
Parameters to generate data at along with circle

\paragraph{\tt field\_on\_circle\_prefix}
\label{href_t:field_on_circle_prefix}
\verb|[File\_Prefix]| \\
File prefix for field data on the circle as a function of the longitude.

\paragraph{\tt spectr\_on\_circle\_prefix}
\label{href_t:spectr_on_circle_prefix}
\verb|[File\_Prefix]| \\
File prefix for field data on the circle as a function of the longitude.

\paragraph{\tt field\_on\_circle\_format}
\label{href_t:field_on_circle_format}
\verb|[ASCII or gzip]| \\
File format of the data files. {\tt ASCII} or {\tt gzip} are available.

\paragraph{\tt pick\_circle\_coord\_ctl}
\label{href_t:pick_circle_coord_ctl}
\verb|[Coordinate]| \\
Coordinate system for veator fields are defeined by {\tt spherical} or {\tt cylindrical}.

\paragraph{\tt pick\_cylindrical\_radius\_ctl}
\label{href_t:pick_cylindrical_radius_ctl}
\verb|[S]| \\
cylinderical radius of the circle {\tt [S]} is defined as a real value.

\paragraph{\tt pick\_vertical\_position\_ctl}
\label{href_t:pick_vertical_position_ctl}
\verb|[Z]| \\
Vertical position of the circle {\tt [Z]} is defined as a real value.

%

\subsection{\tt visual\_control}
\label{href_t:visual_control}
Visualization modules are defined in this block. Parameters for cross sections and isosurfaces are defined in this block. \\
\hyperref[href_i:visual_control]{(Back to {\tt visual\_control})}

%
%
\subsection{\tt cross\_section\_ctl}
\label{href_t:cross_section_ctl}
Control parameters for cross sectioning are defined in this block. \\
\hyperref[href_i:cross_section_ctl]{(Back to {\tt cross\_section\_ctl)}}

\paragraph{\tt section\_file\_prefix}
\label{href_t:section_file_prefix}
\verb|[file_prefix]| \\
File prefix for cross section data is defined as character \verb|[file_prefix]|.

\paragraph{\tt psf\_output\_type}
\label{href_t:psf_output_type}
\verb|[file_format]| \\
File format for cross section data is defined as character \verb|[file_format]|. The following formats are available;
\begin{description}
\item{\tt VTK: }               VTK format
\item{\tt VTK\_gz: }           Compressed VTK format (Available if zlib library is linked)
\item{\tt PSF: }               Binary section data format
\item{\tt PSF\_gzip: }         Compressed Binary section data format (Available if zlib library is linked)
\end{description}

\subsubsection{\tt surface\_define}
\label{href_t:surface_define}
Each cross section is defined in this block. \\
\hyperref[href_i:cross_section_ctl]{(Back to {\tt cross\_section\_ctl)}} \\

\paragraph{\tt section\_method}
\label{href_t:section_method}
\verb|[METHOD]| \\
Method of the cross sectioning is defined as character \verb|[METHOD]|. Supported cross section is shown in Table \ref{table:surface_list}
%
\begin{table}[htp]
\caption{Supported cross sections}
\begin{center}
\begin{tabular}{|c|c|}
\hline
\verb|[METHOD]| & Surface type \\ \hline
\verb|equation| & Quadrature surface \\
% & $a x^2 + b y^2 + c z^2 + d y x + e z x + f x y + g x + h y + j z + k = 0$ \\
\verb|plane| & Plane surface \\
%& $a \left(x-x_{0} \right) + b \left(y-y_{0} \right) + c \left(z-z_{0} \right) = 0$ \\
\verb|sphere| & Sphere \\
%& $\left(x-x_{0} \right)^2 + \left(y-y_{0} \right)^2 + \left(z-z_{0} \right)^2 = r^2$  \\
\verb|ellipsoid| & Ellipsoid  \\
%& $\left(\frac{x-x_{0}}{a} \right)^2 + \left( \frac{y-y_{0}}{b} \right)^2 + \left( \frac{z-z_{0}}{c} \right)^2 = 1$ \\
\hline
\end{tabular}
\end{center}
\label{table:surface_list}
\end{table}
%

\paragraph{\tt coefs\_ctl}
\label{href_t:psf_coefs_ctl}
\verb|[TERM]	[COEFFICIENT]| \\
This array defines coefficients for a quadrature surface described by 
\begin{eqnarray*}
a x^2 + b y^2 + c z^2 + d y z + e z x + f x y + g x + h y + j z + k &=& 0.
\end{eqnarray*}
Each coefficient $a$ to $k$ are defined by the name of the term \verb|[TERM]| and real value \verb|[COEFFICIENT]| as shown in Table \ref{table:psf_coefs}.
%
\begin{table}[htp]
\caption{List of coefficient labels for quadrature surface}
\begin{center}
\begin{tabular}{|c|c||c|c||c|c|}
\hline
\verb|[TERM]| & Defined value & \verb|[TERM]| & Defined value & \verb|[TERM]| & Defined value \\ \hline
\verb|x2| & $a$ & \verb|y2| & $b$  & \verb|z2| & $c$ \\
\verb|yz| & $d$ & \verb|zx| & $e$  & \verb|xy| & $f$ \\
\verb|x | & $g$ & \verb|y| & $h$  & \verb|z| & $i$ \\
\verb|const| & $h$ &  &   & &  \\ \hline
\end{tabular}
\end{center}
\label{table:psf_coefs}
\end{table}
%

\paragraph{\tt radius}
\label{href_t:psf_radius}
\verb|[SIZE]| \\
\verb|[SIZE]| defines radius $r$ for a sphere surface defined by 
\begin{eqnarray*}
\left(x-x_{0} \right)^2 + \left(y-y_{0} \right)^2 + \left(z-z_{0} \right)^2 = r^2. 
\end{eqnarray*}

\paragraph{\tt normal\_vector}
\label{href_t:psf_normal_vector}
\verb|[DIRECTION]	[COMPONENT]| \\
This array defines normal vector $(a, b, c)$ for a plane surface described by 
\begin{eqnarray*}
a \left(x-x_{0} \right) + b \left(y-y_{0} \right) + c \left(z-z_{0} \right) = 0. 
\end{eqnarray*}
Each component is defined by \verb|[DIRECTION]| and real value \verb|[COMPONENT]| as shown in Table \ref{table:psf_normal}.
%
\begin{table}[htp]
\caption{List of coefficient labels for vector}
\begin{center}
\begin{tabular}{|c|c|}
\hline
\verb|[DIRECTION]| & Defined value \\ \hline
\verb|x| & $a$ \\
\verb|y| & $b$ \\
\verb|z| & $c$ \\ \hline
\end{tabular}
\end{center}
\label{table:psf_normal}
\end{table}
%

\paragraph{\tt axial\_length}
\label{href_t:psf_axial_length}
\verb|[DIRECTION]	[COMPONENT]| \\
This array defines size $(a, b, c)$ of an ellipsoid surface described by 
\begin{eqnarray*}
\left(\frac{x-x_{0}}{a} \right)^2 + \left( \frac{y-y_{0}}{b} \right)^2 + \left( \frac{z-z_{0}}{c} \right)^2 = 1. 
\end{eqnarray*}
Each component is defined by \verb|[DIRECTION]| and real value \verb|[COMPONENT]| as shown in Table \ref{table:psf_normal}.
%

\paragraph{\tt center\_position}
\label{href_t:psf_center_position}
\verb|[DIRECTION]	[COMPONENT]| \\
Position of center $(x_{0}, y_{0}, z_{0})$ of sphere or ellipsoid is defined this array. Position on a plane surface $(x_{0}, y_{0}, z_{0})$ is also defined. Each component is defined by \verb|[DIRECTION]| and real value \verb|[COMPONENT]| as shown in Table \ref{table:psf_position}.
%
\begin{table}[htp]
\caption{List of coefficient labels for vector}
\begin{center}
\begin{tabular}{|c|c|}
\hline
\verb|[DIRECTION]| & Defined value \\ \hline
\verb|x| & $x_{0}$ \\
\verb|y| & $y_{0}$ \\
\verb|z| & $z_{0}$ \\ \hline
\end{tabular}
\end{center}
\label{table:psf_position}
\end{table}
%

\paragraph{\tt section\_area\_ctl}
\label{href_t:section_area_ctl}
Areas for the cross sectioning are defined in this array. The following groups can be defined in this block.
%
\begin{description}
	\item{\tt outer\_core} Outer core.
	\item{\tt inner\_core} Inner core (If exist).
	\item{\tt external} External of the core (If exist).
	\item{\tt all} Whole simulation domain.
\end{description}

%
\subsubsection{\tt output\_field\_define}
\label{href_t:output_field_define}
Field data on the cross section are defined in this block. \\
\hyperref[href_i:cross_section_ctl]{(Back to {\tt cross\_section\_ctl)}} \\

%
\paragraph{\tt output\_field}
\label{href_t:psf_output_field}
Field informations for cross section are defined in this array. Name of the output fields is defined by \verb|[FIELD]|, and component of the fields is defined by \verb|[COMPONENT]|. Labels of the field name are listed in Table \ref{table:fields}, and labels of the component are listed in Table \ref{table:components}. \\
%
\begin{table}[htp]
\caption{List of field type for cross sectioning and isosurface module}
\label{table:components}
\begin{center} 
\begin{tabular}{|c|c|}
\hline
 \verb|[COMPONENT]| & Field type  \\ \hline \hline
 \verb|scalar| & scalar field  \\ \hline
 \verb|vector| & Cartesian vector field \\ \hline
 \verb|x| & $x$-component  \\ \hline
 \verb|y| & $y$-component  \\ \hline
 \verb|z| & $z$-component  \\ \hline
 \verb|radial| & radial ($r$-) component  \\ \hline
 \verb|theta| & $\theta$-component  \\ \hline
 \verb|phi| & $\phi$-component  \\ \hline
 \verb|cylinder_r| & cylindrical radial ($s$-) component  \\ \hline
 \verb|magnitude| & magnitude of vector  \\ \hline
\end{tabular}
\end{center}
\end{table}
%
%


\subsection{\tt isosurface\_ctl}
\label{href_t:isosurface_ctl}
Control parameters for isosurfacing are defined in this block. \\
\hyperref[href_i:isosurface_ctl]{(Back to {\tt isosurface\_ctl)}}

%
%
\paragraph{\tt isosurface\_file\_prefix}
\label{href_t:isosurface_file_prefix}
\verb|[file_prefix]| \\
File prefix for isosurface data is defined as character \verb|[file_prefix]|.

\paragraph{\tt iso\_output\_type}
\label{href_t:iso_output_type}
File format for isosurface data is defined as character \verb|[file_format]|. The following formats are available;
\begin{description}
\item{\tt VTK: }               VTK format
\item{\tt VTK\_gz: }           Compressed VTK format (Available if zlib library is linked)
\item{\tt ISO: }               Binary isosurface data format
\item{\tt ISO\_gzip: }         Compressed Binary isosurface data format (Available if zlib library is linked)
\end{description}

\subsubsection{\tt isosurf\_define}
\label{href_t:isosurf_define}
Each isosurface is defined in this block. \\
\hyperref[href_i:isosurface_ctl]{(Back to {\tt isosurface\_ctl)}}

\paragraph{\tt isosurf\_field}
\label{href_t:isosurf_field}
Field name for isosurface is defined by \verb|[FIELD]|. Labels of the field name are listed in Table \ref{table:fields}. \\
%
\paragraph{\tt isosurf\_component}
\label{href_t:isosurf_component}
Component name for isosurface is defined by \verb|[COMPONENT]|. Labels of the component are listed in Table \ref{table:components}.

%
\paragraph{\tt isosurf\_value}
\label{href_t:isosurf_value}
Isosurface value is defined as real value \verb|VALUE|.

\paragraph{\tt isosurf\_area\_ctl}
\label{href_t:isosurf_area_ctl}
Areas for the isosurfacing are defined in this array. The same groups can be defined as \hyperref[href_t:psf_output_field]{\tt section\_area\_ctl}.

%
\subsubsection{\tt field\_on\_isosurf}
\label{href_t:field_on_isosurf}
Field data on the isosurface are defined in this block. \\
\hyperref[href_i:isosurface_ctl]{(Back to {\tt isosurface\_ctl)}}

%
\paragraph{\tt result\_type}
\label{href_t:result_type}
Output data type is defined by \verb|[TYPE]|. Following types can be defined:
%
\begin{description}
	\item{\tt constant} Constant value is set as a result field. The amplitude is set by \verb|result_value|.
	\item{\tt field} field data on the isosurface are written. Fields to be written are defined by \verb|output_field| array.
\end{description}

%
\paragraph{\tt result\_value}
\label{href_t:result_value}
Isosurface value is defined as real value \verb|VALUE|.

%
\paragraph{\tt output\_field}
\label{href_t:iso_output_field}
Field informations for cross section are defined in this array. Name of the output fields is defined by \verb|[FIELD]|, and component of the fields is defined by \verb|[COMPONENT]|. Labels of the field name are listed in Table \ref{table:fields}, and labels of the component are listed in Table \ref{table:components}. \\
%
%
\subsection{\tt output\_field\_file\_fmt\_ctl  [VTK\_format]}
\label{href_t:output_field_file_fmt_ctl}
File format of field data is defined as character \verb|[VTK_format]|. THe following formats are available.
%
\begin{description}
\item{\tt single\_HDF5: }  Merged HDF5 file (Available if HDF5 library is linked)
\item{\tt single\_VTK: }   Merged VTK file (Default)
\item{\tt VTK: }           Distributed VTK file
\item{\tt single\_VTK\_gz: }   Compressed merged VTK file (Available if zlib library is linked)
\item{\tt VTK\_gz: }           Compressed distributed VTK file (Available if zlib library is linked)
\end{description}

\subsection{\tt dynamo\_vizs\_control}
\label{href_t:dynamo_vizs_control}
Visualization for zonal mean, RMS, and truncated magnetic field are defined in this block. Parameters for cross section is set for zonal mean and RMS, and spherical harmonics degree of the truncated magnetic field is also defined here. \\
\hyperref[href_i:dynamo_vizs_control]{(Back to {\tt dynamo\_vizs\_control)}}

%
%
\paragraph{\tt zonal\_mean\_section\_ctl}
\label{href_t:zonal_mean_section_ctl}
Control parameters for cross section of the zonal mean field are defined in this block. This block has the same control items as \hyperref[href_i:cross_section_ctl]{\tt cross\_section\_ctl}. In the external file {\tt [zonal\_mean\_section\_control\_file]}, con trol block starts from {\tt cross\_section\_ctl}.


\paragraph{\tt zonal\_RMS\_section\_ctl}
\label{href_t:zonal_RMS_section_ctl}
Control parameters for cross section of the zonal RMS field are defined in this block. This block has the same control items as \hyperref[href_i:cross_section_ctl]{\tt cross\_section\_ctl}. In the external file {\tt [zonal\_RMS\_section\_control\_file]}, con trol block starts from {\tt cross\_section\_ctl}.

\paragraph{\tt crustal\_filtering\_ctl}
\label{href_t:crustal_filtering_ctl}
Set the truncation degree to make the truncated magnetic field by the crustal magnetic field. The spherical harmonics degree of the truncated magnetic field is defined in {\tt truncation\_degree\_ctl}. In the external file {\tt [zonal\_mean\_section\_control\_file]}, con trol block starts from {\tt cross\_section\_ctl}.

%

\subsection{\tt new\_data\_files\_def}
\label{href_t:new_data_files_def}
File names and number of processes for new domain decomposed data are defined in this block. \\
\hyperref[href_i:new_data_files_def]{(Back to {\tt control\_assemble\_sph)}}

\paragraph{\tt delete\_original\_data\_flag}
\label{href_t:delete_original_data_flag}
\verb|[delete_original_data_flag]| \\
If this flag set to \verb|YES|, original specter data is deleted at the end of program. 

\subsection{\tt new\_time\_step\_ctl}
\label{href_t:new_time_step_ctl}
Parameters to modify time step and time data in the new restart file. \\
\hyperref[href_i:new_time_step_ctl]{(Back to {\tt control\_assemble\_sph)}}

\paragraph{\tt magnetic\_field\_ratio\_ctl}
\label{href_t:i_step_init_ctl_a} 
\verb|[ISTEP_START]| \\
New time step \verb|[ISTEP_START]| for the restart file is defined by integer.

\paragraph{\tt i\_step\_rst\_ctl}
\label{href_t:i_step_rst_ctl_a} 
\verb|[ISTEP_RESTART]| \\
New step number of restrart file \verb|[ISTEP_RESTART]| is defined by integer.

\paragraph{\tt time\_init\_ctl}
\label{href_t:time_init_ctl_a} 
\verb|[INITIAL_TIME]| \\
New time data \verb|[INITIAL_TIME]| is defined by real.


\subsection{\tt newrst\_magne\_ctl}
\label{href_t:newrst_magne_ctl}
Parameters to modify magnetic field are defined in this block. \\
\hyperref[href_i:newrst_magne_ctl]{(Back to {\tt control\_assemble\_sph)}}

\paragraph{\tt magnetic\_field\_ratio\_ctl}
\label{href_t:magnetic_field_ratio_ctl} 
\verb|[ratio]| \\
Ratio of new magnetic field data to original magnetic field \verb|[ratio]| is defined by real.


\subsection{\tt time\_averaging\_sph\_monitor}
\label{href_t:time_averaging_sph_monitor}
Parameters for time averaging of monitor data files.

\paragraph{\tt start\_time\_ctl}
\label{href_t:tave_start_time_ctl} 
\verb|[TIME]| \\
Start  time \verb|[TIME]| for the restart file is defined by real.

\paragraph{\tt end\_time\_ctl}
\label{href_t:tave_end_time_ctl} 
\verb|[TIME]| \\
End  time \verb|[TIME]| for the restart file is defined by real.

\paragraph{\tt volume\_integrate\_prefix}
\label{href_t:volume_integrate_prefix} 
\verb|[File\_Prefix]| \\
File prefix  \verb|[File\_Prefix]|  for  volume integrated data is defined by text.

\paragraph{\tt volume\_sph\_spectr\_prefix}
\label{href_t:volume_sph_spectr_prefix} 
\verb|[File\_Prefix]| \\
File prefix  \verb|[File\_Prefix]|  for  volume integrated spectrum data is defined by text.

\paragraph{\tt sphere\_integrate\_prefix}
\label{href_t:sphere_integrate_prefix} 
\verb|[File\_Prefix]| \\
File prefix  \verb|[File\_Prefix]|  for sphere integrated data is defined by text.

\paragraph{\tt layer\_sph\_spectr\_prefix}
\label{href_t:layer_sph_spectr_prefix} 
\verb|[File\_Prefix]| \\
File prefix  \verb|[File\_Prefix]|  for  layered spectrum data is defined by text.

\paragraph{\tt picked\_sph\_prefix}
\label{href_t:picked_sph_prefix} 
\verb|[File\_Prefix]| \\
File prefix  \verb|[File\_Prefix]|  for  picked up spectrum data is defined by text.



\newpage
\input{tex_src/gpl2.tex}
\end{appendices}

\end{document}
