\newpage
\section{Installation}


\subsection{Compiler Requirements}
Most of source code of Calypso are written in Fortran2003. Consequently, Fortran compiler with supporting fortran 2003 is required. We can obtain a number of information about Fortran from \url{http://fortranwiki.org/}, and you can also find a table of the supported features of Fortran 2003 standard at \url{http://fortranwiki.org/fortran/show/Fortran+2003+status}. In addition, C compiler is optionally required to us zlib support for compressed data IO. 

\subsection{Library Requirements}
\label{sec:requirements}
Calypso requires the following libraries.
\begin{itemize}
\item GNU make
\item MPI libraries (OpenMPI, MPICH, etc)
\item FFTPACK Ver 5.1D (\url{https://people.sc.fsu.edu/~jburkardt/f_src/fftpack5.1d/fftpack5.1d.html}). The source files for FFTPACK are included in {\tt src/EXTERNAL\_libs} directory.
\end{itemize}
Linux and Max OS X use GNU make as a default 'make' command, but some system (e.g. BSD or SOLARIS) does not use GNU make as default. \verb|configure| command searches and set correct GNU make command. MPI library such as OpenMPI (\url{https://www.open-mpi.org}) or MPICH (\url{https://www.mpich.org}) can be installed by the most of package manager.

In addition, the following environment and libraries can be used (optional).
\begin{itemize}
\item OpenMP 
\item BLAS
\item zlib (https://www.zlib.net)
\item FFTW version 3 (\url{http://www.fftw.org}) including Fortran wrapper
\item PARALLEL HDF5 (\url{https://support.hdfgroup.org/HDF5/PHDF5})  including Fortran wrapper.
\end{itemize}
Note: Calypso does NOT use MPI and OpenMP features in FFTW3. 

In the most of platforms, the Fourier transform by FFTW is faster than that by FFTPACK. 

Zlib is used for compressed data IO. Zlib is installed in most of UNIX platforms.

HDF5 is used for field data output with XDMF format instead of VTK format. The comparison of field data format is described in section ref{sec:VTK}. 

OpenMP is used for the parallelization under the shared memory. Better choice to use both MPI and OpenMP parallelization (so-called Hybrid parallelization) or only using MPI (so-called flat MPI) is depends on the computational platform and compiler. For example, flat MPI has much better performance on Linux cluster with Intel Xeon processors and with Intel fortran compiler, but Hybrid model has better performance on Hitachi SR24000 with Power 8 processors.

\subsubsection{Installation of required softwares for Linux}
GCC, the GNU Compiler Collection (\url{https://gcc.gnu.org}) is already installed in the most of Linux distributions. However, packages for development are need to be installed. For Ubuntu 20, for example, the required compilers and  libraries can be installed by using \verb|apt| command as following::
%
\begin{verbatim}
% sudo apt install build-essential
% sudo apt install pkg-config
% sudo apt install git
% sudo apt install gfortran
% sudo apt install libopenmpi-dev
% sudo apt install zlib1g
% sudo apt install zlib1g-dev
% sudo apt install libblas-dev
% sudo apt install libfftw3-dev
% sudo apt install libhdf5-openmpi-dev
\end{verbatim}

 
\subsubsection{Installation of required softwares for Mac OS}
For MacOS, any fortran compiler needs to be installed because Xcode does not have fortran compiler. The easiest way is installing GCC by using a package manager such as macports (\url{https://www.macports.org}) or homebrew (\url{https://brew.sh/index}). By using the Macports, 

\paragraph{Macports}
The required compiler and packages can be installed as followings as an example using GCC11. GCC in Macports includes gfortran compiler.
%
\begin{verbatim}
% sudo port install gcc11
% sudo port install openmpi-gcc11
% sudo port install fftw-3 +gcc11 +openmpi
% sudo port install hdf5 +fortran +gcc11 +openmpi +threadsafe
\end{verbatim}

\paragraph{Homebrew}
The required compiler and packages can be installed as followings: GCC in Homebrew includes gfortran compiler.
%
\begin{verbatim}
% brew install gcc
% brew install open-mpi
% brew install fftw
% brew install lhdf5-mpi
\end{verbatim}


\subsection{Known problems}
\subsubsection*{FFTPACK and Intel compiler}
FFTPACK fails to compile with Intel fortran using the {\tt `-warn all'} option. Currently the {\tt `-warn all'} option is excluded by Makefile when FFTPACK is compiled.

\subsubsection*{XL fortran}
In XL fortran, preprocessor options is not specified by \verb|-D...|, but \verb|-Wf, '-D...'|. Please edit preprocessor macro option \verb|F90CPPFLAGS| in \verb|work/Makefile| by an editor.

\subsubsection*{Cross compiler support}
{\tt configure} command in Calypso does not support cross compilation. If you want to compile with a cross compiler, please set the variables in Makefile manually (see section \ref{section:no_configure})

\subsection{Directories}

The top directory of Calypso (ex. \verb|[CALYPSO_HOME]|) contains the following directories.
\begin{verbatim}
% cd [CALYPSO_HOME]
% ls
CMakeLists.txt	Makefile.in	configure.in	examples
INSTALL		bin		doc		src
LICENSE		configure	doxygen		work

\end{verbatim}

\begin{description}
\item{\tt bin:      } directory for executable files
\item{\tt cmake:    } directory for cmake configurations
\item{\tt cmake:    } directory for document generated by doxygen
\item{\tt doc:      } documentations
\item{\tt examples: } examples
\item{\tt src:      } source files
\item{\tt work:     } work directory. Compile is done in this directory.
\end{description}

\subsection{Doxygen}
Doxygen (\url{http://www.doxygen.org}) is an powerful document generation tool from source files. We only save a configuration file in this directory because thousands of html files generated by doxygen. The documents for source codes are generated by the following command:
% 
\begin{verbatim}
% cd [CALYPSO_HOME]/doxygen
% doxygen ./Doxyfile_CALYPSO
\end{verbatim}
%
The html documents can see by opening \verb|[CALYPSO_HOME]/doxygen/html/index.html|.  Automatically generated documentation is also available on the CIG website at \url{http://www.geodynamics.org/cig/software/calypso/}.

\subsection{Install using {\tt configure} command }
\subsubsection{Configuration using {\tt configure} command }
Calypso uses the configure script for configuration to install. The simplest way to install programs is the following process in the top directory of Calypso.
% 
\begin{verbatim}
%pwd
[CALYPSO_HOME]
% ./configure
...
% make
...
% make install
\end{verbatim}
%
After the installation, object modules can be deleted by the following command;
% 
\begin{verbatim}
% make clean
\end{verbatim}
%

{./configure} generates a Makefile in the current directory.  Available options for {\tt configure} can be checked using the {\tt ./configure --help} command. The following options are available in the {\tt configure} command.
%
{\small
\begin{verbatim}
Optional Features:
  --disable-option-checking  ignore unrecognized --enable/--with options
  --disable-FEATURE       do not include FEATURE (same as --enable-FEATURE=no)
  --enable-FEATURE[=ARG]  include FEATURE [ARG=yes]
 Optional Packages:
  --with-PACKAGE[=ARG]    use PACKAGE [ARG=yes]
  --without-PACKAGE       do not use PACKAGE (same as --with-PACKAGE=no)
  --with-fftw3            Use fftw3 library
  --with-hdf5=yes/no/PATH full path of h5pcc for parallel HDF5 configuration
  --with-blas=<lib>       use BLAS library <lib>
  --with-zlib=DIR root directory path of zlib installation defaults to
                    /usr/local or /usr if not found in /usr/local
  --without-zlib to disable zlib usage completely

Some influential environment variables:
  CC          C compiler command
  CFLAGS      C compiler flags
  LDFLAGS     linker flags, e.g. -L<lib dir> if you have libraries in a
              nonstandard directory <lib dir>
  LIBS        libraries to pass to the linker, e.g. -l<library>
  CPPFLAGS    (Objective) C/C++ preprocessor flags, e.g. -I<include dir> if
              you have headers in a nonstandard directory <include dir>
  FC          Fortran compiler command
  FCFLAGS     Fortran compiler flags
  MPICC       MPI C compiler command
  MPIFC       MPI Fortran compiler command
  PKG_CONFIG  path to pkg-config utility
  CPP         C preprocessor
  FFTW3_CFLAGS
              C compiler flags for FFTW3, overriding pkg-config
  FFTW3_LIBS  linker flags for FFTW3, overriding pkg-config
  ZLIB_CFLAGS
              C compiler flags for ZLIB, overriding pkg-config
  ZLIB_LIBS       linker flags for ZLIB, overriding pkg-config

\end{verbatim}
}
%
An example of usage of the configure command is the following;
\begin{verbatim}
% ./configure --prefix='/Users/matsui/local' \
? CFLAGS='-O -Wall -g' \
? PKG_CONFIG_PATH='/Users/matsui/local/lib/pkgconfig'
\end{verbatim}

At the end of the configuration, The following message can use to check if libraries can be refered correctly:

{\small
\begin{verbatim}
-----   Configuration summary   -------------------------------

host:        "x86_64-apple-darwin16.7.0"
host_alias:  ""
XL_FORTRAN:    ""
Use OpenMP ...        yes

Use BLAS ...          yes

Use FFTW3 ...         yes
Use parallel HDF5 ... yes

Use zlib at ...       yes

---------------------------------------------------------------
\end{verbatim}
}


\subsubsection{Compile}
Compile is performed using the {\tt make} command. The Makefile in the top directory is used to generate another Makefile in the {\tt work} directory, which is automatically used to complete the compilation. The object file and libraries are compiled in the {\tt work} directory. Finally, the executive files are assembled in {\tt bin} directory. You should find the following programs in the {\tt bin} directory as
%
\begin{verbatim}
$ pwd
/home/matsui/git/calypso/bin
$ ls -F
sph_add_initial_field*  sph_mhd*       tests/
sph_initial_field*      sph_snapshot*  utilities/
$ ls -F utilities/
assemble_sph*   make_f90depends*  t_ave_monitor_data*
field_to_VTK*   sectioning*       three_vizualizations*
gen_sph_grids*  section_to_vtk*
\end{verbatim}
%
%
\begin{description}
\item{\tt sph\_mhd:          }\\
 Simulation program
\item{\tt sph\_snapshot:     }\\
 Data transfer from spectrum data to field data
\item{\tt sph\_initial\_field: }\\
 Example program to generate initial field
\item{\tt sph\_add\_initial\_field: }\\
 Example program to add initial field in existing initial field data
\end{description}
%
 Utility and data analysis programs under {\tt bin/utilities} folder are the following:
%
\begin{description}
\item{\tt assemble\_sph:     }\\
 Data transfer program to change number of subdomains.
\item{\tt gen\_sph\_grids:    }\\
 Preprocessing program for data transfer for spherical harmonics transform
\item{\tt t\_ave\_monitor\_data:     }\\
 Time averaging program for monitor data files
\item{\tt field\_to\_VTK:   }\\
 Data transfer program from field and FEM mesh data to VTK format.
\item{\tt section\_to\_vtk:     }\\
 Data transfer program from section and isosurface data to VTK format.
\item{\tt sectioning:     }\\
 Generate cross section and isosurface from field data and FEM mesh data.
\item{\tt make\_f90depends:  }\\
 Program to generate dependency of the source code ({\tt make} command uses to generate {\tt work/Makefile})
\end{description}
%
Test programs for the subroutines in  {\tt bin/test} folder are the following:
%
\begin{description}
\item{\tt compare\_fft\_test:    }\\
	 Check program for Fourier transform with exsisting Fourier transform data. This program is used for the test procedure.
\item{\tt compare\_gauss\_points:    }\\
	 Check program for Gauss-Legendre integration points with exsisting Gauss-Legendre point data. This program is used for the test procedure.
\item{\tt compare\_psf:    }\\
	 Check program for sectioning data output with exsisting Gauss-Legendre point data. This program is used for the test procedure.
\item{\tt compare\_sph\_restart:    }\\
	 Check program for restart data output with exsisting Gauss-Legendre point data. This program is used for the test procedure.
\item{\tt sph\_ene\_check:    }\\
	 Check program for monitor data output with exsisting Gauss-Legendre point data. This program is used for the test procedure.
\item{\tt check\_sph\_grids:    }\\
	 Check program for data communication for spherical harmonics transform
\item{\tt check\_sph\_comms:   }\\
	  Check program for data communications.
\item{\tt gauss\_points:   }\\
	Check program for Gauss-Legendre integration points and its colatitude
\item{\tt test\_schmidt\_Legendre:   }\\
	 Test program for Schmidt quasi-normalized Legendre polynomials
\item{\tt test\_FFTPACK:   }\\
	Test program for Fourier transform by FFTPACK
\item{\tt test\_FFTW3\_f:   }\\
         Test program for Fourier transform by FFTW3 from fortran programs
\item{\tt test\_single\_FFTW3:   }\\
         Test program for Fourier transform by FFTW3 from fortran programs
\end{description}
%
The following library files are also made in {\tt work} directory.
%
\begin{description}
\item{\tt libcalypso.a:    } Calypso library
\item{\tt libcalypso\_c.a:    } Calypso library from C sources
\item{\tt libfftpack.5d.a: } FFTPACK 5.1 library
\end{description}
%

\subsubsection{Clean}
The object and fortran module files in {\tt work} directory is deleted by typing
\begin{verbatim}
% make clean
\end{verbatim}
This command deletes files with the extension {\tt .o}, {\tt .mod}, {\tt .par}, {\tt .diag}, and {\tt ~}.

\subsubsection{Distclean}
To revert the files and directory to the original package, use make distclean as
\begin{verbatim}
% make distclean
\end{verbatim}

\subsubsection{Install}
 The executive files are copied to the install directory \verb|$(INSTDIR)/bin|. The install directory \verb|$(INSTDIR)| is defined in Makefile, and can also set by  \verb|${--prefix}| option for \verb|configure| command. Alternatively, you can use the programs in \verb|${SRCDIR}/bin| directory without running \verb|make install|. If directory \verb|${PREFIX}| does not exist, \verb|make install | creates  \verb|${PREFIX}|,  \verb|${PREFIX}/lib|,  \verb|${PREFIX}/bin|, and  \verb|${PREFIX}/include| directories. No files are installed in \verb|${PREFIX}/lib| and \verb|${PREFIX}/include|.

\subsubsection{Construct dependecies (only for developper)}
The build process of Calypso consists of three steps:
%
\begin{enumerate}
\item Construct dependency of source files {\tt Makefile.depends} in each directory by  {\tt make depends} command.
\item Construct {\tt Makefile} in {\tt [WORKDIR]} directory by {\tt make makemake} command.
\item Move into  {\tt [WORKDIR]}  and run {\tt make} command in {\tt [WORKDIR]} directory ({\tt cd [WORKDIR]; make})
\end{enumerate}
%
gmake supports parallel build by using {\tt -j [\# of process]} option, but the step 1 and 2 do not work correctly under the parallel build, but the step 3 (build under {\tt [WORKDIR} support the parallel build. Consequently, build time can be significantly reduced by run the step 3 sepalately with {\tt -j} option. 
{\color{red} (Caution) Many large computer system does not recommend to use parallel build on the login nodes. Please check the computer center's documentations}

Fortran90 routines need to be build after modules which are used in the routines. C source files also need dependency among include files. Consequently, list of dependency of source files are saved in the file \verb|Makefile.depends| in each directory. When you modify the source files with changing the module usage,  \verb|Makefile.depends| files need to be updated. To update the  \verb|Makefile.depends|files, use the  \verb|make| command at the \verb|[CALYPSO_HOME]| directory as \\
%
\begin{verbatim}
% make depends
\end{verbatim}

This process generate dependencies of the Fortran modules by program \verb|make_f90depends|. For C source files, the dependency is generated by the gcc with \verb|-MM -w -DDEPENDENCY_CHECK| option. Consequently, the dependencies need to be generated by the environment with gcc or compatible compiler. After generating the dependency, you can transfer the modified package and build without using gcc.

\subsection{Install without using configure}
\label{section:no_configure}
It is possible to compile Calypso without using the \verb|configure| command. To do this, you need to edit the \verb|Makefile|. First, copy \verb|Makefile| from template \verb|Makefile.in| as
%
\begin{verbatim}
% cp Makefile.in Makefile
\end{verbatim}
In Makefile, the following variables should be defined.
%
\begin{description}
\item{\verb|SHELL|}    Name of shell command.
\item{\verb|SRCDIR|}   Directory of this Makefile. 
\item{\verb|INSTDIR|}  Install directory.
\item{\verb|MPICHDIR|} Directory names for MPI implementation. If you set fortran90 compiler name for MPI programs in \verb|MPIF90|, you do not need to define this valuable.
\item{\verb|MPICHINCDIR|} Directory names for include files for MPI implementation. If you set fortran90 compiler name for MPI programs in \verb|MPIF90|, you do not need to define this valuable.
\item{\verb|MPILIBS|}   Library names for MPI implementation. If you set fortran90 compiler name for MPI programs in \verb|MPIF90|, you do not need to define this valuable.
\item{\verb|F90_LOCAL|} Command name of local Fortran 90 compiler to compile module dependency listing program.
\item{\verb|MPIF90|} Command name of Fortran90 compiler and linker for MPI programs. If command does not have MPI implementation, you need to define the definition of MPI libraries \verb|MPICHDIR|, \verb|MPICHINCDIR|, and \verb|MPILIBS|.
\item{\verb|AR|}     Command name for archive program (ex. \verb|ar|) to generate libraries. If you need some options for archive command, options are also included in this valuable.
\item{\verb|RANLIB|} Command name for \verb|ranlib| to generate index to the contents of an archive. If system does not have \verb|ranlib|, set \verb|true| in this valuable. \verb|true| command does not do anything for libraries.
\item{}
\item{\verb|F90OPTFLAGS|}  Optimization flags for Fortran90 compiler (including OpenMP flags)
\item{\verb|BLAS_LIBS|} Library lists for BLAS  (ex.  \verb|-lblas|)
\item{\verb|ZLIB_CFLAGS|} Option flags for zlib  (ex.  \verb|-I/usr/include|)
\item{\verb|ZLIB_LIB|}   Library lists for zlib (ex. \verb|-L/usr/lib -lz|)
\item{\verb|FFTW3_CFLAGS|} Option flags for FFTW3  (ex.  \verb|-I/usr/local/include|)
\item{\verb|FFTW3_LIBS|}   Library lists for FFTW3 (ex. \verb|-L/usr/local/lib -lfftw3 -lm|)
\item{\verb|HDF5_FFLAGS|}  Option flags to compile with HDF5. This setting can be found by using hfd5 command \verb|h5pfc -show|.

\item{\verb|HDF5_LDFLAGS|}    Option flags  to link with  HDF5. This setting can be found by using hfd5 command \verb|h5pfc -show|.

\item{\verb|HDF5_FLIBS|}   Library lists for HDF5. This setting can be found by using hfd5 command \verb|h5pfc -show|.

\end{description}
%
